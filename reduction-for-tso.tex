% !TEX root = main-tsoreduction.tex
\section{Reduction for TSO}
\label{sec:reduction-for-tso}
In this section, we explain how the reduction theory of Lipton can be used for TSO.
Our goal is to present sufficient conditions for programs such that when a program satisfies these conditions the program is guaranteed to be unable to distinguish TSO semantics from SC semantics. 
This is the first step of applying reduction to TSO programs.
In the following sections, we will show how we can extend it to programs which are initially TSO-specific.

\begin{definition}[Movers]
\label{def:movers}
Let $P$ be a labelled program and let $\aliequivgeneric$ be an equivalence relation over $\alirunsx {tso} P$.
Let $s$ be a statement that occurs in some method's body $m$ of $P$.
Then, $s$ is {\em left-mover in $\aliequivgeneric$} if for any $\alisequencex r=\alisequencex r[1]\ldots\alisequencex r[k]\in\alirunsx {tso} P$ we have $\alisequencex r[i]\alisequencex r[i+1]=(R',t':s')(R,t:s)$ such that $\alimemtracex {s'} \alipotsox {\alimemtracex {\alisequencex r}} \alimemtracex s$ does not hold, then there exists a run $\alisequencex {r'}$ in $[\alisequencex r]_{\aliequivgeneric}$ such that $\alisequencex {r'}\langle 1,i\rangle=\alisequencex r\langle 1,i-1\rangle \alisequencex r[i+1]$.

Similarly, $s$ is {\em right-mover in $\aliequivgeneric$} if for any $\alisequencex r=\alisequencex r[1]\ldots\alisequencex r[k]\in\alirunsx {tso} P$ we have $\alisequencex r[i-1]\alisequencex r[i]=(R,t:s)(R',t':s')$ such that $\alimemtracex {s} \alipotsox {\alimemtracex {\alisequencex r}} \alimemtracex {s'}$ does not hold, then there exists a run $\alisequencex {r'}$ in $[\alisequencex r]_{\aliequivgeneric}$ such that $\alisequencex {r'}\langle 1,i\rangle=\alisequencex r\langle 1,i-2\rangle \alisequencex r[i]\alisequencex r[i-1]$.
\end{definition}

Intuitively, $s$ is left mover (resp. right mover) if reordering $s$ before (resp. after) any other statement that is concurrent with $s$ does not change the behavior (all runs belonging to the same equivalence have the same behavior).
In the classic definition of reduction, the equivalence relation requires that the two sequences be permutations of each other and that the end states are identical, which corresponds to $\tsoequiv$ in the TSO context.
The reason for the added generality will become clear once we generalize reduction with abstraction.

Our first result about movers follows immediately from definitions.
\begin{lemma}
Let $P$ be a program.
All of its TSO runs are SC-like if all remote writes are left-movers in $\tsoequiv$.
Dually, all of its TSO runs are SC-like if all local actions (i.e. ignoring remote writes) are right-movers in $\tsoequiv$.
\end{lemma}
\begin{proof}
Take any run $\alisequencex r\in \alirunsx {tso} P$.
If all remote writes are left-movers in $\tsoequiv$, then there exists a run $\alisequencex {r'}\tsoequiv\alisequencex {r}$ such that all local writes are synchronous. 
Then by Def.~\ref{def:sc-like}, $\alisequencex r$ is SC-like.
A symmetric argument applies for the second case.
\end{proof}

%All of its TSO runs are observationally SC-like if all remote writes are left-movers in $\tsoequiv$.
%Dually, all of its TSO runs are observationally SC-like if all local actions (excluding remote writes) are right-movers in $\tsoequiv$.
%\end{lemma}
This result depends on a strong constraint, all updates being left-movers or all reads being right-movers, which is unlikely to be satisfied by many programs.
In what follows we will provide a series of incrementally more general results.
Fix a labelled program $P=\{m_1,\ldots,m_n\}$.
Let $\sigma$ be a partitioning of (the methods of) $P$.
The {\em $\sigma$-partitioned runs} of $P[T]$ is the subset of $\alirunsx {tso} {P[T]}$ containing all and only those runs in which each $t$ executes methods from the same partition.
\begin{lemma}
Let $P$ be a program and $\sigma$ be a partitioning of $P$.
If $\sigma$ is such that in $\sigma$-partitioned runs of $P[T]$ for each partition, all remote write actions executed by all the methods in the partition are left-movers or all local actions executed by all the methods in the partition are right-movers, then $\sigma$-partitioned runs are SC-like.
\end{lemma}
Call a subset of $\alirunsx {tso} {P[T]}$ {\em singular} if it contains all and only those runs in which each thread $t\in T$ runs at most one method; i.e. the {\sc\small Init}-transition is executed at most once by each $t$.
Then the following is a particular instance of the previous result.
\begin{corollary}
All singular TSO runs of $P[T]$ are SC-like if for each method $m_i$, either all of its local actions are right-movers in $\tsoequiv$ or all of its remote write actions are left-movers in $\tsoequiv$. 
\end{corollary}
We can specialize even further, obtaining the following, which is also implied by triangular-race freedom of \cite{Owe2009}.
\begin{corollary}
Let $P$ be such that each $m_i$ either only updates the shared memory (no read actions) or only reads shared memory (no write actions).
Then all singular TSO runs of $P[T]$ are SC-like.
\end{corollary}

%Now we will investigate the impact of placing fence statements in restricting the non-SC-like behaviors of programs.
\paragraph{Write atomicity.}
We will call a statement $s$ in $P$ {\em atomic} if all transitions whose source is $s$ are executed without interference. 
In particular, a memory update statement ($s=\alimemx i\mathtt{:=} e$) is atomic if the two transitions caused by $s$ are adjacent; i.e. the local write action due to $s$ is synchronous.
The reason we introduce write atomicity should be clear: By Lem.~\ref{lem:synchronous-write}, if one can prove that for any TSO run of program $P$ there exists an equivalent run in which $s$ is atomic, then we can safely transform $s$ to $\aliatomicx {s\mathtt{;} \alifence}$.
This substitution decreases the number of possible interleavings, which in turn simplifies the analysis of the program.
Indeed, the latter is a step towards the ultimate goal of analyzing the whole program under SC semantics.
For the following fix an update statement $s$ and let $s^{loc}$ and $s^{rem}$ denote its local and remote write actions.
\begin{lemma}
If in any TSO run in which $s^{loc}$ is executed by some thread $t$, all statements executed by $t$ up to the occurrence of $s^{loc}$ are atomic and $s^{rem}$ is left-mover in $\tsoequiv$, then $s$ is atomic.
\end{lemma}
Observe that we have to require all the statements executed by $t$ preceding $s^{loc}$ be atomic.
To understand why we need this (sufficiency) condition, consider the following run:
\[
\mathbf{\alpha}\ \cdot\ s^{loc}\ \cdot\ \mathbf{\beta}_1\ \cdot\ \remwritex t x v\ \cdot\ \mathbf{\beta}_2\ \cdot\ s^{rem}
\]
If the remote write action $\remwritex t x v$, whose matching local action must precede $s^{loc}$, cannot move to the left of every action in $\beta_1$, then even though $s^{rem}$ itself is a left-mover, there is not an equivalent run in which $s^{loc}$ and $s^{rem}$ are adjacent because of the presence of $\remwritex t x v$ in between the two.

However, the above argument leads to the following special instance which incidentally gives an insight about how fence statements can lead to SC-like programs.
\begin{corollary}\label{cor:prec-fence}
If in any TSO run in which $s^{loc}$ is executed by thread $t$, there exists a fence action executed by $t$ preceding $s^{loc}$, no other local write actions by $t$ occur between the two actions and $s^{rem}$ is left-mover, then $s$ is atomic.
\end{corollary}
Essentially, the antecedent of the Cor.~\ref{cor:prec-fence} implies the absence of $\remwritex t x v$ of the above sample run.
Thus, unlike the general case, in order to argue that a write immediately following a fence statement is atomic, we only need to prove that its matching remote write action is left-mover. 
There is a dual of this result.
\begin{lemma}
If in any TSO run in which $s^{loc}$ is executed by thread $t$, there does not exist any read action executed by $t$ until the occurrence of $s^{rem}$, then $s$ is atomic.
\end{lemma}
To see why this holds, observe that a local write action is always a right-mover because the changes it makes are invisible to other threads.
Then, to conclude that a write statement is atomic, one has to consider the mover type of all possible local actions that can come between the local and remote write actions.
A well-known way of restricting this sequence of actions is through the use of a fence, which we state next as a special instance of the previous result which is yet another way fence statements can result in  SC-like behaviors.
\begin{corollary}\label{cor:succ-fence}
Let $s;C;\alifence$ be a code block in some $m_i$ such that $C$ does not contain any read of shared memory (no statements of the form $r:=\alimemx e$).
Then $s$ and all the writes in $C$ are atomic.
\end{corollary}

\paragraph{Compositionality.}
Library implementations can be seen as programs which are to be executed in arbitrary contexts.
Unlike SC semantics where atomicity of methods carry over to arbitrary execution contexts as long as there is a separation of memory footprints, atomicity of methods under TSO semantics does not immediately translate into atomicity in other execution contexts even when the footprints are distinct.
We let {\em non-interfering execution context} for $P$ denote any program $P'$ which does not access any location that $P$ does.
We will now state our main compositionality results.

\begin{theorem}[Compositionality-Remote]
Let $m$ be a method in $P$ such that all of its write statements are proved atomic by showing that its remote write actions are left-movers.
Then, $m$ remains atomic in any non-interfering execution context for $P$ which executes a fence statement every time $m$ completes and returns.
\end{theorem}
\begin{theorem}[Compositionality-Local]
Let $m$ be a method in $P$ such that all of its write statements are proved atomic by showing that its local actions are right-movers.
Then, $m$ remains atomic in any non-interfering execution context which executes a fence statement before each call to $m$.
\end{theorem}
These follow from the assumption of non-interference and Cor.~\ref{cor:succ-fence} and Cor.~\ref{cor:prec-fence}.
These results show that there is no silver bullet for placing a fence statement relative to library calls. 
The safe move is to delimit each call by fence statements, but one can do away with one or both (if the method itself has a beginning or ending fence statement) of those statements after checking the atomicity of each write in the method.

\paragraph{Completeness.}
We end this section by stating the completeness of reduction.
Call a TSO run {\em unambiguous} if there do not exist two distinct write actions updating the same location with the same value.
If the definition of $\tsoequiv$ is strengthened to require that the mapping between reads and the writes from which they obtain their value (known usually as the reads-from mapping), then the requirement of unambiguity is not needed.
Restricted to the set of unambiguous runs, we have the following result.
\begin{theorem}
Let $P$ be a program and $s$ a memory update statement in $P$.
Let $\alisequencex r$ be an unambiguous TSO run of $P$, in the form 
\[
\alpha\ \cdot\ s^{loc}\ \cdot\ \beta\ \cdot s^{rem}
\]
where $s^{loc}$ is executed by $t$.
Then $\alisequencex r$ has an $\tsoequiv$-equivalent run in which $s$ is atomic iff there is a permuation of $\beta\aliperm\beta_1\beta_2$ such that 
\[
\alisequencex r\ \tsoequiv\ \alpha\ \cdot\ \beta_1\ \cdot\ s^{loc}\ \cdot\ s^{rem}\ \cdot\ \beta_2 
\]
\end{theorem}
\begin{proof}[Proof (Sketch)]
One direction is by definition.
For the other direction, assume that such a permutation does not exist and let $\alilenx {\beta}=k$.
Without loss of generality, assume that $s^{loc}$ cannot move to the right of $\beta[1]$ and $s^{rem}$ cannot move the left of $\beta[k]$.
Otherwise, move both actions to the right and to the left until such actions which must exist by the assumption are found.
Then, it must be the case that there is an increasing sequence $i_1,\ldots,i_n$ of indices in $[1,k]$ such that $\beta[i_j]$ cannot move to the right of $\beta[i_{j+1}]$ for $1<j<n$ with $i_1=1$ and $i_n=k$.
But if such a sequence exists, $s$ cannot be atomic by the unambiguity assumption.
\end{proof}

%
%{\sc Code snippets demonstrating the results.}
