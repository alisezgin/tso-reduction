%-----------------------------------------------------------------------------
%
%               Template for sigplanconf LaTeX Class
%
% Name:         sigplanconf-template.tex
%
% Purpose:      A template for sigplanconf.cls, which is a LaTeX 2e class
%               file for SIGPLAN conference proceedings.
%
% Guide:        Refer to "Author's Guide to the ACM SIGPLAN Class,"
%               sigplanconf-guide.pdf
%
% Author:       Paul C. Anagnostopoulos
%               Windfall Software
%               978 371-2316
%               paul@windfall.com
%
% Created:      15 February 2005
%
%-----------------------------------------------------------------------------


\documentclass[preprint,9pt]{sigplanconf}

% The following \documentclass options may be useful:

% preprint      Remove this option only once the paper is in final form.
% 10pt          To set in 10-point type instead of 9-point.
% 11pt          To set in 11-point type instead of 9-point.
% authoryear    To obtain author/year citation style instead of numeric.

\usepackage{amsmath,amssymb,amsthm}
\usepackage{mathrsfs}
\usepackage{alltt}
\usepackage{xspace}
\usepackage{mathpartir}
\usepackage{stmaryrd}

\newcommand{\COMMENT}[1]{}

\begin{document}

\special{papersize=8.5in,11in}
\setlength{\pdfpageheight}{\paperheight}
\setlength{\pdfpagewidth}{\paperwidth}

\conferenceinfo{POPL '15}{Month d--d, 20yy, City, ST, Country} 
\copyrightyear{2015} 
\copyrightdata{978-1-nnnn-nnnn-n/yy/mm} 
\doi{nnnnnnn.nnnnnnn}

% Uncomment one of the following two, if you are not going for the 
% traditional copyright transfer agreement.

%\exclusivelicense                % ACM gets exclusive license to publish, 
                                  % you retain copyright

%\permissiontopublish             % ACM gets nonexclusive license to publish
                                  % (paid open-access papers, 
                                  % short abstracts)

%\titlebanner{Reduction for TSO}        % These are ignored unless
\preprintfooter{Reduction for TSO}   % 'preprint' option specified.

\title{Reducing TSO to SC by Reduction}
%\subtitle{TSO Simplified}

\authorinfo{}{}{}
%\authorinfo{Ismail Kuru\and Serdar Tasiran}
%           {Koc University}
%           {ikuru@ku.edu.tr/stasiran@ku.edu.tr}
%\authorinfo{Ali Sezgin}
%           {University of Cambridge}
%           {ali.sezgin@cl.cam.ac.uk}

\maketitle

\begin{abstract}
%A prominent way of analyzing programs written for relaxed memory models is to check whether it is sound, for the particular program under analysis, to assume sequential consistency (SC) which is taken to be the tractability threshold for concurrent reasoning.
%The approach is based on establishing a {\em data race freedom} result, which essentially identifies for a given memory model the class of programs which cannot manifest non-SC behaviors.
Analysis of programs running on total store ordering (TSO) memory model can be restricted to sequential consistency (SC) analysis, thereby porting the original problem to a well-established and understood domain, if the program does not have triangular races.
%The total store ordering (TSO) memory model in particular has been fully characterized: a program will be SC-like, i.e. without any non-SC behaviors, if and only if it is triangular race free.
Unfortunately, checking whether a program, even when constrained to finite data domains, has a triangular race belongs to {\sc PSpace}.
Furthermore, it is unclear what one is to do except for reasoning in full TSO semantics if the program fails to avoid those races.

In this paper, we tackle the problem of TSO program analysis, possibly in the presence of triangular races.
We begin by generalizing Lipton's reduction theory, hitherto limited to SC, for TSO programs.
Based on an interleaving semantics for TSO in which every write is split into a pair of actions, a write into buffer and its flush from the buffer, reduction arguments lead to proofs of write atomicity. 
In order to prove that a TSO program is race-free, it is sufficient to prove all of its writes atomic.
We state results on sufficient conditions for ensuring write atomicity, compositionality, and prove completeness of reduction theory under TSO semantics.

For programs that do contain triangular races, we introduce abstraction: $P'$ abstracts $P$ if every behavior of $P$ is also a behavior of $P'$.
In essence, $P'$ avoids races by non-deterministically guessing values read or written.
This added non-determinism leads to less dependence between concurrent actions, which results in more atomic writes. We illustrate the use of abstraction by transforming a sender/receiver implementation with triangular race to one without. 

We also show how write atomicity proofs can be mechanized.
Our methodology is based on transforming TSO programs to equivalent SC programs by simulating each executiong thread of the former by two tightly coupled execution threads in the latter.
We demonstrate our methodology by proving the atomicity of writes in double checked initialization.
\end{abstract}

%\category{CR-number}{subcategory}{third-level}

% general terms are not compulsory anymore, 
% you may leave them out
%\terms
%program verification

%\keywords
%reduction, total-store ordering, sequential consistency, race freedom, safety checking 

% !TEX root = main-tsoreduction.tex
\newcommand{\dontcare}{\ensuremath{\star}}
\newcommand{\genmemaccess}{\ensuremath{\mathsf{M}}}
\newcommand{\genmemaccesstso}{\ensuremath{\genmemaccess_{TSO}}}
\newcommand{\genread}{\ensuremath{\mathsf{R}}}
\newcommand{\aliwrite}{\ensuremath{\mathsf{W}}}
\newcommand{\aliwritex}[3]{\ensuremath{\aliwrite_{#1}(#2/#3)}}
\newcommand{\aliread}{\ensuremath{\mathsf{R}}}
\newcommand{\alireadx}[3]{\ensuremath{\aliread_{#1}(#2/#3)}}
\newcommand{\locwrite}{\ensuremath{\aliwrite^{\mathsf{l}}}}
\newcommand{\locwritex}[3]{\ensuremath{\locwrite_{#1}(#2/#3)}}
\newcommand{\remwrite}{\ensuremath{\aliwrite^{\mathsf{r}}}}
\newcommand{\remwritex}[3]{\ensuremath{\remwrite_{#1}(#2/#3)}}
\newcommand{\alibarrier}{\ensuremath{\mathsf{B}}}
\newcommand{\alibarrierx}[1]{\ensuremath{\alibarrier_{#1}}}
\newcommand{\alilock}{\ensuremath{\mathsf{L}}}
\newcommand{\alilockx}[1]{\ensuremath{\alilock_{#1}}}
\newcommand{\aliunlock}{\ensuremath{\mathsf{U}}}
\newcommand{\aliunlockx}[1]{\ensuremath{\aliunlock_{#1}}}


\newcommand{\alitrue}{\ensuremath{\mathit{true}}}
\newcommand{\alifalse}{\ensuremath{\mathit{false}}}




%preceding formal-framework
\newcommand{\alilen}{\ensuremath{\mathsf{len}}}
\newcommand{\alilenx}[1]{\ensuremath{\alilen(#1)}}
\newcommand{\alilendef}{\ensuremath{\alilen(\mathbf{e})}}
\newcommand{\tsoalph}{\ensuremath{\mathit{Act}}}
\newcommand{\aliprojx}[2]{\ensuremath{#1\downarrow_{#2}}}
\newcommand{\aliperm}{\ensuremath{\sim_{\pi}}}
\newcommand{\alipotso}{\ensuremath{<^{tso}}}
\newcommand{\alipotsox}[1]{\ensuremath{\alipotso_{#1}}}
\newcommand{\aliposc}{\ensuremath{<^{sc}}}
\newcommand{\aliposcx}[1]{\ensuremath{\aliposc_{#1}}}
\newcommand{\alimatch}{\ensuremath{\mu}}
\newcommand{\alimatchx}[1]{\ensuremath{\alimatch(#1)}}



%preceding reduction-for-tso
\newcommand{\alisplit}{\ensuremath{\mathsf{Split}}}
\newcommand{\alisplitx}[1]{\ensuremath{\alisplit(#1)}}
\newcommand{\alisplitprogx}[1]{\ensuremath{#1^S}}
\newcommand{\aliloctrans}{\ensuremath{\tau_{l}}}
\newcommand{\aliloctransx}[1]{\ensuremath{\aliloctrans(#1)}}
\newcommand{\aliremtrans}{\ensuremath{\tau_{r}}}
\newcommand{\aliremtransx}[1]{\ensuremath{\aliremtrans(#1)}}
\newcommand{\aliequivgeneric}{\ensuremath{\simeq}}

%preceding operational-semantics
\newcommand{\alicontrol}{\ensuremath{\mathsf{Ctrl}}}
\newcommand{\alicontrolx}[1]{\ensuremath{\alicontrol(#1)}}
\newcommand{\alivaluation}{\ensuremath{\mathsf{Val}}}
\newcommand{\alivaluationx}[1]{\ensuremath{\alivaluation(#1)}}
\newcommand{\alisucc}{\ensuremath{\mathsf{Succ}}}
\newcommand{\alisuccx}[1]{\ensuremath{\alisucc(#1)}}
\newcommand{\aliprec}{\ensuremath{\mathsf{Prec}}}
\newcommand{\aliprecx}[1]{\ensuremath{\aliprec(#1)}}
\newcommand{\alifirst}{\ensuremath{\mathsf{Fst}}}
\newcommand{\alifirstx}[1]{\ensuremath{\alifirst(#1)}}
\newcommand{\alisequencex}[1]{\ensuremath{\vec{#1}}}
\newcommand{\alieval}{\ensuremath{\llbracket \rrbracket}}
\newcommand{\alievalx}[1]{\ensuremath{\llbracket #1 \rrbracket}}
\newcommand{\alibuffer}{\ensuremath{\mathsf{Buf}}}
\newcommand{\alibufferx}[1]{\ensuremath{\alibuffer(#1)}}
\newcommand{\alireadbuffer}{\ensuremath{\mathsf{RdBuf}}}
\newcommand{\alireadbufferx}[2]{\ensuremath{\alireadbuffer(#1,#2)}}
\newcommand{\aliexecmode}{\ensuremath{\mathsf{Mode}}}
\newcommand{\aliatomiclock}{\ensuremath{\mathsf{lck}}}
\newcommand{\aliatomicexitlabel}{\ensuremath{l_{x}}}
\newcommand{\alienabled}{\ensuremath{\mathsf{Enb}}}
\newcommand{\alienabledx}[1]{\ensuremath{\alienabled(#1)}}
\newcommand{\alilts}{\ensuremath{\mathnormal{LTS}}}
\newcommand{\aliltsx}[1]{\ensuremath{\alilts(#1)}}
\newcommand{\alitrace}{\ensuremath{\mathsf{Tr}}}
\newcommand{\alitracex}[1]{\ensuremath{\alitrace(#1)}}
\newcommand{\alimemtrace}{\ensuremath{\mathsf{Mem}}}
\newcommand{\alimemtracex}[1]{\ensuremath{\alimemtrace(#1)}}
\newcommand{\aliosrules}{\ensuremath{Rules}}

\newcommand{\aliltsstates}{\ensuremath{Q}}
\newcommand{\aliltsstatesx}[1]{\ensuremath{\aliltsstates^{#1}}}
\newcommand{\aliltstransitions}{\ensuremath{Tr}}
\newcommand{\aliltstransitionsx}[1]{\ensuremath{\aliltstransitions^{#1}}}

\newcommand{\aliruns}{\ensuremath{\mathbf{R}}}
\newcommand{\alirunsx}[2]{\ensuremath{\aliruns_{#1}(#2)}}

%preceding abstraction-tso (Abstracting TSO programs)
\newcommand{\aliabsrulex}[1]{\textnormal{\sc\small #1}}
\newcommand{\alihavocval}{\ensuremath{\star}}
\newcommand{\alicompequiv}{\ensuremath{\tsoequiv_c}}



\section{Introduction}
\label{sec:intro}
A prominent way of analyzing programs written for relaxed memory models is to check whether it is sound, for the particular program under analysis, to assume sequential consistency (SC) which enjoys a plethora of analysis techniques and tools.
The approach is based on establishing a {\em data race freedom} result, which essentially identifies for a given memory model the class of programs which cannot manifest non-SC behaviors.

Total store ordering (TSO) is a well-known relaxed memory model variants of which are employed in commercial processors, most notably x86 family of processors~\cite{SSO+2010}.
Both TSO and SC give the illusion that the thread-local program order is respected for memory accesses.
Unlike SC where updates are assumed to take effect instantaneously over all threads, in TSO updates are observably split into two: a locally visible update and an instantaneous remote update not necessarily simultaneous with the local update.
This split is due to a thread local store buffer which writes have to go through before becoming visible by other threads.
Operational models of TSO formalize this explicitly: local updates are inserted into the thread local queue; entries are removed from these queues asynchronously and non-deterministically with each removal updating a single global shared memory location. 

It was shown in \cite{Owe2010} that a TSO program does not have non-SC behaviors iff it does not have a triangular race, called triangular race freedom (TRF).
A triangular race is one in which one thread updates a location, then reads another location which is concurrently updated by another thread.
A typical example is one where thread $t$ writes 1 to $A$ followed by a read of $B$ and thread $u$ writes 1 to $B$ followed by a read of $A$.
It is possible for both reads to return 0, the initial values for $A$ and $B$, because the remote writes are delayed past the reads.
Such an execution is non-SC because at least one of the reads should observe the other thread's write.

Since TRF is both necessary and sufficient for excluding non-SC behaviors, a methodology for TRF checking is crucial in simplifying TSO program analysis.
The only previous work that considered checking TRF statically is the work of Bouajjani et al.~\cite{BDM2013}.
Their method based on SC reachability is complete, but it suffers from high complexity ({\sc\small Pspace}) and relies on the finiteness of data domains and on an a priori fixed number of threads.

We mentioned that one strives to convert TSO programs to that of SC programs because the latter is much better studied.
One well-known approach is Lipton's reduction theory~\cite{Lip1975}.
It limits program analysis to a set of representative runs which capture the behavior of the whole program, possibly relative to a desired property.
It is based on removing superficial concurrency by determining those sequentially composed actions which do not add new behavior when they interfere with their execution environment, the process known as determining {\em mover types} of statements.
Even though the only essential requirement for the application of reduction is interleaving semantics, it has always been limited to SC program analysis.
Its application to relaxed memory models was questionable because it was not clear how to argue the mover types of statements when their effects were non-atomic.

In this paper, we apply reduction theory to TSO programs.
Our first contribution is a deeper understanding of TSO programs.
Following earlier work mentioned above, we show how reduction theory based arguments can be used to check TRF.
We view the run of a TSO program as a sequence in which each memory write is translated into two actions: a local write and a remote write.
Instead of having an all or nothing approach, we use reduction to check whether separating the local and remote writes arbitrarily introduces any new behavior as opposed to executing them consecutively.
This notion of write atomicity is the crux of our analysis:
A program is TRF if all of its write statements are atomic.
Reduction arguments yield surprisingly insightful results.
For instance, we show that a write statement is atomic because its remote write is left-mover, then putting a fence immediately {\em before} the statement guarantees the atomicity of that write in all non-interfering execution contexts.
We also prove the completeness of reduction in proving write atomicity, hence TRF.

Our second contribution is the simplification of the analysis of TSO programs which do contain triangular races, a domain to which so far only full fledged TSO analysis has been applied. 
Our main observation is that replacing a read of a memory location into a register with a non-deterministic assignment to that register has the potential of removing triangular races.
Such replacements are instances of a more general approach known as abstraction, previously used in~\cite{EQT2009} for SC program analysis.
We formalize the notion of abstraction for TSO programs and describe how abstraction along with reduction can be used to turn non-TRF programs into TRF programs.
Following the description of the methodology, we demonstrate its use on a program with triangular race.

Our third and final contribution is the implementation of these theoretical results.
Orthogonal to what we have done so far, we introduce a program transformation algorithm which converts a TSO program (with or without triangular races) into an equivalent SC program.
The construction converts each execution thread of the TSO program into a pair of tightly coupled execution threads.
Intuitively, whatever the thread in the original TSO program does is simulated by this pair. 
All local actions are simulated by one, all remote write actions are simulated by the other.
Once the transformed program is obtained, we can implement all our previous results related to write atomicity and abstraction, which we do by mechanically verifying write atomicity of double checked initialization.
We also briefly discuss the applicability of reduction to other relaxed memory models, which as we have already hinted, is not fundamentally impossible as long as the relaxed memory model admits an interleaving operational semantics.
 
\subsection{Overview}
\label{subsec:overview}
%In this section we are going to walk through several examples illustrating the main concepts of our approach.
%The discussion will be necessarily kept at a semi-formal level; all relevant formal definitions are given in the following sections.
We start by explaining how one reasons about TSO programs using reduction.

\paragraph{Reduction for TSO.}
Consider a code snippet from a program with three threads:
\begin{eqnarray*} 
 C_{t1} &  C_{u1} &  C_{v1}\\
X \mathtt{:= 1;} &  Y\mathtt{:= 2;} &  p\mathtt{:=}X\mathtt{;}\\
& r\mathtt{:= }X\mathtt{;} &  q\mathtt{:=}Y\mathtt{;}\\
C_{t2} & C_{u2} & C_{v2}
\end{eqnarray*}
where $C_*$ represent code segments which refer to neither $X$ nor $Y$.
In the specified segment, the thread on the left, $t$, writes 1 into shared variable $X$.
The thread in the middle, $u$, writes 2 to shared variable $Y$ and then reads the value of $X$ into local variable $r$.
Finally, the thread on the right, $v$, reads the values of $X$ and $Y$ into local variables $p$ and $q$.

It is possible to observe $r=q=0$ and $p=1$ under TSO semantics.
For instance, the execution segment
\[
\locwritex u Y 2\cdot \alpha\cdot \locwritex t X 1\cdot \beta\cdot \remwritex t X 1\cdot \gamma\cdot \remwritex u Y 2 
\]
where $\locwritex t X 1$ represents the insertion of the write of $X$ by $t$ into its store buffer and $\remwritex t X 1$ represents the flushing of the associated entry from the buffer and $\alpha,\beta,\gamma$ are sequences of actions.
The read values are possible if $u$ executes its read either in $\alpha$ or $\beta$ and $v$ executes its reads in $\gamma$.

In terms of reduction theory, we can equivalently claim the non-SC of the given execution by arguing that it is impossible to {\em move} the local write by $u$ next to its remote write without changing the values read by threads $u$ and $v$.
In the sample execution given above, since the read of $X$ by $u$ is not in $\gamma$ and it cannot be reordered with $\remwritex t X 1$ without changing its observed value, the local write $\locwritex u Y 2$ cannot move to the right of $\remwritex t X 1$ either.
On the other hand, the remote write $\remwritex u Y 2$ cannot move to the left of every action in $\gamma$ since it  contains the read of $Y$ executed by $v$. 
A TSO execution has an equivalent SC execution only when all local and remote write actions can be put together without changing the read values, which means that the above sequence is indeed non-SC.

\paragraph{Write atomicity.}
The sample program has non-SC behavior, but even before attempting to turn it into a TRF program, we observe that the local and remote write actions due to the write by $t$ can always be put together if for the moment we ignore the presence of $C_{t1}$ and $C_{t2}$; i.e. assume that $t$ executes only the write to $X$.
This is because a local write action, which only updates the thread local state (changing buffer contents), moves to the right of every concurrent action without changing the overall behavior of the execution.
As we show, one can place a fence statement immediately after an atomic write statement without restricting the overall behavior of the program.

\paragraph{Removing triangular race.}
For the sake of simplicity, assume that the value of $q$ is not subsequently used by $v$; i.e. $q$ is not read in $C_{v2}$.
Then, we can obtain a new program by replacing the read of $Y$ with $q\mathtt{:=}\alihavocval$, which intuitively can assigns any value to $q$.
The behaviors of this new program is a proper superset of the original program because of additional non-determinism. 
The fact that $Y$ is not read by $v$ in the new program immediately allows to prove that the previously non-atomic write to $Y$ now is atomic. 
However, unlike the atomicity argument for the write of $X$ by $t$, the write to $Y$ is atomic because we can move $\remwritex u Y 2$ to the left of $\remwritex t X 1$ because $\gamma$ does not contain the conflicting read of $Y$. 
Here again we assume that $C_{v1}$ and $C_{v2}$ do not exist which brings us to our next observation.

\paragraph{Compositionality.}
At this point we know that in the abstracted program, the write by $t$ is atomic because its local action is a right-mover (abstraction does not affect its atomicity argument), and the write by $u$ is atomic because its remote action is a left-mover.
We will show that the write to $X$ remains atomic if the first action in $C_{t2}$ is a fence statement.
Dually, the write to $Y$ remains atomic if the last action in $C_{v1}$ is a fence statement.
Analyzing write atomicity based on reduction allows us to make these distinctions in fence placements.

\paragraph{Program transformation and mechanical verification.}
Transforming TSO programs into equivalent SC programs usually entails embedding an array per thread, representing the local store buffer.
Then, each write is inserted into this array and a non-deterministic loop, representing flushing of the contents of the store buffer, is placed between each statement of the original program.
This encoding for us is not suitable because we want to explicitly check the mover types of local and remote writes. 
To that end, we make use of the fact that there is a bijection between buffer insertions and removals that respect thread local program order.

In our sample code, the write to $X$ by $t$ will be transformed into two writes, one by $t$ to a new local copy of $X$ and one by $t'$ (the {\em dual} thread of $t$) to $X$.
An important requirement is that the local write by $t$ always precede the remote write by $t'$, and this we achieve by using a semaphore like structure per write statement, which is incremented by $t$ and decremented by $t'$. 

Once an equivalent SC program is obtained, any SC analysis tool can be used to reason about the program.
Here we are primarily interested in the mover types of local and remote write actions and propose one particular way of doing it.

\paragraph{Contributions.} 
To summarize, in this paper we:
\begin{itemize}
\item develop a formal reasoning framework for TSO programs based on reduction,
\item obtain numerous theoretical results adding to the understanding of what separates TSO programs with triangular races from those without,
\item introduce abstraction for TSO programs in order to obtain TRF programs from those that are not,
\item present a novel transformation from TSO programs to equivalent SC programs,
\item propose a way to mechanically verify TRF.
\end{itemize}

\paragraph{Related work.}
Program verification under relaxed memory models has been a field of intense study. 
In this discussion of related work, we focus only on TSO and discuss related work on model checking, fence insertion, reachability analysis for computational models, testing and monitoring only when the correctness criteria or underlying ideas are relevant to our work. 
Our technique is distinguished from this work fundamentally in that our proofs are valid for an arbitrary number of threads, memory addresses, or arbitrary (not necessarily finite domain) program variables and do not bound the length of the store buffer. 
Furthermore, our proofs cover the entire set of executions of a program and are not limited to a subset of interleavings defined, e.g., by bounding the number of context switches. 

We take as our formal basis the abstract machine memory model described in the operational semantics of Owens et al.~\cite{OSS2009}. 
Our work provides a static, mechanical checking method for and generalizes triangular race freedom (TRF) introduced by Owens~\cite{Owe2010}. 
TRF and the notion of TSO-robustness by Bouajjani et al.~\cite{BDM2013} are very closely related. 
Since our mechanical verification is SMT-based, it is not restricted to finite state programs or a finite number of threads but involves potentially undecidable SMT queries. 

The use of code-to-code translation as a means for converting a verification task on a program running on a relaxed memory model to another verification task expressed on a sequentially consistent program is common in the literature (e.g., \cite{FBP2011,BDM2013,DMV+2013,AKN+2013}. 
We do not consider code-to-code translation to be a verification technique in and of itself, rather, a mechanism for realizing a certain reduction from one problem to another. 
In terms of the crux of the reduction, techniques differ widely. 
Our code-to-code translation is also similar -- its value is in enabling us to use abstraction and reduction to prove triangular race freedom statically. 
Our translation differs from others in that it is not limited to a given, fixed number of threads. 

\cite{Rid2010} introduces a proof system for reasoning about x86 assembly programs running against the weak x86-TSO memory model. 
This Rely-Guarantee proof system enables the system such that one processor can refer to other's local state. 
Mechanical proofs are handled by Hol in the backend. Our work provides more automation by providing pre-defined proof rules that facilitate the verification of TRF on the original or abstracted program. 

\cite{JLP+2014} presents a proof methodology to verify the correctness of compiler translations from a Java-like intermediate representation to a low-level structured register transfer language (RTL) representation under the TSO model. 
The refinement method used makes use of some notions (e.g., atomicity, reduction) common with our work. 
Their technique assumes that non-interference between code segments has been verified separately and verifies code transformations under this assumption. 
Our work centers on verifying the desired non-interference under TSO. 

\paragraph{Roadmap.}
In Sec.~\ref{sec:formal-framework}, we set the formal framework.
In Sec.~\ref{sec:reduction-for-tso}, we introduce reduction theory for TSO programs.
In Sec.~\ref{sec:abstracting-tso-programs}, we formalize the notion of abstraction for TSO programs and illustrate the methodology on a handshake based send/receive implementation.
In Sec.~\ref{sec:mechanical-verification}, we present our transformation technique converting TSO programs into equivalent SC programs.
Section~\ref{sec:conclusion} concludes the paper.




\newtheorem{definition}{Definition}
\newtheorem{proposition}{Proposition}
\newtheorem{lemma}{Lemma}
\newtheorem{corollary}{Corollary}
\newtheorem{theorem}{Theorem}
%\newtheorem{proof}{Proof}

%\newenvironment{proof}{\bf{Proof.\,\,}\rm}{$\hspace*{\fill}\Box$\par}

\section{Formal Framework}
\label{sec:formal-framework}

\begin{figure}
\begin{tabular}{lcll}
action ($a$,$b$) & ::= & \locwritex t x v & (local write: insertion of the write\\
& & & of value $v$ into location $x$ by thread $t$)\\
& $|$ & \remwritex t x v & (remote write: shared memory update\\
& & & of $x$ with $v$ by $t$)\\
& $|$ & \alireadx t x v & (a read of value $v$ from $x$ by $t$)\\
& $|$ & \alibarrierx t & (memory barrier by $t$)
%& $|$ & \alilockx t & (the beginning of a \\
%& $|$ & \aliunlockx t
\end{tabular}
\caption{The alphabet $\tsoalph$ of memory actions in TSO.}
\label{fig:grammar-tso}
\end{figure}
\paragraph{Notation.} 
The symbols we are going to use and their meanings are given in Fig.~\ref{fig:grammar-tso}.
For simplicity, we omit locked operations which can be modeled by the given actions.
A local write $\locwritex t x v$ {\em matches} the remote write $\remwritex u y w$ iff $t=u$, $x=y$ and $v=w$.
We will use the notation $\tsoalph_{op,thr,loc}$ to denote the subset of $\tsoalph$ which contains all actions of type $op\in\{\locwrite, \remwrite, \aliread, \alibarrier\}$, by thread $thr$, and into location $loc$.
Omitting parameters denotes existential quantification; e.g. $\tsoalph_{-,t,-}$ is the set of all actions done by thread $t$.


Let $\mathbf{e}$ be a sequence over $\tsoalph$.
We will use indexed notation to refer to the elements in $\mathbf{e}$: $\mathbf{e}[i]$ is the $i^{th}$ action in $\mathbf{e}$.
Similarly, we let $\mathbf{e}\langle i,j\rangle$ denote the segment of $\mathbf{e}$ from $\mathbf{e}[i]$ to $\mathbf{e}[j]$ with both ends inclusive.
The length of $\mathbf{e}$ is written as $|\mathbf{e}|$ and gives the number of actions in $\mathbf{e}$.
Let $\pi$ be a permutation over $[1,k]$ and $\mathbf{e}$ be of length $k$.
We use $\pi(\mathbf{e})$ to denote the sequence $\mathbf{e}[\pi(1)]\ldots \mathbf{e}[\pi(k)]$.
Two executions $\mathbf{e}$ and $\mathbf{f}$ are {\em permutationally equivalent}, written $\mathbf{e} \aliperm \mathbf{f}$, if there is a permutation $\pi$ such that $\mathbf{e}=\pi(\mathbf{f})$.

The projection of $\mathbf{e}$ into $\tsoalph_{o,t,l}$, written as $\mathbf{e}\downarrow_{o,t,l}$, is the subsequence obtained by keeping only those actions in $\tsoalph_{o,t,l}$.
For instance, $\aliprojx {\mathbf{e}} {\locwrite,t,-}$ is the subsequence of $\mathbf{e}$ consisting of all the local writes done by $t$.


\paragraph{TSO-executions.}
Let $\mathbf{e}$ be a sequence over $\tsoalph$.
It is {\em matched} if for any $t$ there is an injection $\mu$ from $\aliprojx {\mathbf{e}} {\remwrite,t,-}$ to $\aliprojx {\mathbf{e}} {\locwrite,t,-}$ such that $\mu(a)=b$ implies that $b$ matches $a$.
A local write $\mathbf{e}[i]$ is {\em buffered at $j$} if it is matched by a remote write at position $l>j$.
We will call a matched $\mathbf{e}$ {\em complete} if there are no buffered writes at $|\mathbf{e}|$.

A TSO-execution is a matched sequence $\mathbf{e}$ over $\tsoalph$ subject to well-formedness constraints:
\begin{itemize}
\item The order of remote writes per thread respects the order of their matching local writes.
Formally, for $i<j$ such that $\mathbf{e}[i]=\locwritex t x v$ and $\mathbf{e}[j]=\locwritex t y w$ and there is $k$ with $\mu(\mathbf{e}[k])=\mathbf{e}[j]$, then there is $l$ such that $j<l<k$ and $\mu(\mathbf{e}[l])=\mathbf{e}[i]$.
\item The read values are consistent.
Formally, if $\mathbf{e}[j]=\alireadx t x v$, then either i) $\locwritex t x v$ is the most recent buffered write at $j$ by $t$ to $x$, or ii) no buffered write to $x$ by $t$ at $j$ exists and $\remwritex u x v$ is the most recent remote write, or iii) neither condition applies and $v$ is the initial value.
\item The barrier operations can only happen when the buffer is empty.
Formally, if $\mathbf{e}[j]=\alibarrierx t$ and if $\locwritex u x v$ is buffered at $j$, then $t\neq u$.
\end{itemize}



\begin{definition}[Atomic]
A local write $\mathbf{e}[i]$ is {\em atomic} if it is not buffered at $i+1$.
A complete TSO-execution $\mathbf{e}$ is called {\em atomic} if all of its local writes are atomic.
\end{definition}

Each TSO-execution $\mathbf{e}$ induces a partial order $\alipotsox {\mathbf{e}}$ over $\{\mathbf{e}[i] \mid i\in[1,|\mathbf{e}|]\}$ such that $a\alipotsox {\mathbf{e}} b$ if $a$ occurs before $b$ in $\mathbf{e}$, $a,b\in\tsoalph_{-,t,-}$ and one of the following holds:
\begin{itemize}
\item $a\in\tsoalph_{\remwrite,-,-}$ iff $b\in\tsoalph_{\remwrite,-,-}$.
\item $b=\alimatchx a$.
\end{itemize}

\paragraph{SC-executions.}
An SC-execution $\mathbf{s}$ is a sequence over the alphabet $\{\aliwritex t x v\}\cup\{\alireadx t x v\}$.
The actions of the form $\aliwritex t x v$, denoting the write of value $v$ into location $x$ by thread $t$, are called {\em write} actions.
Each SC-execution has to satisfy: if $\mathbf{s}[i]=\alireadx t x v$, then either i) there is $j<i$ with $\mathbf{s}[j]=\aliwritex u x v$ and there are no write actions to $x$ in $\mathbf{s}\langle j+1,i\rangle$, or ii) there is no write action in $\mathbf{s}\langle 1,i\rangle$ and $v$ is $\bot$.

Similar to TSO-executions, each SC-execution $\mathbf{s}$ induces a partial order $\aliposcx {\mathbf{s}}$ such that $a\aliposcx {\mathbf{s}} b$ if $a$ occurs before $b$ in $\mathbf{s}$ and both belong to the same thread.

\newcommand{\SCofTSO}{\ensuremath{\mathsf{SC}}}
\newcommand{\SCofTSOx}[1]{\ensuremath{\SCofTSO(#1)}}
\newcommand{\SCofTSOstrict}{\ensuremath{\SCofTSO_S}}
\newcommand{\SCofTSOstrictx}[1]{\ensuremath{\SCofTSOstrict(#1)}}
\newcommand{\aliconvert}{\ensuremath{\ulcorner \urcorner}}
\newcommand{\aliconvertx}[1]{\ensuremath{\ulcorner #1 \urcorner}}
\newcommand{\tsoequiv}{\ensuremath{\approx}}
\newcommand{\tsoequivstrict}{\ensuremath{\tsoequiv_S}}
\newcommand{\equivclassx}[2]{\ensuremath{[#1]_{#2}}}
\newcommand{\tighter}{\ensuremath{\sqsubseteq}}
\newcommand{\tighterstrict}{\ensuremath{\tighter_S}}
\newcommand{\tightclosure}{\ensuremath{\mathsf{T}}}
\newcommand{\tightclosurex}[1]{\ensuremath{\tightclosure(#1)}}
\newcommand{\tightclosurestrict}{\ensuremath{\tightclosure_S}}
\newcommand{\tightclosurestrictx}[1]{\ensuremath{\tightclosurestrict(#1)}}

\paragraph{Correspondence between TSO and SC.}
We define a mapping $\aliconvert$ from the actions of TSO to those of SC as follows.
\[
\aliconvertx a \stackrel{def}{=} 
 \begin{cases}
  \aliwritex t x v & \text{, if } a=\locwritex t x v\\
  \alireadx t x v & \text{, if } a=\alireadx t x v\\
  \varepsilon & \text{, if } a=\remwritex t x v\\
  \varepsilon & \text{, if } a=\alibarrierx t
 \end{cases}
\]
For a TSO-execution $\mathbf{e}$, let $\aliconvertx {\mathbf{e}}$ denote the sequence obtained by the point-wise extension of $\aliconvert$.

Let $\mathbf{e}$ be a complete TSO-execution.
We define $\SCofTSOx {\mathbf{e}}$ to be the set of all SC-executions $\mathbf{s}$ such that $\aliconvertx {\aliprojx {\mathbf{e}} {-,t,-}}=\aliprojx {\mathbf{s}} {-,t,-}$.
Informally, $\mathbf{s}$ belongs to $\SCofTSOx {\mathbf{e}}$ if it is an SC-execution and respects the thread local ordering of read and local write actions.

A particular subset of $\SCofTSOx {\mathbf{e}}$ is of interest to us.
Let $\SCofTSOstrictx {\mathbf{e}}\subseteq \SCofTSOx {\mathbf{e}}$ be the set of all $\mathbf{s}$ such that $\aliprojx {\mathbf{e}} {\remwrite,-,-} [i] = \remwritex t x v$ iff $\aliprojx {\mathbf{s}} {\aliwrite,-,-}[i] = \aliwritex t x v$.
Informally, $\mathbf{s}$ will be in $\SCofTSOstrictx {\mathbf{e}}$ if additionally the order among the write actions in $\mathbf{s}$ preserves the order among the remote write actions in $\mathbf{e}$.
We have the following result following immediately from definitions.

\begin{proposition}\label{prop:complete-tso}
For any atomic TSO-execution $\mathbf{e}$, $\aliconvertx {\mathbf{e}} \in \SCofTSOstrictx {\mathbf{e}}$.
\end{proposition}
Proposition~\ref{prop:complete-tso} implies that an atomic TSO-execution can always be transormed into an SC-execution by simply replacing each (adjacent) pair of local and remote write actions by their image under $\aliconvert$.

\paragraph{Equivalence ($\tsoequiv$) and partial order ($\tighter$) over TSO-executions.}
Two complete TSO-executions $\mathbf{e}$ and $\mathbf{f}$ are {\em equivalent}, written $\mathbf{e}\tsoequiv\mathbf{f}$, if one can be obtained from the other by moving the actions such that the per thread order is preserved.
Formally, $\mathbf{e}\tsoequiv\mathbf{f}$ if $\mathbf{e}\aliperm\mathbf{f}$ and $\aliprojx {\mathbf{e}} {-,t,-}=\aliprojx {\mathbf{f}} {-,t,-}$.
They are {\em strictly} equivalent, written $\mathbf{e}\tsoequivstrict\mathbf{f}$, if the order among the remote writes is also preserved.
Formally, $\mathbf{e}\tsoequivstrict\mathbf{f}$ if $\mathbf{e}\tsoequiv\mathbf{f}$ and $\aliprojx {\mathbf{e}} {\remwrite,-,-}=\aliprojx {\mathbf{f}} {\remwrite,-,-}$.
Whenever no confusion is likely to arise, we will use $\pi$ to denote one of the permutations between two (strictly) equivalent TSO-executions establishing their (strict) equivalence.

Atomicity induces a partial order $\tighter$ on TSO-equivalent traces.
A TSO-execution $\mathbf{e}$ is {\em tighter} than $\mathbf{f}$, written $\mathbf{e}\tighter\mathbf{f}$ if $\mathbf{e}\tsoequiv\mathbf{f}$ and whenever $\mathbf{f}[i]$ is an atomic local write, then so is $\mathbf{e}[\pi^{-1}(i)]$.
In other words, $\mathbf{e}$ is tighter than an equivalent $\mathbf{f}$ if all the atomic writes of the latter are also atomic in the former. 
The execution $\mathbf{e}$ is strictly tighter than $\mathbf{f}$, written $\mathbf{e}\tighterstrict\mathbf{f}$, if $\mathbf{e}$ is tighter than $\mathbf{f}$ and they are strictly equivalent.

Each TSO-execution thus induces a set of (strictly) tighter executions.
Let $\tightclosurex{\mathbf{e}}=\{\mathbf{f} \mid \mathbf{f}\tighter\mathbf{e}\}$; that is, the set of all TSO-executions tighter than $\mathbf{e}$.
Let $\tightclosurestrictx {\mathbf{e}}$ denote the subset of $\tightclosurex{\mathbf{e}}$ consisting of only strictly equivalent executions.

\begin{definition}[SC-like]
Let $\mathbf{e}$ be a TSO-execution.
It is called {\em observationally SC-like} if $\tightclosurex {\mathbf{e}}$ contains an atomic execution.
It is called {\em SC-like} if $\tightclosurestrictx {\mathbf{e}}$ contains an atomic execution.
\end{definition}
Since for any TSO-execution $\mathbf{e}$ we have $\tightclosurestrictx {\mathbf{e}}\subseteq \tightclosurex {\mathbf{e}}$, SC-like is a stronger property than observationally SC-like.
The terms are not arbitrarily named as the following proposition shows.

\begin{proposition}
Let $\mathbf{e}$ be a complete TSO-execution.
It is observationally SC-like iff $\tightclosurex {\mathbf{e}} \cap \SCofTSOx {\mathbf{e}} \neq \emptyset$.
It is SC-like iff $\tightclosurestrictx {\mathbf{e}} \cap \SCofTSOstrictx {\mathbf{e}} \neq \emptyset$.
\end{proposition}


\newcommand{\independent}{\ensuremath{\nleftarrow}}
\newcommand{\independentstrict}{\ensuremath{\independent_S}}
\newcommand{\rulelocal}{(\ensuremath{\mathbf{Loc}})}
%\newcommand{\ruleread}{(\ensuremath{\mathbf{Read}})}
\newcommand{\rulereml}{(\ensuremath{\mathbf{RemL}})}
\newcommand{\ruleremr}{(\ensuremath{\mathbf{RemR}})}

\newcommand{\rulethr}{(\ensuremath{\mathbf{Thr}})}
\newcommand{\rulemat}{(\ensuremath{\mathbf{Mat}})}
\newcommand{\ruleloc}{(\ensuremath{\mathbf{Loc}})}
\newcommand{\rulerem}{(\ensuremath{\mathbf{Rem}})}
\newcommand{\ruleremstrict}{(\ensuremath{\mathbf{RemS}})}
\newcommand{\aliconflict}{\ensuremath{\mathsf{Conf}}}
\newcommand{\aliconflictstrict}{\ensuremath{\aliconflict_S}}

\newcommand{\shuffleset}{\ensuremath{\mathsf{Shuff}}}
\newcommand{\shufflesetx}[1]{\ensuremath{\shuffleset(#1)}}

\paragraph{Conflict Relations.}
Two TSO actions $a$ and $b$ {\em conflict}, written $\aliconflict(a,b)$, if one of the following holds:
\begin{description}
\item[\rulerem] $a,b\in\tsoalph_{\remwrite,t,-}$.
\item[\rulethr] $a,b\in\tsoalph_{\{\locwrite,\aliread,\alibarrier\},t,-}$.
\item[\ruleloc] $a,b\in\tsoalph_{-,-,x}$ and $\{a,b\}\nsubseteq\tsoalph_{\aliread,-,-}$.
\end{description}
They {\em strictly} conflict, $\aliconflictstrict(a,b)$, if {\rulerem} is replaced with
\begin{description}
\item[\ruleremstrict] $a,b\in\tsoalph_{\remwrite,-,-}$.
\end{description}

Let $\mathbf{e}$ be a TSO-execution of length $k$.
A sequence $\mathbf{f}$ is a {\em shuffling} of $\mathbf{e}$ if $\mathbf{e}\aliperm\mathbf{f}$ and there is some $i$ such that for all $j\notin\{i,i+1\}$, $\mathbf{e}[j]=\mathbf{f}[j]$ and $\mathbf{e}[i]=\mathbf{f}[i+1]$.
The position $i$ is called the {\em pivot} (of shuffling).
A shuffling of $\mathbf{e}$ with pivot $i$ is {\em valid} if $\pi(\mathbf{e})$ is a TSO-execution and $\mathbf{e}[i]$ does not conflict with $\mathbf{e}[i+1]$.
Let $\shufflesetx {\mathbf{e}}$ denote the closure of the valid shufflings on $\mathbf{e}$.
Formally, $\shufflesetx {\mathbf{e}}$ is defined as the set satisfying
\begin{itemize}
\item $\mathbf{e}\in\shufflesetx {\mathbf{e}}$,
\item $\mathbf{f}\in\shufflesetx {\mathbf{e}}$ implies there is $\mathbf{g}\in\shufflesetx {\mathbf{e}}$ and $\mathbf{f}$ is a valid shuffling of $\mathbf{g}$.
\end{itemize}

Observe that there are two further constraints implicitly enforced by requiring a valid shuffling with pivot $i$. 
First, if $\mathbf{e}[i+1]$ is a barrier action by thread $t$ and $\mathbf{e}[i]$ is a remote write action by the same thread, then they cannot be reordered as barrier cannot occur when there is a buffered write. 
Second, if $\mathbf{e}[i]$ is a local write action by $t$ and $\mathbf{e}[i+1]$ is the matching remote write action, reordering them is not allowed since the resulting sequence will not be a TSO-execution.

The shuffling closure of $\mathbf{e}$ gives an operational under-approximation of TSO-execution equivalence $\tsoequiv$ as the following result indicates.

\begin{proposition}
For complete TSO-execution $\mathbf{e}$, $\shufflesetx {\mathbf{e}} \subseteq \equivclassx {\mathbf{e}} {\tsoequiv}$.
\end{proposition}


\newcommand{\aliprog}{\ensuremath{\mathtt{prog}}}
\newcommand{\alimem}{\ensuremath{\mathtt{mem}}}
\newcommand{\alimemx}[1]{\ensuremath{\alimem[#1]}}
\newcommand{\alifence}{\ensuremath{\mathtt{fence}}}
\newcommand{\aliassume}{\ensuremath{\mathtt{assume}}\xspace}
\newcommand{\aliassumex}[1]{\ensuremath{\mathtt{assume}\ #1}}
\newcommand{\aliassert}{\ensuremath{\mathtt{assert}}\xspace}
\newcommand{\aliassertx}[1]{\ensuremath{\mathtt{assert}\ #1}}
\newcommand{\alireg}{\ensuremath{\mathtt{r}}}
\newcommand{\aliregx}[1]{\ensuremath{\alireg[#1]}}
\newcommand{\aliif}{\ensuremath{\mathtt{if}}}
\newcommand{\alithen}{\ensuremath{\mathtt{then}}}
\newcommand{\alielse}{\ensuremath{\mathtt{else}}}
\newcommand{\aliifelsex}[3]{\ensuremath{\aliif\ #1\ \alithen\ \{#2\}\ \alielse\ \{#3\}}}
\newcommand{\aliwhile}{\ensuremath{\mathtt{while}}}
\newcommand{\aliwhilex}[2]{\ensuremath{\aliwhile(#1)\ \{#2\}}}
\newcommand{\aliatomic}{\ensuremath{\mathtt{atomic}}}
\newcommand{\aliatomicx}[1]{\ensuremath{\aliatomic\{#1\}}}
\newcommand{\aliskip}{\ensuremath{\mathtt{skip}}}
\newcommand{\alitid}{\ensuremath{\mathtt{tid}}}




\begin{figure}
\begin{tabular}{lcl}
$P$ & ::= & $M,P \mid \varepsilon$\\
$M$ ($m$,$m',\ldots$) & ::= & $T\ N(T)\ \{ C \}$\\
$C$ ($c$,$c',\ldots$) & ::= & $S$ ; $C \mid \varepsilon$\\
$S$ ($s$,$s',\ldots$) & ::= & $R := \alimemx {E} \mid \alimemx {E} := R \mid$\\
& &  $R := E \mid \alifence \mid$\\
& & $\aliifelsex {E} {C} {C}\mid$\\ 
& & $\aliwhilex E C\mid \aliatomicx {C}$\\
& & $\aliassume\ E \mid \aliassert\ E \mid \aliskip$\\
$T$ & ::= & $\langle Bool\rangle \mid \langle Int\rangle$\\
$N$ & ::= & $\langle Name\rangle$\\
%$L$ & ::= & $\langle Label\rangle$\\
$R$ ($r$,$r',\ldots$) & ::= & $\aliregx i$ ($i\in\mathbb{N}$)\\
$E$ ($e$,$e',\ldots$) & ::= & $R \mid i \mid N(T) \mid \alitid \mid E+E \mid E-E \mid \ldots$\\
& & $E\wedge E \mid E\vee E \mid \neg E \mid \ldots $
\end{tabular}
\label{fig:program-grammar}
\caption{A simple programming language.}
\end{figure}

\newcommand{\aliprogstmt}{\ensuremath{\mathsf{Stmt}}}
\newcommand{\aliprogstmtx}[1]{\ensuremath{\aliprogstmt (#1)}}

\subsection{Programming Language}
\label{subsec:programming-language}
In this subsection, we formalize the programming language we are going to use in the rest of the paper.

\paragraph{Syntax.}
We use a simple programming language, whose grammar is given in Fig.~\ref{fig:program-grammar}.
The nonterminal $R$, ranged over by $r$, refers to {\em registers} and are used to model thread local address space.
The nonterminal $E$, ranged over by $e$, refers to well-formed {\em expressions} that can be written using registers, mathematical and logical operators, and the return values of procedures.
The shared data space is mapped by $\alimemx e$, where $e$ is an expression which evaluates to $\mathbb{N}$.
A 
Instructions can be used to read from memory ($r := \alimemx e$), update the contents of a memory location ($\alimemx e := e'$), empty the store buffer (\alifence), assign a value to a register ($r := e$).
The model control flow, we have the usual branching ($\aliifelsex e {e_1} {e_2}$) and looping ($\aliwhilex e$) instructions.
Additionally, we will also use an explicit blocking instruction ($\aliassume\ e$), which blocks as long as $e$ evaluates to $\alifalse$.
Finally, the instruction ($\aliassert\ e$) is used to claim that $e$ should hold whenever this instruction can execute.
Let $\aliprogstmtx P$ denote the set of statements used in program $P$.

\newcommand{\alilabel}{\ensuremath{\mathsf{Lab}}}
\newcommand{\alilabelx}[1]{\ensuremath{\alilabel(#1)}}
\newcommand{\alicontrol}{\ensuremath{\mathsf{Ctrl}}}
\newcommand{\alicontrolx}[1]{\ensuremath{\alicontrol(#1)}}
\newcommand{\alivaluation}{\ensuremath{\mathsf{Val}}}
\newcommand{\alivaluationx}[1]{\ensuremath{\alivaluation(#1)}}
\newcommand{\alisucc}{\ensuremath{\mathsf{Succ}}}
\newcommand{\alisuccx}[1]{\ensuremath{\alisucc(#1)}}
\newcommand{\aliprec}{\ensuremath{\mathsf{Prec}}}
\newcommand{\aliprecx}[1]{\ensuremath{\aliprec(#1)}}
\newcommand{\alifirst}{\ensuremath{\mathsf{Fst}}}
\newcommand{\alifirstx}[1]{\ensuremath{\alifirst(#1)}}
\newcommand{\alisequencex}[1]{\ensuremath{\vec{#1}}}
\newcommand{\alieval}{\ensuremath{\llbracket \rrbracket}}
\newcommand{\alievalx}[1]{\ensuremath{\llbracket #1 \rrbracket}}
\newcommand{\alibuffer}{\ensuremath{\mathsf{Buf}}}
\newcommand{\alibufferx}[1]{\ensuremath{\alibuffer(#1)}}
\newcommand{\alireadbuffer}{\ensuremath{\mathsf{RdBuf}}}
\newcommand{\alireadbufferx}[2]{\ensuremath{\alireadbuffer(#1,#2)}}
\newcommand{\aliatomiclock}{\ensuremath{\mathsf{lck}}}
\newcommand{\aliatomicexitlabel}{\ensuremath{l_{x}}}
\newcommand{\alienabled}{\ensuremath{\mathsf{Enb}}}
\newcommand{\alienabledx}[1]{\ensuremath{\alienabled(#1)}}
\newcommand{\alilts}{\ensuremath{\mathnormal{LTS}}}
\newcommand{\aliltsx}[1]{\ensuremath{\alilts(#1)}}
\newcommand{\alitrace}{\ensuremath{\mathsf{Tr}}}
\newcommand{\alitracex}[1]{\ensuremath{\alitrace(#1)}}
\newcommand{\alimemtrace}{\ensuremath{\mathsf{Mem}}}
\newcommand{\alimemtracex}[1]{\ensuremath{\alimemtrace(#1)}}
\newcommand{\aliosrules}{\ensuremath{Rules}}

\begin{figure*}[th]
\begin{mathpar}
\fbox{$(\alicontrol,\alivaluation,\alibuffer,\aliatomiclock)\xrightarrow{Rl,t:s}(\alicontrol',\alivaluation',\alibuffer',\aliatomiclock')$}\\

\inferrule*
{Rl=\textnormal{\sc\small Init} \\ M_i\in P \\ M_i= n(a)\ \{C\} \\ \alicontrolx t =\varepsilon}
{\alivaluation[(t,\alireg_{in})\mapsto a] \\ \alicontrol[t\mapsto \alifirstx C]}

\inferrule*
{Rl=\textnormal{\sc\small Rd} \\ \alicontrolx t = \alisequencex l \cdot l' \\ \alilabelx s = l' \\\\
s=r:=\alimemx e \\ \alienabledx t}
{\alivaluation[(t,r)\mapsto \alivaluationx {\alivaluation{\alievalx e}t}] \\ \alicontrol[t\mapsto \alisequencex l \cdot \alisuccx{l'}]}

\inferrule*
{Rl=\textnormal{\sc\small RdM} \\ \alicontrolx t = \alisequencex l \cdot l' \\ \alilabelx s = l' \\\\ 
s=r := \alimemx e \\ \alivaluation{\alievalx e}t \notin \alibufferx t \\ \alienabledx t}
{\alivaluation[(t,r)\mapsto \alivaluationx {\alivaluation{\alievalx e}t}] \\ \alicontrol[t\mapsto \alisequencex l \cdot \alisuccx{l'}]}

\inferrule*
{Rl=\textnormal{\sc\small RdB} \\ \alicontrolx t = \alisequencex l \cdot l' \\ \alilabelx s = l' \\\\ 
s=r := \alimemx e \\ \alivaluation{\alievalx e}t \in \alibufferx t \\ \alienabledx t}
{\alivaluation[(t,r)\mapsto \alireadbufferx t {\alivaluation{\alievalx e}t}] \\ \alicontrol[t\mapsto \alisequencex l \cdot \alisuccx{l'}]}


\inferrule*
{Rl=\textnormal{\sc\small Wr} \\ \alicontrolx t = \alisequencex l \cdot l' \\ \alilabelx s = l' \\\\
s=\alimemx e := r \\ \alienabledx t}
{\alivaluation[(t,r)\mapsto {\alivaluation \alievalx e}t] \\ \alicontrol[t\mapsto \alisequencex l \cdot \alisuccx{l'}]}

\inferrule*
{Rl=\textnormal{\sc\small WrB} \\ \alicontrolx t = \alisequencex l \cdot l' \\ \alilabelx s = l' \\\\ 
s=\alimemx e := r \\ \alibufferx t = \alisequencex {b} \\ \alienabledx t}
{\alibuffer[t\mapsto \alisequencex b \cdot(\alivaluation{\alievalx e}t,\alivaluationx {t,r})] \\ \alicontrol[t\mapsto \alisequencex l \cdot \alisuccx{l'}]}

\inferrule*
{Rl=\textnormal{\sc\small WrM} \\ \alibufferx t = (a,d)\cdot\alisequencex {b} \\ \alienabledx t}
{\alibuffer[t\mapsto \alisequencex b] \\ \alivaluation[a\mapsto d]}

\inferrule*
{Rl=\textnormal{\sc\small WrR} \\ \alicontrolx t = \alisequencex l \cdot l' \\ \alilabelx s = l' \\\\ 
s= r := e \\ \alienabledx t}
{\alivaluation[(t,r)\mapsto \alivaluation{\alievalx e}t] \\ \alicontrol[t\mapsto \alisequencex l \cdot \alisuccx{l'}]}

\inferrule*
{Rl=\textnormal{\sc\small Fnc} \\ \alicontrolx t = \alisequencex l \cdot l' \\ \alilabelx s = l' \\\\
s=\alifence \\ \alibufferx t = \varepsilon \\ \alienabledx t}
{\alicontrol[t\mapsto \alisequencex l \cdot \alisuccx{l'}]}

\inferrule*
{Rl=\textnormal{\sc\small IfT} \\ \alicontrolx t = \alisequencex l \cdot l' \\ \alilabelx s = l' \\\\
s=\aliifelsex e {c_1} {c_2} \\ \alivaluation{\alievalx e}t=\alitrue \\ \alienabledx t}
{\alicontrol[t\mapsto \alisequencex l \cdot \alisuccx{l'} \cdot \alifirstx {c_1}]}  

\inferrule*
{Rl=\textnormal{\sc\small IfE} \\ \alicontrolx t = \alisequencex l \cdot l' \\ \alilabelx s = l' \\\\
s=\aliifelsex e {c_1} {c_2} \\ \alivaluation{\alievalx e}t=\alifalse \\ \alienabledx t}
{\alicontrol[t\mapsto \alisequencex l \cdot \alisuccx{l'} \cdot \alifirstx {c_2}]}  

\inferrule*
{Rl=\textnormal{\sc\small WhI} \\ \alicontrolx t = \alisequencex l \cdot l' \\ \alilabelx s = l' \\\\
s=\aliwhilex e {c} \\ \alivaluation{\alievalx e}t=\alitrue \\ \alienabledx t}
{\alicontrol[t\mapsto \alisequencex l \cdot l' \cdot {\alifirstx {c}}]}

\inferrule*
{Rl=\textnormal{\sc\small WhE} \\ \alicontrolx t = \alisequencex l \cdot l' \\ \alilabelx s = l' \\\\
s=\aliwhilex e {c} \\ \alivaluation{\alievalx e}t=\alifalse \\ \alienabledx t}
{\alicontrol[t\mapsto \alisequencex l \cdot \alisuccx {l'}]}

\inferrule*
{Rl=\textnormal{\sc\small Skp} \\ \alicontrolx t = \alisequencex l \cdot l' \\ \alilabelx s = l' \\ s=\aliskip}
{\alicontrol[t\mapsto \alisequencex l \cdot \alisuccx {l'}]}

\inferrule*
{Rl=\textnormal{\sc\small AtB} \\ \alicontrolx t = \alisequencex l \cdot l' \\ \alilabelx s = l' \\\\
s=\aliatomicx c \\ \aliatomiclock = -1}
{\alicontrol[t\mapsto \alisequencex l \cdot \alisuccx {l'} \cdot \aliatomicexitlabel \cdot \alifirstx {c}] \\ \aliatomiclock = t}

\inferrule*
{Rl=\textnormal{\sc\small AtE} \\ \alicontrolx t = \alisequencex l \cdot \aliatomicexitlabel \\ \aliatomiclock=t}
{\alicontrol[t\mapsto \alisequencex l] \\ \aliatomiclock = -1}

\inferrule*
{Rl=\textnormal{\sc\small AsT} \\ \alicontrolx t = \alisequencex l \cdot l' \\ \alilabelx s = l' \\\\
s=\aliassertx{e} \\ \alivaluation{\alievalx e}t=\alitrue \\ \alienabledx t}
{\alicontrol[t\mapsto \alisequencex l \cdot \alisuccx {l'}]}

\inferrule*
{Rl=\textnormal{\sc\small AsF} \\ \alicontrolx t = \alisequencex l \cdot l' \\ \alilabelx s = l' \\\\
s=\aliassertx{e} \\ \alivaluation{\alievalx e}t=\alifalse \\ \alienabledx t}
{\aliatomiclock = -2}

\inferrule*
{Rl=\textnormal{\sc\small Asm} \\ \alicontrolx t = \alisequencex l \cdot l' \\ \alilabelx s = l' \\\\
s=\aliassumex{e} \\ \alivaluation{\alievalx e}t=\alitrue \\ \alienabledx t}
{\alicontrol[t\mapsto \alisequencex l \cdot \alisuccx {l'}]}
\end{mathpar}
\caption{Operational semantics for TSO and SC.}
\label{fig:operational-semantics}
\end{figure*}


\newcommand{\aliltsstates}{\ensuremath{Q}}
\newcommand{\aliltsstatesx}[1]{\ensuremath{\aliltsstates^{#1}}}
\newcommand{\aliltstransitions}{\ensuremath{Tr}}
\newcommand{\aliltstransitionsx}[1]{\ensuremath{\aliltstransitions^{#1}}}


\paragraph{Operational Semantics.}
A {\em labelled program} is a pair $(P,\alilabel)$ consisting of a program $P$ and a {\em labelling function} $\alilabel:\aliprogstmtx P \mapsto L$ mapping each statement of $P$ to a unique {\em label} in the set of labels, $L$.
When no confusion is likely to arise, we use $P$ to refer also to a labelled program in which case $\alilabel$ is implicit.
We use a labelled transition system (LTS) to define the semantics of a labelled program under either TSO or SC.

Two statements $s$, $s'$ in $P$ are {\em consecutive} if $s;s'$ appears in $P$.
In such a case, $s'$ is the {\em sequential successor} of $s$ and $s$ the {\em sequential predecessor} of $s'$, denoted as $s'=\alisuccx s$ and $s=\aliprecx {s'}$.
If $s$ has no sequential successor, we let $\alisuccx s=\varepsilon$.
A {\em code block} is any code ($C$) delimited by two matching braces.
The unique first statement in a non-empty code block is referred to as $\alifirstx C$.
Observe that if $s$ is the last statement in a code block, then $\alisuccx s=\varepsilon$.

A {\em program state} of $P$ is a tuple $(\alicontrol,\alivaluation,\alibuffer,\aliatomiclock)$.
The {\em control flow} $\alicontrol:T \mapsto L^*$ maps each thread in the set of thread identifiers $T$ to some sequence over labels.
Intuitively, $\alicontrol$ encodes the execution context for each thread in the program.
If $\alicontrolx t=\alisequencex {l}$, then $\alisequencex {l} \langle 1,|\alisequencex {l}|-1\rangle$ keeps the nesting history and $\alisequencex {l}[|\alisequencex {l}|]$ is the label of the next statement to be executed by thread $t$.
The {\em valuation} $\alivaluation:(\mathbb{N}\cup T\times R)\mapsto \mathbb{N}$ maps each memory location and register of each thread to its content, assumed to be a non-negative integer.
Intuitively, $\alivaluation$ holds the storage state: the contents of each register of each thread and the contents of the (single) global memory.
The contents of a memory location $i$, $\alimemx i$, is accessed by $\alivaluationx {i}$, and the content of some register $r$ used by a thread $t$ is accessed by $\alivaluationx {t,r}$.
The {\em buffer view} $\alibuffer:T \mapsto (\mathbb{N}\times\mathbb{N})^*$ maps each thread to a sequence over pairs of natural numbers.
Intuitively, $\alibufferx t=(a_1,d_1)\ldots(a_k,d_k)$ means that the store buffer of thread $t$ contains $k$ buffered writes, the oldest being to location $a_1$ of value $d_1$ and the newest being to location $a_k$ of value $d_k$.
We use $a\in\alibufferx t$ to denote the fact that there is $j\in[1,k]$ with $a_j=a$.
In such a case, let $\alireadbufferx t a$ denote $d_j$ such that $a_j=a$ and for all $i>j$, $a_i\neq a$.
Finally, the {\em lock state} $\aliatomiclock$ is a variable ranging over $\{-1,-2\}\cup T$.
Intuitively, $\aliatomiclock$ is $-1$ if no thread is currently executing an atomic block, is $t$ if thread $t$ is executing an atomic block.
When $\aliatomiclock$ is set to $-2$, it denotes an assertion violation on occurrence of which all threads are blocked.
The transition relation of program $P$ is given in Fig.~\ref{fig:operational-semantics}.

\newcommand{\aliruns}{\ensuremath{\mathbf{R}}}
\newcommand{\alirunsx}[2]{\ensuremath{\aliruns_{#1}(#2)}}


\paragraph{Program runs.}
A {\em run} of program $P$ is an alternating sequence of states and transition labels conforming to the operational semantics of Fig.~\ref{fig:operational-semantics}.
Formally, a run $\alisequencex r$ is a sequence $q_0t_1q_1\ldots t_kq_k$, where $q_i$'s are states of $\aliltsx P$ and for each $i\in[1,k]$, $q_{i-1}\xrightarrow{t_i}q_i$ is a transition of $\aliltsx P$.
The sequence of transitions $t_1\ldots t_k$ is called the {\em trace} of the run, denoted by $\alitracex {\alisequencex r}$.

A program run is {\em TSO compliant} if it does not contain the {\sc \small Wr} transition.
Similarly, a program run is {\em SC compliant} if it does not the {\sc \small WrB, WrM} transitions.
For notational convenience we leave {\sc \small Fnc} rules in SC compliant runs.
It is easy to show that in SC compliant runs {\alifence} and {\aliskip} are interchangeable. 
Let $\alirunsx {tso} P$ denote the set of all TSO compliant runs of $P$.
Similarly, let $\alirunsx {sc} P$ denote the set of all SC compliant runs of $P$.

\begin{figure*}[th]
\[
\alimemtracex {q,(Rl,t:s)} \stackrel{def}{=} 
 \begin{cases}
  \alireadx t {q.\alivaluation{\alievalx e}t} {q.\alivaluationx {q.\alivaluation{\alievalx e}t}} & ,\ Rl\in\{\textnormal{\sc\small Rd,RdM}\},\ s=r:=\alimemx e\\
%  \alireadx t {q.\alivaluation{\alievalx e}t} {q.\alivaluationx {q.\alivaluation{\alievalx e}t}} &,\ Rl=\textnormal{\sc\small RdM},\ s=r:=\alimemx e\\
  \alireadx t {q.\alivaluation{\alievalx e}t} {q.\alireadbufferx t {q.\alivaluation{\alievalx e}t}} &,\ Rl=\textnormal{\sc\small RdB},\ s=r:=\alimemx e\\
  \aliwritex t {q.\alivaluation{\alievalx e}t} {q.\alivaluationx {t,r}} &,\ Rl=\textnormal{\sc\small Wr},\ s=\alimemx e:= r\\
  \locwritex t {q.\alivaluation{\alievalx e}t} {q.\alivaluationx {t,r}} &,\ Rl=\textnormal{\sc\small WrB}\\
  \remwritex t a d &,\ Rl=\textnormal{\sc\small WrM},\ q.\alibufferx t=(a,d)\cdot \alisequencex b\\
  \alibarrierx t &,\ Rl=\textnormal{\sc \small Fnc}\\
  \varepsilon &,\ \textnormal{otherwise}
 \end{cases}
\]
\caption{Mapping transitions to memory operations.}
\label{fig:trans-to-memory}
\end{figure*}


\paragraph{Memory traces.}
In order to relate program runs to executions of the previous section, we define a mapping $\alimemtrace:\aliltsstatesx P\times\aliltstransitionsx P \rightarrow \tsoalph$ in Fig.~\ref{fig:trans-to-memory}.

Let $\alisequencex r=q_0t_1q_1\ldots t_kq_k$ be a run.
The memory trace of $\alisequencex r$ is the sequence $\alimemtracex {q_0,t_1}\cdot \alimemtracex {q_1,t_2} \cdot \ldots \cdot \alimemtracex {q_{k-1},t_k}$.
With an abuse in notation, we let $\alimemtracex {\alisequencex r}$ denote the memory trace of $\alisequencex r$.
The following result establishes the link between runs and executions.

\begin{proposition}
Let $\alisequencex r$ be a TSO (resp. SC) compliant run.
Then $\alimemtracex {\alisequencex r}$ is a TSO-execution (resp. SC-execution).
\end{proposition}


% !TEX root = main-tsoreduction.tex
\section{Reduction for TSO}
\label{sec:reduction-for-tso}
In this section, we explain how the reduction theory of Lipton can be used for TSO.
Our goal is to present sufficient conditions for programs such that when a program satisfies these conditions the program is guaranteed to be unable to distinguish TSO semantics from SC semantics. 
This is the first step of applying reduction to TSO programs.
In the following sections, we will show how we can extend it to programs which are initially TSO distinguishing.

\begin{definition}[Movers]
\label{def:movers}
Let $P$ be a labelled program and let $\aliequivgeneric$ be an equivalence relation over $\alirunsx {tso} P$.
Let $s$ be a statement that occurs in some method's body $m$ of $P$.
Then, $s$ is {\em left-mover in $\aliequivgeneric$} if for any $\alisequencex r=\alisequencex r[1]\ldots\alisequencex r[k]\in\alirunsx {tso} P$ we have $\alisequencex r[i]\alisequencex r[i+1]=(R',t':s')(R,t:s)$ such that $\alimemtracex {s'} \alipotsox {\alimemtracex {\alisequencex r}} \alimemtracex s$ does not hold, then there exists a run $\alisequencex {r'}$ in $[\alisequencex r]_{\aliequivgeneric}$ such that $\alisequencex {r'}\langle 1,i\rangle=\alisequencex r\langle 1,i-1\rangle \alisequencex r[i+1]$.

Similarly, $s$ is {\em right-mover in $\aliequivgeneric$} if for any $\alisequencex r=\alisequencex r[1]\ldots\alisequencex r[k]\in\alirunsx {tso} P$ we have $\alisequencex r[i-1]\alisequencex r[i]=(R,t:s)(R',t':s')$ such that $\alimemtracex {s} \alipotsox {\alimemtracex {\alisequencex r}} \alimemtracex {s'}$ does not hold, then there exists a run $\alisequencex {r'}$ in $[\alisequencex r]_{\aliequivgeneric}$ such that $\alisequencex {r'}\langle 1,i\rangle=\alisequencex r\langle 1,i-2\rangle \alisequencex r[i]\alisequencex r[i-1]$.
\end{definition}

Intuitively, $s$ is left mover (resp. right mover) if reordering $s$ before (resp. after) any other statement that is concurrent with $s$ does not change the behavior (all runs belonging to the same equivalence have the same behavior).
In the classic definition of reduction, the equivalence relation requires that the two sequences be permutations of each other and that the end states are identical, which corresponds to $\tsoequivstrict$ in the TSO context.
The reason for the added generality will become clear once we generalize reduction with abstraction.

Our first result about movers follows immediately from definitions.
\begin{lemma}
Let $P$ be a labelled program.
All of its TSO runs are SC-like if all remote writes are left-movers in $\tsoequivstrict$.
Dually, all of its TSO runs are SC-like if all local actions (excluding remote writes) are right-movers in $\tsoequivstrict$.

All of its TSO runs are observationally SC-like if all remote writes are left-movers in $\tsoequiv$.
Dually, all of its TSO runs are observationally SC-like if all local actions (excluding remote writes) are right-movers in $\tsoequiv$.
\end{lemma}
This result depends on a strong constraint which is unlikely to be satisfied by many programs.
In what follows we will provide a series of incrementally more general results.
Fix a labelled program $P=\{m_1,\ldots,m_n\}$.
\begin{lemma}
All TSO runs of $P$ are SC-like if for each method $m_i$, either all of its local actions are right-movers in $\tsoequivstrict$ or all of its remote write actions are left-movers. 
\end{lemma}
This result helps us to reason about methods individually.
We have the following corollary, which is also implied by triangular-race freedom of \cite{Owe2009}.
\begin{corollary}
Let $P$ be such that each $m_i$ either only updates the shared memory (no read actions) or only reads shared memory (no write actions).
Then all of its TSO runs are SC-like.
\end{corollary}

Now we will investigate the impact of placing fence statements in restricting the non-SC-like behaviors of programs.
For the following fix $s^{loc}$ and $s$ as matching local and remote write actions.
\begin{lemma}
If in any TSO run in which $s^{loc}$ is executed by some thread $t$, all the combined write actions executed by $t$ up to the occurrence of $s^{loc}$ are atomic and $s$ is left-mover, then $s^{loc}$ and $s$ are atomic.
\end{lemma}
This naturally leads to the following special instance which gives an insight about how fence statements lead to SC-like programs.
\begin{corollary}
If in any TSO run in which $s^{loc}$ is executed by thread $t$, there exists a fence action executed by $t$ preceding $s^{loc}$, no other local write actions by $t$ occur between the two actions and $s$ is left-mover, then $s^{loc}$ and $s$ are atomic.
\end{corollary}
This means that, unlike the general case, in order to argue that a write immediately following a fence statement is atomic, we only need to prove that its matching remote write action is left-mover. 
There is a dual of this result which we state next.
\begin{lemma}
If in any TSO run in which $s^{loc}$ is executed by thread $t$, there does not exist any read action executed by $t$ until the occurrence of $s$, then $s^{loc}$ is right-mover and $s^{loc}$ and $s$ are atomic.
\end{lemma}
The next result which is a special case of the previous result explains why the use of fence statements restricts the runs of a TSO program to SC-like behaviors.
\begin{corollary}
Let $s;C;\alifence$ be a code block in some $m_i$ such that $C$ does not contain any read of shared memory (no statements of the form $r:=\alimemx e$).
Then $s$ and all the writes in $C$ are atomic.
\end{corollary}


{\sc Code snippets demonstrating the results.}


% !TEX root = main-tsoreduction.tex
\section{Abstracting TSO programs}
\label{sec:abstracting-tso-programs}
The main problem with the existing body of work on TSO program verification is the impossibility of handling programs which do contain TS-specific runs, i.e. programs with triangular races.
Our approach so far allows us to at least alleviate some of the difficulties in reasoning by showing that certain write statements can be taken to be atomic.
In this section, we go one step further and show abstraction can be used to turn a program with TSO-specific runs into one that contains only SC-like runs.

%The use of abstraction in reduction was successfully demonstrated in~\cite{EQT2009} in the context of sequentially consistent programs.
%Since the soundness of the method crucially depends on the atomicity of each action, it was not clear how one can adopt those techniques to weaker memory models.
%Here we address and resolve the issue of non-atomic writes for TSO.
In order to define program abstraction, we need to define a new equivalence relation among TSO runs.
%The idea is to make sure that equivalent runs have identical computations even though they might have different statements.
Let us call two TSO-runs of equal length $\alisequencex r$ and $\alisequencex {r'}$ {\em computationally equal}, written $\alisequencex r \alicompequiv \alisequencex {r'}$, if for all $i$, $\tau[i]=R,t:s$ and $\tau'[i]=R',t':s'$  imply $R=R'$, $t=t'$, $s$ and $s'$ have the same label, expressions in both $s$ and $s'$ evaluate to the same value, and in the case of an assignment, the left hands are the same.
Intuitively, computationally equal runs will result in identical execution paths, even though there may be syntactic differences between the executed statements.
\begin{definition}[Program Abstraction]
Let $P$ and $P'$ be two programs.
We say that $P'$ {\em abstracts} $P$, if one of the following holds:
\begin{itemize}
\item $P'$ has a failed run, or
\item $P$ does not have a failed run and for each terminated run $\alisequencex r\in\alirunsx {tso} P$, there exists a terminated run $\alisequencex {r'}\in\alirunsx {tso} {P'}$ such that $\alisequencex r\alicompequiv \alisequencex {r'}$.
\end{itemize}
\end{definition}
Intuitively, $P'$ abstract $P$ if $P'$ contains an assertion violation or has more behaviors than $P$.
This means that if $P'$ can be proven to contain no assertion violations, then neither does $P$.
We should note that the other direction, that when $P$ does not contain an assertion violation neither should $P'$, does not hold in general.

\begin{table}
\begin{tabular}{ll}
%$\aliabsrulex {InsertAssert}$ & $s \leadsto \aliatomicx {\aliassertx e; s}$\\
%$\aliabsrulex {WeakAssume}$ & $\aliassumex e \leadsto \aliassumex e'$\\
%& \{provided $e\Rightarrow e'$\}\\
$\aliabsrulex {ValNondet}$ & $r := \alimemx e \leadsto r := \alihavocval$\\
& $r := e \leadsto r := \alihavocval$\\ 
& $\alimemx e := r \leadsto \alimemx e := \alihavocval$\\
$\aliabsrulex {CtrlNondet}$ & 
 \begin{tabular}[t]{ll}
 $s \leadsto$ & $\aliif\ {\alihavocval}$\\
 & $\alithen\ \mathtt{\{}\aliassumex e; s'\mathtt{\}}$\\
 & $\alielse\ \mathtt{\{}\aliassumex e'; s''\mathtt{\}}$\\
 & \{provided $e\vee e'$ is tautology, and\\ 
 & $s'$ and $s''$ are abstractions of $s$\}
\end{tabular}
\end{tabular}
\caption{Substitution rules guaranteeing sound abstraction.}
\label{tab:abs-rules}
\end{table}

\paragraph{Abstraction rules.}
There are many ways to ensure that a syntactic manipulation of $P$ results in another program $P'$ abstracting the former.
In Table~\ref{tab:abs-rules}, we list two such possible rules which are essentially individual statement replacements that provide a sound abstraction.

%The first substitution rule, {\aliabsrulex {InsertAssert}}, is the insertion of an assertion to any statement. 
%Recall that an assert statement either does nothing (if its predicate evaluates to true) or results in a failed run.
%Both cases trivially satisfy the conditions of abstraction.

%A dual rule is to weaken assumptions, the rule {\aliabsrulex {WeakAssume}}.
%In this case, the assume statement with a weaker predicate will allow for more behaviors.

The first rule, {\aliabsrulex {ValNondet}}, introduces non-deterministic reads or writes.
Let $\alihavocval$ be an expression that can evaluate to any integer value.
\footnote{Essentially, with the introduction of $\alihavocval$ we are changing the evaluation operator $\alieval$ from a mapping from expressions to values to a mapping from expression to sets of values.}
Then replacing any expression with $e$ with $\alihavocval$ is another abstraction.

The other rule {\aliabsrulex {CtrlNondet}}, introduces non-deterministic control flow.
The idea is to replace a statement $s$ by an if-then-else statement such that the branches are not necessarily mutually exclusive, i.e. certain states may satisfy both $e$ and $e'$ of the rule in Table~\ref{tab:abs-rules}.
This is a sound abstraction because no matter which branch is taken (and at any state at least one of these branches are enabled) whatever $s$ was doing, possibly more, in the original program will be done.  
For a detailed exposition of how abstraction in conjunction with reduction leads to a natural style of reasoning in safety proofs, the reader can consult~\cite{EQT2009}.

\paragraph{Why it works?}
Before going into examples, let us briefly discuss how abstraction leads to an SC-like program. 
Consider the template of a triangular race as given in~\cite{Owe2010} below:
\begin{eqnarray*}
&\locwritex t A 1 \cdot \alireadx t B 0 \cdot \locwritex u B 2 \cdot \remwritex u B 2\\
&\ \cdot\ \alireadx v B 2 \cdot \alireadx v A 0 \cdot \remwritex t A 1
\end{eqnarray*}
In this TSO-execution, $t$ observes the initial value for $B$, whereas $v$ observes the write to $B$ by $u$ before the write to $A$ by $t$ because the remote write action of $t$ comes after both of the reads of $v$.
In order to construct an equivalent SC-execution, both statements of $t$ should precede the statement of $u$ because the read of $B$ by $t$ returns the initial value 0.
However, for the reads of $v$ to be consistent the write to $B$ should occur before the write to $A$, which contradicts the previous requirement.

Looking more closely into the execution, we see that it is impossible to make the write to $A$ by $t$ atomic. 
Recall that for a local and remote write pair to be atomic, it must be possible to partition into two the sequence of actions that separate the two write actions such that the local write moves to the right of every action of the left partition and the remote write moves to the left of every action of the right partition.
In the given memory trace, $\locwritex t A 1$ is a right-mover but is followed by a read of $B$ which cannot move to the right of $\remwritex u B 2$.
This implies that $\remwritex t A 1$ should move to the left of every action up to $\locwritex u B 2$.
But this is impossible because it cannot move to the left of $\alireadx v A 0$.

Now consider the abstraction which removes the read of $B$ by $t$.
In other words, if the read of $B$ by $t$ is due to the statement $s$={\tt \aliregx 1:=$B$}, consider the program obtained by replacing $s$ with {\tt \aliregx 1:=\alihavocval}.
This essentially removes the read of $B$ by $t$ from the memory trace since the abstract read does not depend on any write.
This in turn means that instead of the above execution, we will have the following:
\begin{eqnarray*}
&\locwritex t A 1 \cdot \locwritex u B 2 \cdot \remwritex u B 2\\
&\ \cdot\ \alireadx v B 2 \cdot \alireadx v A 0 \cdot \remwritex t A 1
\end{eqnarray*}
which can be brought into an atomic equivalent form as:
\begin{eqnarray*}
&\locwritex u B 2 \cdot \remwritex u B 2\\
&\ \cdot\ \alireadx v B 2 \cdot \alireadx v A 0 \cdot \locwritex t A 1 \cdot \remwritex t A 1
\end{eqnarray*}
Of course, this is a simple example where the read value by $t$ is not used, so the abstraction might make sense trivially. 
We next work through several simple templates which demonstrate the power of abstraction: transforming a program into one that is SC-like which can be used to prove safety properties of interest.
 
\subsection{Send/Receive}
\label{subsec:send-receive}

\begin{figure}[h]
\begin{tabular}{p{.2\textwidth}p{.2\textwidth}}
\begin{alltt}Recv()
 \(Rdy\):=1;
 r[2]:=\(Buf\);
 if r[2]=0 then
   r[1]:=\(Flag\);
   while r[1]=0 
     r[1]:=\(Flag\);
   r[2]:=\(Buf\);\end{alltt}
&
\begin{alltt}Send(\(d\))
 \(Buf\):=\(d\);
 r[1]:=\(Rdy\);
 while r[1]=0 
   r[1]:=\(Rdy\);
 \(Flag\):=1;\end{alltt}
\end{tabular}
\caption{Sender/Receiver template with a triangular race.}
\label{fig:send-receive}
\end{figure}

Consider the code given in Fig.~\ref{fig:send-receive}, which represents a standard synchronization pattern, where a sender sets a flag after having prepared its message.
The intended operation proceeds as follows:
The sender begins by preparing the message (represented by writing $d$ into $Buf$), and then spins on $Rdy$.
After observing $Rdy$ equal to 1 it sends the message ready signal by setting $Flag$ to 1.
The receiver begins by setting $Rdy$ (initially 0) to 1 to tell the sender that it is ready to receive a message.
Before spinning on $Flag$, the receiver reads the current content of $Buf$ (0 means message not in yet).
If it observes a non-zero value denoting a valid message, it skips the entire spinning block.
Otherwise, it spins on $Flag$ and after observing it to be equal to 1, it reads the message from $Buf$.
Let us consider the set of TSO programs in which two threads $t$ and $u$ run {\tt Recv} and {\tt Send(42)} respectively.

\paragraph{Triangular race.}
This program does have a triangular race depicted by the following memory trace
\begin{eqnarray*}
& \locwritex t {Rdy} 1 \cdot\ \alireadx t {Buf} 0\cdot\ \locwritex u {Buf} d \cdot\ \remwritex u {Buf} d\\
&\cdot\ \alireadx u {Rdy} 0\cdot\ \remwritex t {Rdy} 1
\end{eqnarray*}
The first three actions, the local write to $Rdy$ and read of $Buf$ by $t$ followed by a write to $Buf$ by $u$ gives the triangular race.
The remaining part of the execution shows how the race can be extended into a non SC-like run. 

\paragraph{Removing the triangular race.}
Following the discussion about the general case regarding triangular races, one might be tempted to abstract the read of $Buf$ by $t$ altogether.
However, that would make the rest of the code behave incorrectly since a non-zero value for $Buf$ means a valid message which need not be equal to what $u$ is about to write, $d$.

Let us instead consider the following abstraction for the statement {\tt $\aliregx 2$:=$Buf$}:
\begin{eqnarray*}
&&\aliif\ {\alihavocval}\\
&&\alithen\ \aliatomicx {\aliassumex {Buf=0};\ \aliregx 2:=Buf;}\\
&&\alielse\ \aliatomicx {\aliassumex {Buf\neq0};\ \aliregx 2:=\alihavocval;}
\end{eqnarray*}
This replacement is an instance of {\sc\small CtrlNondet} and thus leads to a sound abstraction.
We claim that in the new abstract program $P'$ the write to $Rdy$ by {\tt Recv} is atomic.
%In order to prove this claim we have to show that the memory trace of any run of $P'$ has an equivalent SC-execution.
Let $\alisequencex r$ be a run in $\alirunsx {tso} {P'}$ whose memory trace is of the form
\[
\alpha\cdot\ \locwritex t {Rdy} 1\cdot\ \tau\cdot\ \remwritex t {Rdy} 1\cdot\ \beta
\]
where $\alpha$, $\tau$ and $\beta$ are sequences of memory actions.
We have to show that there are sequences of actions $\tau_1$ and $\tau_2$ such that $\tau=\tau_1\tau_2$, $\locwritex t {Rdy} 1$ moves to the right of every action in $\tau_1$, and $\remwritex t {Rdy} 1$ moves to the left of every action in $\tau_2$. 
We set $\tau_1=\tau$ and $\tau_2=\varepsilon$ and show that the following memory trace belongs to a run of $P'$.
\[
\alpha\cdot\ \tau\cdot\ \locwritex t {Rdy} 1\cdot\ \remwritex t {Rdy} 1\cdot\ \beta
\]

First, observe that $\tau_1$ cannot contain the write to $Flag$ by $u$ because that action only happens when $u$ ends its spinning by reading 1 from $Rdy$ which can only happen in $\beta$, i.e. after the remote write action $\remwritex t {Rdy} 1$.
This in turn means that the farthest $t$ can go in $\tau$ is the reading of $Flag$.
Any read action is a right (and left) mover with respect to another action not writing to the location read.
Thus, in case $\tau_1$ contains read(s) of $Flag$ they can all move to right until $\remwritex t {Rdy} 1$.

Since a local write action is always a right mover, we are left with the read of $Buf$ by $t$.
If the remote write action $\remwritex u {Buf} d$ is not in $\tau$, then we are done.
Assume that \remwritex u {Buf} d happens after some prefix $\tau_p$ of $\tau$, i.e.
\[
\alpha\cdot\ \locwritex t {Rdy} 1\cdot\ \tau_p\cdot\ \remwritex u {Buf} d\cdot\ \tau_q\cdot\ \remwritex t {Rdy} 1\cdot\ \beta
\]
If the read of $Buf$ by $t$ happens in $\tau_q$, we are done.
So assume that the read of $Buf$ happens in $\tau_p$.
Since it happens before the remote write action, the state at which the abstract read statement occurs must have $Buf=0$.
According to the abstraction, this means that the {\tt then} branch must have been taken, which sets $\aliregx 2$ to 0.
If the read action moves to the right of $\remwritex u {Buf} d$, then because $Buf\neq 0$ holds after the remote write, the {\tt else} branch will be taken instead.
That in turn implies that $\aliregx 2$ can be assigned any value, including 0. 
So, the sequence
\[
\alpha\cdot\ \cdot\ \tau_p'\cdot\ \remwritex u {Buf} d\cdot\ \tau_q'\cdot\ \locwritex t {Rdy} 1\cdot\ \remwritex t {Rdy} 1\cdot\ \beta
\]
where $\tau_p'$ does not contain any action by $t$ is also a memory trace of $P'$.
Since $\tau$ was taken to be arbitrary, we conclude that the write to $Rdy$ by $t$ is atomic.

\paragraph{Execution context.}
Observe that our argument also establishes the remote write action to $Buf$ as a left mover. 
However, there is a fundamental difference between the two arguments.
Since we proved the atomicity of the write to $Rdy$ by showing that all actions of $t$ until the remote write action to $Rdy$ occurs are right-movers, our result does not depend on the client's state, in particular the store buffer state. 
On the other hand, because our argument for the atomicity of the write to $Buf$ is based on showing that its remote write action is a left-mover, the result depends on the client's store buffer state. 
For instance, if both threads' buffers contained remote write actions to the same location before calling {\tt Recv} and {\tt Send} then we will not be able to move the remote write to $Buf$ next to its matching local write.
In other words, if one wants to have atomicity of these writes in any execution context, {\tt Recv} must end with a fence whereas {\tt Send} must begin with a fence.



% !TEX root = main-tsoreduction.tex
\section{Mechanical Verification of Write Atomicity}
\label{sec:mechanical-verification}
In this section, we show how write atomicity can be mechanically checked.
We begin by describing a transformation which maps TSO programs to equivalent SC programs.
The asynchronous nature of the buffered updates is captured by a splitting of each thread into two threads, one executing the local actions, the other executing the remote actions.
The queue semantics of each store buffer is captured by program order over remote actions enforced by SC.
%The results we have presented thus far are used to verify the atomicity of any statement in the transformed program under SC semantics.
We illustrate our methodology by verifying the atomicity of writes in double checked initialization. 
%In particular, we show how mover types of certain statements can be used to conclude that a program cannot have any TSO distinguishing behavior.
%Finally, we define an abstraction relation among TSO programs.
%This abstraction forms the pillar of a reasoning methodology which enables one to start with a program with TSO distinguishing behavior and to end with a more abstract program which is SC-like.


\begin{figure*}
\begin{tabular}{p{.35\textwidth}p{.3\textwidth}p{.34\textwidth}}
\begin{alltt}\(RdL(e,r,l)\) \{  
 cnt:=ExecCnt[\(l\)][tid]+1;
 ver:=AdrVer[\(e\)][tid];
 if ver>AdrVer[\(e\)][tido]
 then \{\(r\):=WrVal[e][ver][tid]; \}
 else \{\(r\):=mem[\(e\)];
 RdVal[l][cnt][tid]:=r;
 ExecCnt[\(l\)][tid]:=cnt;
\}

\(RdR(r,l)\) \{
 cnt:=ExecCnt[\(l\)][tid]+1;
 \(\aliassume\) cnt<=ExecCnt[\(l\)][tido];
 r := RdVal[\(l\)][cnt][tido];
 ExecCnt[\(l\)][tid]:=cnt;
\}
\end{alltt} &

\begin{alltt}\(WrL(e,r,l)\) \{
 cnt:=ExecCnt[\(l\)][tid]+1;
 ver:=AdrVer[\(e\)][tid]+1;
 WrVal[\(e\)][ver][tid]:=r;
 AdrVer[\(e\)][tid]:=ver;
 ExecCnt[\(l\)][tid]:=cnt;
\}

\(WrR(e,l)\) \{
 cnt:=ExecCnt[\(l\)][tid]+1;
 \(\aliassume\) cnt<=ExecCnt[\(l\)][tido];
 ver:=AdrVer[\(e\)][tid]+1;
 mem[e]:=WrVal[\(e\)][cnt][tido];
 AdrVer[\(e\)][tid]:=ver;
 ExecCnt[\(l\)][tid]:=cnt;
\}\end{alltt} & 

\begin{alltt}
\(FenceL\) \{
 \(\aliassume \forall e.\) 
   AdrVer[\(e\)][tid]
        ==
   AdrVer[\(e\)][tido];
\}

\(FenceR\) \{
 \(\aliskip\);
\}\end{alltt}
\end{tabular}
\caption{The TSO to SC transformation macros.}
\label{fig:transformation-macros}
\end{figure*}

\subsection{Program Transformation}
\label{subsec:program-transformation}
Let $P=(\{m_1,\ldots,m_n\},\alilabel)$ be a program.
For notational convenience, we let $l_s$ to denote the label $l$ of $s$, i.e. $\alilabelx s=l$.
For a thread identifier set $T$, let $T'$ denote a primed duplicate of $T$ such that $t\in T$ iff $t'\in T'$.
The {\em split transformation} of $P[T]$ is another program $\alisplitprogx P[T\cup T']=(\{m^{loc}_1,m^{rem}_1,\ldots,m^{loc}_n,m^{rem}_n\},\alilabel')$.
Intuitively, this transformation separates local writes from remote writes by way of adding a new auxiliary thread.
The execution of method $m_i$ by some thread $t$ will be simulated by the execution of the methods $\aliloctransx {m_i}$ and $\aliremtransx {m_i}$ by threads $t$ and $t'$, respectively.
For notational convenience, we let $(t')'=t$.
A transition (or corresponding action) is {\em local} if it is executed by some $t\in T$; otherwise, it is called {\em remote}.
We will omit the explicit mention of $T$ and $T\cup T'$ in the following.
%All statement macros keep track of how many times a statement with label $l$ is executed, the value in {\tt StExec[$l$][tid]}.

In order to explicitly split the local and remote writes, we introduce several arrays:
\begin{itemize}
\item {\tt AdrVer[$i$][$t$]} holds the version number of updates to $\alimemx i$ by thread $t$.
\item {\tt WrVal[$i$][$v$][$t$]} holds the value written to $\alimemx i$ with version number $v$ by thread $t$.
\item {\tt RdVal[$l$][$c$][$t$]} holds the value observed when $t$ executed the statement with label $l$ for the $c^{th}$ time.
\item {\tt ExecCnt[$l$][$t$]} holds the number of times the statement with label $l$ is executed by $t$.
%\item {\tt Other[$t$]} holds the thread identifier matching with $t$ (see below).
\end{itemize}

Let $s$ be a statement in $m_i$.
The {\em local transformation} and the {\em remote transformation} of $s$, written as $\aliloctransx s$ and $\aliremtransx s$ respectively, are given below:
\begin{eqnarray*}
 \aliloctransx s  & \stackrel{def}{=} &
  \begin{cases}
   \aliatomicx {RdL(e,r,l_s)} & , s=r:=\alimemx e\\
   \aliatomicx {WrL(e,r,l_s)} & , s=\alimemx e:= r\\
   \aliatomicx {FenceL} & , s=\alifence\\
   s & , \textnormal{otherwise}
  \end{cases}\\
 \aliremtransx s & \stackrel{def}{=} &
  \begin{cases}
   \aliatomicx {RdR(e,r,l_s)} & , s=r:=\alimemx e\\
   \aliatomicx {WrR(e,r,l_s)} & , s=\alimemx e:= r\\
   \aliatomicx {FenceR} & , s=\alifence\\
   s & , \textnormal{otherwise}
  \end{cases}
\end{eqnarray*}
By convention, we let $\aliloctransx \varepsilon=\aliremtransx \varepsilon=\varepsilon$.
We let $\alilabel'$ be such that $\alilabel'(\aliloctransx s)=l_s$ and $\alilabel'(\aliremtransx s)=l_s'$.
The definitions of the macros $RdL$, $RdR$, etc. are given in Fig.~\ref{fig:transformation-macros}.
Both of these transformations are extended to code blocks respecting the program order.
Formally, if $C=s;C'$ is a code block, $\aliloctransx C$ and $\aliremtransx C$ are given by $\aliloctransx s; \aliloctransx {C'}$ and $\aliremtransx s; \aliremtransx {C'}$, respectively.
This way we identify each method $m_i=\{C_i\}$ with two methods $m^{loc}_i=\{\aliloctransx {C_i}\}$ and $m^{rem}_i=\{\aliremtransx {C_i}\}$ by applying $\aliloctrans$ and $\aliremtrans$ to each statement in $m_i$.




\paragraph{Read macros.}
A read statement $s=r:=\alimemx e$ by $t$ is simulated by $t$ executing $\aliatomicx{RdL(e,r,l_s)}$ and $t'$ executing $\aliatomicx{RdR(r,l_s)}$.
The local read starts by reading the execution count $c$ for $s$ and update count to {\alimemx e} by $t$.
It then checks whether it should read from the {\em buffer} or from the {\em memory}.
If the number of updates done by $t$ (local write actions) and the number of updates done by $t'$ (remote write actions) are equal, then memory is read; otherwise, the latest local write to $\alimemx e$ is read, the value in {\tt WrVal[$e$][ver][$t$]}.
It registers the value it read with the current execution count into {\tt RdVal[$l_s$][$c$][$t$]}.
It ends by incrementing the statement execution count.

A remote read starts by reading the execution count $c$ of $s$ by $t'$.
It then makes sure that the current statement is enabled by checking whether the remote execution count of $s$ is at most equal to local execution count of the same statement (by $t$).
If the assumption is satisfied, then the value as read by the local thread, the value in {\tt RdVal[$l_s$][$c$][$t$]}, is assigned to $r$.
It ends by incrementing the statement execution count.

Note that, since all macros are within {\aliatomic} blocks, either the whole macro executes or none of it. 
This implies that the remote read macro can only execute after its associated local read macro has executed.
An alternative transformation would be to make the local and remote read operations tightly coupled: each local read macro is immediately followed by its associated remote read macro.
This would have the advantage of avoiding the assume statement for read macros.
For the sake of uniformity, though, since such assume statements are essential in capturing the asynchronous nature of remote writes, we opted for this encoding.

\paragraph{Write macros.}
A write statement $s=\alimemx e:=r$ by $t$ is simulated by $t$ executing $\aliatomicx{WrL(e,r,l_s)}$ and $t'$ executing $\aliatomicx{WrR(e,l_s)}$.
The local write starts by reading the current statement count $c$ for $s$ and update count $v$ to {\alimemx e} by $t$.
It then registers the current update value into {\tt WrVal[$e$][$v$][$t$]}.
It ends by incrementing the update and statement execution counts.

The remote write starts by reading the current statement count $c$ for $s$ by $t'$.
It then ensures that the current macro is enabled by checking that the number of times the local counterpart of the statement is executed is at least as big as $c$.
If that is the case, the remote update count for $e$ is updated and the value written by the local counterpart, the value in {\tt WrVal[$e$][$c$][$t$]}, is written into location $\alimemx e$.
It ends by incrementing the update and statement execution counts done by $t'$.

\paragraph{Fence macros.}
A fence statement $s=\alifence$ by $t$ is simulated by $t$ executing $\aliatomicx{FenceL}$ and $t'$ executing $\aliatomicx{FenceR}$.
Recall that $t$ can execute $\alifence$ only when its store buffer is empty, i.e. all the remote write actions corresponding to the local write actions by $t$ have occurred.
In order to ensure this, local fence blocks until for every location $e$, the local update count is equal to the remote update count (note that the remote update count can never exceed the local update count). 
The remote fence is simply set to be no-op, i.e. \aliskip.

It might seem counter-intuitive to give the remote thread the capability of moving beyond a fence statement when the local has not executed yet, but as we shall see where exactly the remote fence occurs is inconsequential to the soundness argument.

%We set $T'$ be the set of {\em primed} thread identifiers derived from $T$, such that $t'\in T'$ iff $t\in T$.
In order to avoid inessential technicality, we modify the operational semantics.
We update the rule {\sc\small Init} to account for the local-remote pairing threads.
\begin{mathpar}
\inferrule*
{Rl=\textnormal{\sc\small Init'} \\ M_i\in P \\ M_i= n(a)\ \{C\} \\ \alicontrolx t =\varepsilon}
{\alivaluation[(t,\alireg_{in})\mapsto a] \\ \alicontrol[t\mapsto \alifirstx {\tau_l(C)}, t'\mapsto \alifirstx {\tau_r(C)}]}
\end{mathpar}  
We change the definition of {\alienabledx t} as follows:
\[
\alienabledx t \stackrel{def}{=} (\aliatomiclock=-1) \ \vee\ (\aliatomiclock=t)\ \vee\ (\aliatomiclock=t')
\]
Finally, we introduce a new keyword {\tt tido} such that for any valuation $\alivaluation$ and thread $t$, we have $\alivaluation\alievalx {\mathtt{tido}} t=t'$.

\newcommand{\locseqequiv}{\ensuremath{\sim_l}}






\subsection{Soundness of Transformation}
\label{subsec:soundness}
In this subsection, we prove that the program transformation defined above is sound.
That is, $P$ running under TSO semantics is equivalent to $\alisplitprogx P$ running under SC semantics.
To that end, we define an equivalence relation $\locseqequiv$ over the SC runs of $\alisplitprogx P$, assign a representative to each $\locseqequiv$ class, and establish the bijective relation between the representatives and the TSO runs of $P$.

Let us begin with an important observation about the orderings of local and remote transitions.
\begin{lemma}\label{lem:transformation-bijection}
Let $P$ be a program and $\alisplitprogx P$ be its transformation.
Then, in any terminated SC-compliant run of $\alisplitprogx P$, there is a bijection $\mu$ between the local and remote actions such that the assignments to registers are identical.
Furthermore, if $x_l$ is a local read or write transition and $x_r=\mu(x_l)$ then $x_l$ occurs before $x_r$.
\end{lemma}
\begin{proof}[Proof (Sketch)]
The bijection is constructed by pairing the $k^{th}$ local transition executed by $t$ with the $k^{th}$ remote transition executed by $t'$.
Observe that the only time the two threads can diverge is when they read from memory, i.e. a {\sc\small Rd} transition.
But in that case, the value read by the local thread is registered and the remote read gets that value.
The second part follows from the use of {\aliassume} statements in the remote actions and the bijection can be established by matching each local transition with a remote transition such that they both observe the same (unique) value for {\tt cnt}.
\end{proof}

Let $\alisequencex r$ and $\alisequencex {r'}$ be permutationally equivalent SC-compliant runs of $\alisplitprogx P$.
They are {\em locally equivalent}, $\alisequencex {r}\locseqequiv \alisequencex {r'}$, if the projections of their traces to local actions and remote write actions are identical.
Formally, $\alisequencex r\locseqequiv \alisequencex {r'}$ iff $\alisequencex r\aliperm \alisequencex {r'}$ and
\[
\alitracex {\alisequencex r}\downarrow_{\{-,T:-\}\cup\{\textnormal{\sc\small Wr},T':-\}} = \alitracex {\alisequencex {r'}}\downarrow_{\{-,T:-\}\cup\{\textnormal{\sc\small Wr},T':-\}}
\]
%where $\downarrow_{\{R,A:S\}}$ is the projection operator which removes all transitions $r,t:s$ such that $r\notin R$ or $t\notin T$ or $s\notin S$ and $-$ stands for empty set.

The run $\alisequencex {r}$ is called {\em canonical} if for all $i<|\alisequencex {r}|$, $t\in T$ with $\alitracex {\alisequencex {r}}[i]=R,t:\aliloctransx s$, the following conditions are satisfied:
\begin{itemize}
\item if $R\neq\textnormal{\sc\small Wr}$, then $\alitracex {\alisequencex {r}}[i+1]=R,t':\aliremtransx s$.
\item if $s=\alifence$, then $\alitracex {\alisequencex {r}}[i+1]=\textnormal{\sc\small Skp},t':\aliremtransx s$.
\end{itemize}
Informally, a run is canonical if all local actions (transitions executed by unprimed threads) are immediately followed by their matching remote actions, except for write actions.

\begin{lemma}\label{lem:canonical}
Let $\alisequencex {r}$ be a TSO-compliant run of $P$.
Then there exists a unique canonical run in $[\alisequencex {r}]_{\locseqequiv}$.
\end{lemma}
\begin{proof}[Proof (Sketch)]
A remote transition either is completely defined by its matching local transition (when $R=\textnormal{\sc\small Rd}$) or by its local state (when $R\notin\{\textnormal{\sc\small Wr},\textnormal{\sc\small Rd}\}$).
This means that all memory independent transitions $R,t':\aliremtransx s$ are both-movers relative to all transitions but those done by $t'$.
All read transitions $\textnormal{\sc\small Rd},t':\aliremtransx s$ are left-movers and by construction cannot come before their matching local actions.
This implies that all transitions can be rearranged to obtain a canonical run within the same equivalence class.
Uniqueness comes from the fact that within the equivalence class the relative ordering among local actions and remote writes is preserved which means that in the canonical sequence each remote action has a well-defined position.
\end{proof}


We extend the definition of computationally equal to enable a comparison between TSO-runs of $P$ and the SC-runs of $\alisplitprogx P$ such that the requirement for equality between states with matching index values is replaced by an equality between the components $\alivaluation$, $\alicontrol$ both restricted to $T$ and $\aliexecmode$ of the states.
We conclude by stating the main result of this section.

\begin{theorem}\label{thm:soundness}[Soundness of Transformation]
Let $P$ be a program, $\alisequencex r$ be a TSO-run of $P$.
Then, for every TSO-run of $P$ there exists a computationally equal canonical SC-run of $\alisplitprogx P$,
and for every SC-run of $\alisplitprogx P$, there exists a computationally equal TSO-run of $P$. 
\end{theorem}
\begin{proof}[Proof (Sketch)]
Let $\alisequencex r$ be a TSO-compliant run of $P$.
We construct a computatinally equal SC-run $\alisequencex {r'}$ of $\alisplitprogx P$.
Every {\sc\small Init} transition of $\alisequencex r$ is replaced with an {\sc\small Init'} transition.
Every $R,t:s$ with $R\in \{\textnormal{\sc\small RdM},\textnormal{\sc\small RdB}\}$ is replaced with $\textnormal{Rd},t:\aliloctransx s$ followed by $\textnormal{Rd},t:\aliremtransx s$.
Every $\textnormal{\sc\small WrB},t:s$ is replaced with $\textnormal{\sc\small Wr},t:\aliloctransx s$.
Every $\textnormal{\sc\small WrM},t:s$ is replaced with $\textnormal{\sc\small Wr},t':\aliremtransx s$.
Finally, all other transitions $R,t:s$ are replaced with $R,t:\aliloctransx s$ followed by $R,t':\aliremtransx s$.
That such a run exists follows from Lemma~\ref{lem:transformation-bijection}.

For the other direction, it is enough to consider only canonical SC-runs by Lem.~\ref{lem:canonical}.
Replace all adjacent local and remote transitions $R,t:\aliloctransx s$ followed by $R,t':\aliremtransx s$ with $R,t:s$, when $R\neq \textnormal{\sc\small Wr}$ and $s\neq\alifence$.
If $s=\alifence$, then replace the transitions $\textnormal{\sc\small Asm},t:\aliloctransx s$ followed by $\textnormal{\sc\small Skp},t':\aliremtransx s$ with $\textnormal{\sc\small Fnc},t:s$.
If $R=\textnormal{\sc\small Wr}$, then replace $R,t:\aliloctransx s$ with $\textnormal{\sc\small WrB},t:s$, and replace $R,t':\aliremtransx s$ with $\textnormal{\sc\small WrM},t:s$.
\end{proof}






\subsection{Example - Double Checked Initialization}
\label{subsec:example-double-check}

\begin{figure}[ht]
\begin{alltt}DoubleCheckInit() 
\{
 r:=\(Obj\);
 \(\aliif\) r=0 \(\alithen\) 
 \{
  \(\aliatomic\) \{ \(\aliassume\) {\(Lck\)=0}; \(Lck\):=tid; \(\alifence\); \}
  r:=\(Obj\);
  \(\aliif\) r=0 \(\alithen\) \{ r:=42; \(Obj\):=r; \}
  \(Lck\):=0;
 \}
\}\end{alltt}
\caption{The code for double check initialization. The parameter {\tt a} holds the address for {\tt Obj}.}
\label{fig:double-check}
\end{figure}

Consider the code given in Fig.~\ref{fig:double-check}, which is derived from double checked initialization.
The objective of the procedure is to ensure that a shared object $Obj$ is initialized exactly once.
In order to simplify presentation, we assume that $Obj$ is initialized to 42 known at compile time.

The procedure starts by checking whether the object of interest $Obj$ has already been initialized.
If $Obj$ has indeed been initialized, the procedure terminates.
Otherwise, the procedure switches to the initialization phase.
This phase starts by acquiring a (global) lock which protects the initialization operation on the object.
If the lock is available, which is when $Lck$ is equal to 0, it is acquired by setting $Lck$ to the thread identifier of the current thread, {\tt tid}, and flushing the contents of the buffer ensuring the global visibility of the lock being acquired.
Interestingly enough, this is the only place in the code where a fence is used.

After successfully acquiring the lock, the current state of $Obj$ is checked again.
This is necessary to ensure that no other (concurrent thread) has initialized $Obj$ in the meantime.
If $Obj$ is initialized by some other thread, the lock is released by resetting $Lck$ back to 0 and the procedure terminates.
Otherwise, $Obj$ is assigned to 42, which is followed by the release of the lock and the termination of the procedure.

\paragraph{Challenges.}
In order to show that the write to $Obj$ is atomic, one has to prove that its remote write is a left-mover.
In its current form, write to $Obj$ conflicts with all the accesses to $Obj$ executed by other threads.
It is possible to remove two of those conflicts by annotating the $Lck$ protected code with an auxiliary variable which is set to {\tt tid} during lock acquire and reset to 0 after release and asserting that during the read and the write actions in the lock protected region the auxiliary variable must be equal to {\tt tid}.\footnote{This is standard approach used in proving atomicity of lock protected regions using reduction; see~\cite{EQT2009}.} 

The other conflict cannot be removed by simple annotation.
As a matter of fact, due to the unprotected read, the program is TSO-specific.
Thus, we need to abstract the program which we do similar to the abstraction done in the Send/Receive example of the previous section.
After the abstraction, the remote write action is established as a left-mover, which ends up showing that the following program is SC-like.\footnote{To avoid clutter, we omit the annotations for the lock region.}

\begin{alltt}DoubleCheckInit() 
\{
 if \(Obj=0\) then \{ r:=\(Obj\);\} else \{ r:=\(\alihavocval\); \}
 \(\aliif\) r=0 \(\alithen\) 
 \{
  \(\aliatomic\) \{ \(\aliassume\) {\(Lck\)=0}; \(Lck\):=tid; \(\alifence\); \}
  r:=\(Obj\);
  \(\aliif\) r=0 \(\alithen\) \{ r:=42; \(Obj\):=r; \}
  \(Lck\):=0;
 \}
\}\end{alltt}

Note that the abstract program can now be used to prove that $Obj$ is initialized at most once under SC semantics, which implies that the original program under TSO semantics was correct.



\COMMENT{
\section{Discussion}
It is not fair to claim that the work presented in this paper exists in a vacuum disconnected with what exists and has no openings to new related domains.
We end the paper by arguing to the contrary on both accounts.

\paragraph{Mover Analysis in SC.}
This is perhaps not surprising: There is an intimate connection between atomicity proofs for programs under SC semantics and the write atomicity proofs we present here.
We should first identify the subtle difference between the two usages of {\em atomicity}, which regrettably has acquired an ambiguous status.
A program is atomic if every interleaving execution of its methods is equivalent to a sequential execution in which methods execute in isolation. 
This definition of atomicity can be applied to code blocks, which is what reduction theory is essentially used for under SC semantics.
It is typical practice to prove that a particular write or read is a mover to increase the size of an atomic block, thereby decreasing the number of possible interleavings.

One natural question to ask is whether we can do what is being done for SC atomicity proofs to show that a program $P$ is TRF, without even using the split transformation.
That is, can we use the mover type of a write statement in $P$ under SC semantics to deduce that it is atomic under TSO semantics?
The answer to this question is in general negative.
Consider the occurrences of the local and remote write actions of a write statement $s$ under TSO semantics:
\[
\alpha\ \cdot\ \locwritex t x v\ \cdot\ \gamma\ \cdot\ \remwritex t x v\ \cdot\ \beta
\]
The way the tool {\sc qed} determines the mover type of an action is to check whether that action moves to the left (right) of every other concurrent action.
The use of assertions aims at capturing refining the set of potential concurrent actions, i.e. an over-approximation of the actions that might occur in $\gamma$.
This leads us to a couple of {\em backward compatibility} results, i.e. porting mover types under SC to TSO.
\begin{lemma}
Let $P$ be a TSO program and $s$ be a statement in it.
Assume that $s$ is proved to be left-mover in {\sc qed} under SC semantics (uses the same $P$ replacing all {\alifence} statements with {\aliskip}). 
\begin{itemize}
\item If $s$ is a simple write statement, $s$ is atomic under TSO semantics.
\item if $s$ is of the form $\aliatomicx {\aliassertx {\varphi}; w}$, $\varphi$ is a predicate which can be falsified by write statements none of which is ever buffered, then $w$ is atomic under TSO semantics.
\end{itemize}
\end{lemma}
\begin{proof}[Proof (Sketch)]
If $s$ is a simple write statement, then it cannot move to the left of another action by introducing a failure.
This implies that it does not conflict with any other statement in the program.
The additional statements in TSO semantics do not introduce additional conflicts (a write under SC and its remote action under TSO have the same interference relative to concurrent actions).

In the second case, the assertion helps the checker to narrow down the set of concurrent actions as explained above.
Since the remote write 
\end{proof}

\paragraph{Other Relaxed Memory Models.}
What makes a memory model relaxed is



{\sc Ali stopped here.}
}

%%%%%%%%%%%%%%%%%%%%%%%%%%%%%%%%%%%%%%%%%%%%%%%%%%%%%%
%%%%%%%%%%%%%%%%%%%%%%%%%%%%%%%%%%%%%%%%%%%%%%%%%%%%%%
%%%%%%%%%%%%%%%%%%%%%%%%%%%%%%%%%%%%%%%%%%%%%%%%%%%%%%
%%%%%%%%%%%%%%%%%%%%%%%%%%%%%%%%%%%%%%%%%%%%%%%%%%%%%%
%%%%%%%%%%%%%%% Ali stopped here  %%%%%%%%%%%%%%%%%%%%
%%%%%%%%%%%%%%%%%%%%%%%%%%%%%%%%%%%%%%%%%%%%%%%%%%%%%%
%%%%%%%%%%%%%%%%%%%%%%%%%%%%%%%%%%%%%%%%%%%%%%%%%%%%%%
%%%%%%%%%%%%%%%%%%%%%%%%%%%%%%%%%%%%%%%%%%%%%%%%%%%%%%
%%%%%%%%%%%%%%%%%%%%%%%%%%%%%%%%%%%%%%%%%%%%%%%%%%%%%%






\section{Conclusion}
In this paper, we show how TSO programs can be analyzed using reduction theory.
We argue that an explicit reasoning based on the local/remote dichotomy of write actions that exist in TSO semantics allows us to investigate the problematic from a different perspective. 
Besides adopting well-established SC analysis concepts such as atomicity of statements to TSO programs, we are also able to distinguish between different types of atomicity (right-mover local write vs. left-mover remote write) which has consequences for fence placement to ensure proper usage.
As for programs with non-SC behaviors, we propose a systematic methodology to convert them into programs which have only SC behaviors at the expense of increased non-determinism.
We also develop a new transformation from TSO programs to equivalent SC programs which is amenable to mechanical checking of write atomicity.

We do not think that we have comprehensively listed all interesting results for TSO when reduction is the main intellectual tool.
We believe that many specific execution contexts will benefit from a detailed analysis of mover types of local and remote actions.
These analyses may lead to less cost in synchronization, not only less number of fence statements but also correct use of optimistic and racy reads which have been very tricky to reason about. 
This is a line of work which we are planning to work on.

Our approach admittedly, at least at the theoretical level, deals with TSO semantics, not to prove the property one might be interested in, but rather to check whether that property can be verified under SC semantics.
Be that it may, it is still interesting and potentially quite rewarding if one can specify sufficient conditions the satisfaction of which guarantees the mover type of a write under SC semantics carries over to TSO semantics.
In other words, when can we claim that a remote write is a left-mover when its corresponding statement under SC semantics is shown to be left-mover?

Finally, and definitely the most important extension to our work is to investigate reduction for other relaxed memory models.
As we said before, in principle reduction theory can be applied for any memory model as long as it can be modelled using interleaving semantics.
Other relaxed memory models would not be as easy to formulate for reduction as TSO, mainly because not only writes are even less atomic (each thread may observe an arbitrary interleaving of other threads' remote writes), but also program order which both TSO and SC respect is relaxed.
All of this withstanding, reduction theory might prove to be an elegant tool to capture data race freedom theorems for various relaxed memory models.

%\appendix
%\section{Appendix Title}

%This is the text of the appendix, if you need one.

%\acks

%Acknowledgments, if needed.

% We recommend abbrvnat bibliography style.

\bibliographystyle{abbrvnat}
\begin{thebibliography}{10}

\bibitem{EQT2009}
T.~Elmas, S.~Qadeer, and S.~Tasiran.
\newblock A calculus of atomic actions.
\newblock In {\em POPL}, pages(2-15), 2009.

\bibitem{Lam1979}
L. Lamport.
\newblock How to make a multiprocessor computer that correctly executes multiprocess programs.
\newblock In {\em IEEE Trans. Comp.}, C-28(9): pages(690–691), 1979.

\bibitem{AG1996}
S.~Adve and K.~Gharachorloo.
\newblock Shared memory consistency models: a tutorial.
\newblock In {\em Computer}, 29(12): pages(66–76), 1996.

\bibitem{SJM+2007}
V.A.~Saraswat, R.~Jagadeesan, M.~Michael and C.~Von Praun.
\newblock A theory of memory models.
\newblock In {\em PPoPP}, ACM: pages(161-172), 2007.

\bibitem{Hil1998}
M.~Hill.
\newblock Multiprocessors should support simple memory-consistency models.
\newblock In {\em IEEE Computer}, 31(8): pages(28–34), 1998.

\bibitem{OSS2009}
S.~Owens, S.~Sarkar and P~Sewell.
\newblock A better x86 memory model: x86-TSO.
\newblock In {\em TPHOLs}, pages(391-407), 2009.

\bibitem{Owe2010}
S.~Owens.
\newblock Reasoning about the implementation of concurrency abstractions on x86-TSO.
\newblock In {\em ECOOP}, 2010.

\bibitem{BSS2010}
J.~Burnim, K.~Sen and C.~Stergiou.
\newblock Sound and complete monitoring of sequential consistency in relaxed memory models.
\newblock In {\em Tech. Rep. UCB/EECS} page(31), 2010.

\bibitem{LW2010}
A.~Linden and P.~Wolper.
\newblock  An automata-based symbolic approach for verifying programs on relaxed memory models.
\newblock In {\em SPIN}, pages(212-226), 2010.

\bibitem{LW2013}
A.~Linden and P.~Wolper.
\newblock  A verification-based approach to memory fence insertion in PSO memory systems.
\newblock In {\em TACAS}, pages(339-353), 2013.

\bibitem{GJJ2006}
T. L.~Gall, B.~Jeannet and T.~Jron.
\newblock Verification of communication protocols using abstract interpretation of FIFO queues. 
\newblock In {\em AMAST}, pages(204–219), 2006

\bibitem{PD1995}
S.~Park and D. L.~Dill. 
\newblock An executable specification, analyzer and verifier for RMO (relaxed memory order).
\newblock In {\em SPAA}, pages(34–41), 1995.

\bibitem{HR2006}
T.~Huynh and A.~Roychoudhury.
\newblock A memory model sensitive checker for CSharp. 
\newblock In {\em Formal Methods (FM)}, LNCS 4085, pages(476–491). Springer, 2006.

\bibitem{DPN1993}
D.~Dill, S.~Park, and A.~Nowatzyk. 
\newblock Formal specification of abstract memory models.
\newblock In {\em Symposium on Research on Integrated Systems}, pages(38–52). MIT Press, 1993.

\bibitem{BAM2006}
S.~Burckhardt, R.~Alur, and M.~Martin. 
\newblock Bounded verification of concurrent data types on relaxed memory models: A case study. 
\newblock In {\em Computer-Aided Verification (CAV)}, LNCS 4144, pages(489–502). Springer, 2006.

\bibitem{BAM2007}
S.~Burckhardt, R.~Alur, and M.~Martin. 
\newblock CheckFence: Checking consistency of concurrent data types on relaxed memory models.
\newblock In {\em PLDI}, pages(12–21), 2007.

\bibitem{GYS2004}
G.~Gopalakrishnan, Y.~Yang, and H.~Sivaraj.
\newblock QB or not QB: An efficient execution verification tool for memory orderings.
\newblock In {\em CAV}, LNCS 3114, pages(401–413), 2004.

\bibitem{YGL+2004}
Y.~Yang, G.~Gopalakrishnan, G.~Lindstrom, and K.~Slind.
\newblock Nemos: A framework for axiomatic and executable specifications of memory consistency models. 
\newblock In {\em IPDPS}, 2004.

\bibitem{BM2008}
S.~Burckhardt and M.~Musuvathi.
\newblock Effective program verification for relaxed memory models.
\newblock In {\em Technical Report MSR-TR-2008-12}, Microsoft Research, 2008.

\bibitem{KVY2010}
M.~Kuperstein, M.~Vechev and E.~Yahav.
\newblock Automatic inference of memory fences.
\newblock In {\em FMCAD}, pages(111–119), 2010.

\bibitem{KVY2011}
M.~Kuperstein, M.~Vechev and E.~Yahav.
\newblock Partial-coherence abstractions for relaxed memory models.
\newblock In {\em PLDI}, pages(187-198), 2011.

\bibitem{VY2008}
M.~Vechev and E.~Yahav.
\newblock Deriving linearizable fine-grained concurrent objects.
\newblock In {\em PLDI}, pages(125–135), 2008.

\bibitem{VYY2010}
M.~Vechev, E.~Yahav and G.~Yorsh.
\newblock Abstraction-guided synthesis of synchronization.
\newblock In {\em POPL}  pages(327–338), 2010.

\bibitem{VYB+2007}
M.~Vechev, E.~Yahav, D. F.~ Bacon and N.~Rinetzky.
\newblock CGCExplorer: a semi-automated search procedure for provably correct concurrent collectors.
\newblock In {\em PLDI} pages(456-467), 2007.

\bibitem{VYY2009}
M.~Vechev, E.~Yahav and G.~Yorsh.
\newblock Inferring synchronization under limited observability. 
\newblock In {\em TACAS}, pages(139–154), 2009.

\bibitem{DMV+2013}
A.~Dan, Y.~Meshman, M.~Vechev and E.~Yahav.
\newblock Predicate Abstraction for Relaxed Memory Models. 
\newblock In {\em SAS}, pages(84–104), 2013.

\bibitem{MDV+2014}
Y.~Meshman, A.~Dan, M.~Vechev and E.~Yahav.
\newblock Synthesis of Memory Fences via Refinement Propagation.
\newblock In {\em SAS}, pages(), 2014.

\bibitem{LNP+2012}
F.~Liu, N.~Nedev, N.~Prisadnikov, M.~Vechev and E.~Yahav.
\newblock Dynamic synthesis for relaxed memory models. 
\newblock In {\em PLDI}, pages(429-440), 2012.

\bibitem{ND2013}
B.~Norris and B.~Demsky.
\newblock CDSchecker: checking concurrent data structures written with C/C++ atomics
\newblock In {\em OOPSLA}, pages(131-150), 2013.

\bibitem{FLM2003}
X. Fang, J. Lee, and S. Midkiff.
\newblock Automatic fence insertion for shared memory multiprocessing.
\newblock In {\em ICS}, pages(285–294), 2003.

\bibitem{LP2001}
J.~Lee and D. A.~Padua.
\newblock Hiding relaxed memory consistency with a compiler.
\newblock In {\em IEEE Trans. Comput.}, pages(824–833), 2001.

\bibitem{SS1998}
D.~Shasha and M.~Snir.
\newblock Efficient and correct execution of parallel programs that share memory.
\newblock In {\em ACM Trans. Program. Lang. Syst.}, pages(282–312), 1988.

\bibitem{AMS+2010}
J.~Alglave, L.~Maranget, S.~Sarkar and P.~Sewell. 
\newblock Fences in Weak Memory Models.
\newblock In {\em CAV}, pages(258-272), 2010.

\bibitem{AKN+2014}
J.~Alglave, D.~Kroening, V.~Nimal and D.~Poetzl. 
\newblock  Don't Sit on the Fence - A Static Analysis Approach to Automatic Fence Insertion.
\newblock In {\em CAV}, pages(508-524), 2014.

\bibitem{AKN+2013}
J.~Alglave, D.~Kroening, V.~Nimal and M.~Tautschnig. 
\newblock   Software Verification for Weak Memory via Program Transformation.
\newblock In {\em ESOP}, pages(512-532), 2013.

\bibitem{FBP2011}
M.~Faouzi, A.~Bouajjani and G.~Parlato.
\newblock Getting Rid of Store-Buffers in TSO Analysis.
\newblock In {\em CAV}, pages (99-115), 2011.

\bibitem{AAC+2013}
P. A.~Abdulla, M. F.~Atig, Y. F.~Chen, C.~Leonardsson and A.~Rezine.
\newblock Memorax, a Precise and Sound Tool for Automatic Fence Insertion under TSO.
\newblock In {\em TACAS}, pages(530-536), 2013.

\bibitem{AAC+2012}
P. A.~Abdulla, M. F.~Atig, Y. F.~Chen, C.~Leonardsson and A.~Rezine.
\newblock Automatic fence insertion in integer programs via predicate abstraction.
\newblock In {\em SAS}, pages(164-180), 2012.

\bibitem{HJM+2002}
T.A.~Henzinger, R.~Jhala, R.~Majumdar and G.~Sutre.
\newblock Lazy abstraction
\newblock In {\em POPL}, pages(58-70), 2002.

\bibitem{VN2011}
V.~Vafeiadis and F.Z.~Nardelli.
\newblock Verifying fence elimination optimisations
\newblock In {\em SAS}, pages(146-162), 2011.

\bibitem{BDM2013}
A.~Bouajjani, E.~Derevenetc and R.~Meyer.
\newblock Checking and Enforcing Robustness against TSO.
\newblock In {\em ESOP}, pages(533-553), 2013.

\bibitem{Rid2010}
T.~Ridge
\newblock A Rely-Guarantee proof system for x86-TSO.
\newblock In {\em VSTTE}, pages(55-70), 2010.

\bibitem{JLP+2014}
S.~Jagannathan, V.~Laporte, G.~Petri, D.~Pichardie and J.~Vitek.
\newblock Atomicity refinement for verified compilation.
\newblock In {\em PLDI}, pages(27-27), 2014.

\bibitem{KPH2010}
E.~Koskinen, M.~Parkinson and M.~Herlihy.
\newblock Coarse-grained transactions
\newblock In {\em POPL}, pages(19-30), 2010.

\bibitem{AC2009}
F.~Aleen and N.~Clark.
\newblock Commutativity analysis for software parallelization: letting program transformations see the big picture.
\newblock In {\em ASPLOS}, pages(241-252), 2009.

\bibitem{RD1997}
M.C.~Rinard and P.C.~Diniz.
\newblock Commutativity analysis: a new analysis technique for parallelizing compilers.
\newblock In {\em TOPLAS}, pages(942-991), 1997.

\bibitem{PGZ+2011}
P.~Prabhu, S.~Ghosh, Y.~Zhang, N.P.~Johnson and D.I.~August.
\newblock Commutative set: a language extension for implicit parallel programming.
\newblock In {\em PLDI}, pages(1-11), 2011.

\bibitem{Lip1975}
R.J.~Lipton.
\newblock Reduction: a method of proving properties of parallel programs.
\newblock In {\em CACM}, pages(717-721), 1975.

\bibitem{LFF2012}
H.~Liang, X.~Feng and M.~Fu.
\newblock A rely-guarantee-based simulation for verifying concurrent program transformations.
\newblock In {\em POPL}, pages(455-468), 2012.

\bibitem{JW2011}
A.J.~Turon and M.~Wand.
\newblock A separation logic for refining concurrent objects.
\newblock In {\em POPL}, pages(247-258), 2011.

\bibitem{Fen2009}
X.~Feng.
\newblock Local rely-guarantee reasoning.
\newblock In {\em POPL}, pages(315-327), 2009.

\bibitem{SSO+2010}
P.~Sewell, S.~Sarkar, S.~Owens, F.Z.~Nardelli and M.O.~Myreen.
\newblock x86-TSO: a rigorous and usable programmer's model for x86 multiprocessors.
\newblock In {\em CACM}, pages(89-97), 2010.

\bibitem{BA2008}
H.J.~ Boehm and S.V.~Adve.
\newblock Foundations of the C++ concurrency memory model.
\newblock In {\em PLDI}, pages(68-78), 2008.

\bibitem{BOS+2011}
M.~Batty, S.~Owens, S.~Sarkar, P.~Sewell and T.~Weber.
\newblock Mathematizing C++ concurrency.
\newblock In {\em POPL}, pages(55-66), 2011.

\bibitem{BWB+2011}
J.C.~Blanchette, T.~Weber, M.~Batty, S.~Owens and S.~Sarkar.
\newblock Nitpicking c++ concurrency.
\newblock In {\em PPDP}, pages(113-124), 2011.

\bibitem{SSA+2011}
S.~Sarkar, P.~Sewell, J.~Alglave, L.~Maranget and D.~Williams.
\newblock Understanding POWER multiprocessors.
\newblock In {\em PLDI}, pages(175-186), 2011.

\bibitem{HMS+2012}
S.M.~Haim, L.~Maranget, S.~Sarkar, K.~Memarian, J.~Alglave, S.~Owens, R.~Alur, M.M.K.~Martin, P.Sewell and D.~Williams.
\newblock An axiomatic memory model for POWER multiprocessors.
\newblock In {\em CAV}, pages(495-512), 2012.

\bibitem{AFI+2009}
J.~Alglave, A.Fox, S.Ishtiaq, M.O.~Myreen, S.Sarkar, P.Sewell and F.Z.~Nardelli.
\newblock The semantics of power and ARM multiprocessor machine code.
\newblock In {\em DAMP}, pages(13-24), 2009.

\bibitem{SVN+2013}
J.~Sevcik, V.~Vafeiadis, S.~Jagannathan and P.~Sewell.
\newblock CompCertTSO: A Verified Compiler for Relaxed-Memory Concurrency.
\newblock In {\em JACM}, 2013.




\end{thebibliography}




%\bibliography{tso-reduction-biblio}
% The bibliography should be embedded for final submission.


\end{document}





%%%%%%%%%%%%%%%%%%%%%%%%%%%%%%%%%%%%%%%%%%%%%%%%%%%%%%
%%%%%%%%%%%%%%%%%%%%%%%%%%%%%%%%%%%%%%%%%%%%%%%%%%%%%%
%%%%%%%%%%%%%%%%%%%%%%%%%%%%%%%%%%%%%%%%%%%%%%%%%%%%%%
%%%%%%%%%%%%%%%%%%%%%%%%%%%%%%%%%%%%%%%%%%%%%%%%%%%%%%
%%%%%%%%%%%%%%%%%%%%%%%%%%%%%%%%%%%%%%%%%%%%%%%%%%%%%%
%%%%%%%%%%%%%%%%%%%%%%%%%%%%%%%%%%%%%%%%%%%%%%%%%%%%%%
%%%%%%%%%%%%%%%%%%%%%%%%%%%%%%%%%%%%%%%%%%%%%%%%%%%%%%
%%%%%%%%%%%%%%%%%%%%%%%%%%%%%%%%%%%%%%%%%%%%%%%%%%%%%%
%%%%%%%%%%%%%%%%%%%%%%%%%%%%%%%%%%%%%%%%%%%%%%%%%%%%%%
%%%%%%%%%%%%%%%%%%%%%%%%%%%%%%%%%%%%%%%%%%%%%%%%%%%%%%
%%%%%%%%%%%%%%%%%%%%%%%%%%%%%%%%%%%%%%%%%%%%%%%%%%%%%%
%%%%%%%%%%%%%%%%%%%%%%%%%%%%%%%%%%%%%%%%%%%%%%%%%%%%%%
%%%%%%%%%%%%%%%%%%%%%%%%%%%%%%%%%%%%%%%%%%%%%%%%%%%%%%
%%%%%%%%%%%%%%%%%%%%%%%%%%%%%%%%%%%%%%%%%%%%%%%%%%%%%%
%%%%%%%%%%%%%%%%%%%%%%%%%%%%%%%%%%%%%%%%%%%%%%%%%%%%%%
%%%%%%%%%%%%%%%%%%%%%%%%%%%%%%%%%%%%%%%%%%%%%%%%%%%%%%
%%%%%%%%%%%%%%%%%%%%%%%%%%%%%%%%%%%%%%%%%%%%%%%%%%%%%%
%%%%%%%%%%%%%%% Comments after this point %%%%%%%%%%%%
%%%%%%%%%%%%%%% Comments after this point %%%%%%%%%%%%
%%%%%%%%%%%%%%% Comments after this point %%%%%%%%%%%%
%%%%%%%%%%%%%%% Comments after this point %%%%%%%%%%%%
%%%%%%%%%%%%%%% Comments after this point %%%%%%%%%%%%
%%%%%%%%%%%%%%% Comments after this point %%%%%%%%%%%%
%%%%%%%%%%%%%%%%%%%%%%%%%%%%%%%%%%%%%%%%%%%%%%%%%%%%%%
%%%%%%%%%%%%%%%%%%%%%%%%%%%%%%%%%%%%%%%%%%%%%%%%%%%%%%
%%%%%%%%%%%%%%%%%%%%%%%%%%%%%%%%%%%%%%%%%%%%%%%%%%%%%%
%%%%%%%%%%%%%%%%%%%%%%%%%%%%%%%%%%%%%%%%%%%%%%%%%%%%%%
%%%%%%%%%%%%%%%%%%%%%%%%%%%%%%%%%%%%%%%%%%%%%%%%%%%%%%
%%%%%%%%%%%%%%%%%%%%%%%%%%%%%%%%%%%%%%%%%%%%%%%%%%%%%%
%%%%%%%%%%%%%%%%%%%%%%%%%%%%%%%%%%%%%%%%%%%%%%%%%%%%%%
%%%%%%%%%%%%%%%%%%%%%%%%%%%%%%%%%%%%%%%%%%%%%%%%%%%%%%
%%%%%%%%%%%%%%%%%%%%%%%%%%%%%%%%%%%%%%%%%%%%%%%%%%%%%%
%%%%%%%%%%%%%%%%%%%%%%%%%%%%%%%%%%%%%%%%%%%%%%%%%%%%%%
%%%%%%%%%%%%%%%%%%%%%%%%%%%%%%%%%%%%%%%%%%%%%%%%%%%%%%
%%%%%%%%%%%%%%%%%%%%%%%%%%%%%%%%%%%%%%%%%%%%%%%%%%%%%%
%%%%%%%%%%%%%%%%%%%%%%%%%%%%%%%%%%%%%%%%%%%%%%%%%%%%%%
%%%%%%%%%%%%%%%%%%%%%%%%%%%%%%%%%%%%%%%%%%%%%%%%%%%%%%
%%%%%%%%%%%%%%%%%%%%%%%%%%%%%%%%%%%%%%%%%%%%%%%%%%%%%%
%%%%%%%%%%%%%%%%%%%%%%%%%%%%%%%%%%%%%%%%%%%%%%%%%%%%%%
%%%%%%%%%%%%%%%%%%%%%%%%%%%%%%%%%%%%%%%%%%%%%%%%%%%%%%
%%%%%%%%%%%%%%%%%%%%%%%%%%%%%%%%%%%%%%%%%%%%%%%%%%%%%%
%%%%%%%%%%%%%%%%%%%%%%%%%%%%%%%%%%%%%%%%%%%%%%%%%%%%%%











\COMMENT{
\paragraph{Triangular race.}
This code has a triangular race due to the first read of $Obj$, but before getting into that let us first discuss why the read of $Obj$ within the lock protected region does not lead to a triangular race.
Let us use $R_1$ and $R_2$ to denote the first and second read statements of $\alimemx {\tt Obj}$. 
Let $P$ be the program containing only {\tt DoubleCheckInit}.

Let $\alisequencex r$ be a TSO-run of the form 
\[
\alisequencex r = q_0\tau_1\cdot\ q\xrightarrow{\textnormal{\sc\small Rd},t:R_2}q'\ \cdot\tau_2q_n
\]
For $\alisequencex r$ to contain a triangular race, one condition requires that $t$ have done a write on some variable other than that is read by $R_2$, i.e. a local write statement to a location different from {\tt Obj} in $\tau_1$.
Such a write, the one that updates {\alimemx{l}} in order to acquire the lock, exists.
The second condition requires that another thread write concurrently to {\alimemx{\tt Obj}} as the first transition in $\tau_2$.
But clearly such a concurrent write cannot occur because all writes to {\alimemx{\tt Obj}} are protected by the lock and at the beginning of $\tau_2$, the thread $t$ holds the lock.
Thus, had it not been for the first read $R_1$, this procedure would have been triangular race-free and SC-like.

Now let us repeat the same consideration for $R_2$.
Let the TSO-run $\alisequencex r$ be in the form
\[
\alisequencex r = q_0\tau_1\cdot\ q\xrightarrow{\textnormal{\sc\small Wr},t:W_l}\cdot\ \tau_2\ \cdot q'\xrightarrow{}
\]

}

%%%%%%%%%%%%
%%%the following should go into the QED section where we describe how SC based analysis tools 
%%%can be used to decide the mover types of remote or local write actions.
%%%%%%%%%%%%%
\COMMENT{
We begin by describing a transformation which maps TSO programs to {\em equivalent} SC programs.
We should emphasize that our construction does not employ explicit arrays, but rather implicitly models the locally ordered buffered writes by simulating each thread with two threads.
The SC program enables us to apply reduction directly.
In particular, we show how mover types of certain statements can be used to conclude that a program cannot have any TSO distinguishing behavior.
Finally, we define an abstraction relation among TSO programs.
This abstraction forms the pillar of a reasoning methodology which enables one to start with a program with TSO distinguishing behavior and to end with a more abstract program which is SC-like.


\begin{figure*}
\begin{tabular}{p{.4\textwidth}p{.3\textwidth}p{.3\textwidth}}
\begin{alltt}\(ReadL(e,r,l\sb{s})\) \{\COMMENT{ \(\aliatomic\) \{   }                                 
 if (UCntL[\(e\)][tid]>UCntR[\(e\)][tid])       
  \{\(r\) := \(e\sp{loc}\);\} \{\(r\) := \(\alimemx{e}\);\}                      
 ReadVal[\(l\sp{loc}\sb{s}\)][tid] := \(r\);               
 \(l\sp{loc}\sb{s}\)[tid]++;
\}

\(ReadR(e,r,l\sb{s})\) \{
 \(\aliassume\) \(l\sp{rem}\sb{s}\)[tid] < \(l\sp{loc}\sb{s}\)[tid];
 r := ReadVal[\(l\sp{rem}\sb{s}\)][tid];
 \(l\sp{rem}\sb{s}\)[tid]++;
\}
\end{alltt} &

\begin{alltt}\(WriteL(e,r,l\sb{s})\) \{
 \(e\sp{loc}\) := \(e\);
 \(l\sp{loc}\sb{s}\)[tid]++;
 UCntL[\(e\)][tid]++;

\(WriteR(e,s,l\sb{s})\) \{
 \(\aliassume\) \(l\sp{rem}\sb{s}\)[tid] < \(l\sp{loc}\sb{s}\)[tid];
 \(\alimemx{e}\) := WriteVal[\(l\sp{loc}\sb{s}\)][tid];
 \(l\sp{rem}\sb{s}\)[tid]++;
 UCntR[\(e\)][tid]++;
\}\end{alltt} & 

\begin{alltt}
\(FenceL\) \{
 \(\aliassume \forall e.\) 
   UCntL[\(e\)][tid]
        ==
   UCntR[\(e\)][tid];
\}

\(FenceR\) \{
 \(\aliskip\);
\}\end{alltt}
\end{tabular}
\caption{The TSO to SC transformation macros.}
\label{fig:transformation-macros}
\end{figure*}

\subsection{Program transformation.}
\label{subsec:program-transformation}
Let $P=(\{M_1,\ldots,M_n\},\alilabel)$ be a labeled program.
For notational convenience, we let $l_s$ to denote the label $l$ of $s$, i.e. $\alilabelx s=l$.
The {\em split transformation} of $P$ is another program $\alisplitprogx P=(\{M^{loc}_1,M^{rem}_1,\ldots,M^{loc}_n,M^{rem}_n\},\alilabel')$, whose components are explained below.

Let $s$ be a statement in $M_i$.
The {\em local transformation} and the {\em remote transformation} of $s$, written as $\aliloctransx s$ and $\aliremtransx s$ respectively, are given below:
\begin{eqnarray*}
 \aliloctransx s  & \stackrel{def}{=} &
  \begin{cases}
   \aliatomicx {ReadL(e,r,l_s)} & , s=r:=\alimemx e\\
   \aliatomicx {WriteL(e,r,l_s)} & , s=\alimemx e:= r\\
   \aliatomicx {FenceL} & , s=\alifence\\
   s & , \textnormal{otherwise}
  \end{cases}\\
 \aliremtransx s & \stackrel{def}{=} &
  \begin{cases}
   \aliatomicx {ReadR(e,r,l_s)} & , s=r:=\alimemx e\\
   \aliatomicx {WriteR(e,r,l_s)} & , s=\alimemx e:= r\\
   \aliatomicx {FenceR} & , s=\alifence\\
   s & , \textnormal{otherwise}
  \end{cases}
\end{eqnarray*}
By convention, we let $\aliloctransx \varepsilon=\aliremtransx \varepsilon=\varepsilon$.

These transformations are used to implicitly represent the store buffer.
The local transformation of a read statement $r\mathtt{:=}\alimemx e$ checks whether the latest update to $\alimemx e$ by this thread is still in its buffer.
If so, the value to be read from the buffer is mimicked by reading the new auxiliary variable $e^{loc}$ as seen by this thread.
Otherwise, the value is read from the memory, i.e. the contents of $\alimemx e$.

Both of these transformations are extended to code blocks respecting the program order.
Formally, if $C=s;C'$ is a code block, $\aliloctransx C$ and $\aliremtransx C$ are given by $\aliloctransx s; \aliloctransx {C'}$ and $\aliremtransx s; \aliremtransx {C'}$, respectively.
This way we identify each method $M_i=\{C_i\}$ with two methods $M^{loc}_i=\{\aliloctransx {C_i}\}$ and $M^{rem}_i=\{\aliremtransx {C_i}\}$ by applying $\aliloctrans$ and $\aliremtrans$ to each statement in $M_i$.
}
\COMMENT{
We call an SC run of $\alisplitprogx P$ {\em well-formed} if 
\begin{itemize}
\item it admits a partitioning $\aliloctransx T$ and $\aliremtransx T$ of $T$ such that $t\in\aliloctransx T$ iff 
\item there is a bijection $\mu$ among the set of thread and method pairs $(t,m)$ such that if $\mu(t,m)=(t',m')$, then $t$ is executing $m$ , $t'$ is executing $m'$, $t\neq t'$ and there is a method $\hat{m}$ with $m=\aliloctransx \hat{m}$ and $m'=\aliremtransx \hat{m}$.
\end{itemize}
}

\COMMENT{
%Intuitively, a well-formed run will have both the local and remote copy of a method (never only one of them)
The transformation as defined is both sound and complete which is given as the main result of this section.


\begin{theorem}
Let $P$ be a labelled program.
The TSO runs of $P$ and the SC runs of $\alisplitprogx P$ are isomorphic up to the rearrangement of the initialization of the remote methods, $M^{rem}_i$.
\end{theorem}
\begin{proof}[Sketch]
Let $\alisequencex r$ be a TSO run of $P$.
Then a run of $\alisplitprogx P$ is constructed by replacing all {\sc\small Init} transitions done by $t$ for method $m$ with an {\sc\small Init} transition done by $t^{loc}$ for $m^{loc}$ immediately followed by another {\sc\small Init} transition done by $t^{rem}$ for $m^{rem}$.
All local write transitions {\sc\small WrB} by $t$ are replaced with {\sc\small Wr} transitions by $t^{loc}$.
Similarly all buffered write transitions {\sc\small WrM} by $t$ are replaced with {\sc\small Wr} transitions by $t^{rem}$.
Each fence transition due to statement $s$ of $m$ by $t$ is replaced with $\aliremtransx s$ of $m^{rem}$ by $t^{rem}$.
By an inductive argument, in such cases the predicate in the assume instruction of $\aliremtransx s$ evaluates to true.

The construction in the other direction first moves all initialization transitions {\sc\small Init} of remote copies, i.e. $m^{rem}$ by some $t^{rem}$ immediately after its associated {\sc\small Init} transition of $m^{loc}$.
The replacement of the initialization of 
\end{proof}
}

%We now present a mapping which will output another program $P'$ such that there exists a bijection between $\alirunsx {tso} P$ and $\alirunsx {sc} P$. 



\COMMENT{
Observe that according to Def.~\ref{def:movers}, in order to prove that a statement $s$ is a left-mover we only need to check whether $s$ moves to the left of those statements with which $s$ can simultaneously execute. 
Let us consider a particular instance where $s^{loc}$ is a local write action executed by thread $t$, $s$ is its matching remote write action, and all occurrences of $s^{loc}$ are preceded by a fence statement executed by $t$.
This means that between $s^{loc}$ and $s$ there could be no remote write actions done by $t$.
This in turn implies that if $s$ is left-mover, then the combined write action is atomic.
}


 


\COMMENT{

\begin{definition}[Movers]
Let $\mathbf{E}$ be a set of TSO-executions, and $\aliequivgeneric$ be an equivalence relation over $\mathbf{E}$.
Let $a$ be some TSO action that occurs in some TSO-execution in $\mathbf{E}$.
Then, $a$ is called a {\em left mover per $\aliequivgeneric$} if for any TSO-execution $\mathbf{e}\in\mathbf{E}$ in which $a$ occurs, there is another TSO-execution $\mathbf{e}'\in\mathbf{E}$ such that $\mathbf{e}\aliequivgeneric\mathbf{e}'$ and each occurrence of $a$ in $\mathbf{e}'$ is either immediately preceded by actions that precede $a$ in $\alipotsox {\mathbf{e}'}$ or the first action in $\mathbf{e}'$.

Similarly, $a$ is called a {\em right mover per $\aliequivgeneric$} if for any TSO-execution $\mathbf{e}\in\mathbf{E}$ in which $a$ occurs, there is another TSO-execution $\mathbf{e}'\in\mathbf{E}$ such that $\mathbf{e}\aliequivgeneric\mathbf{e}'$ and each occurrence of $a$ in $\mathbf{e}'$ is either immediately followed by actions that succeed $a$ in $\alipotsox {\mathbf{e}'}$ or the last action in $\mathbf{e}'$.
\end{definition}

The preceding definition is more general than the classical definition of reduction which fixes the interpretation of $\aliequivgeneric$: two executions are equivalent if they have identical end-states.
The reason for the added flexibility should become clear when we introduce abstraction for programs.
However, we will drop the mention of the equivalence relation whenever it is irrelevant to the discussion or clear from the context.

Typically, one relies on an inductive argument to show that a particular action is a mover.
Let $\mathbf{e}$ be a TSO-execution in $\mathbf{E}$.
We say that $\mathbf{e}[i]$ moves to the left of $\mathbf{e}[i-1]$ if there is a TSO-execution $\mathbf{e}'$ such that $\mathbf{e}\aliequivgeneric\mathbf{e}'$, $\mathbf{e}\langle 1,i-2\rangle\cdot\mathbf{e}[i]=\mathbf{e}'\langle 1,i-1\rangle$ and $|\mathbf{e}|\geq|\mathbf{e}'|$.
Intuitively, we stay in the same equivalence class by moving $\mathbf{e}[i]$ one step to the left (towards the beginning of the sequence). 
Then, if one can show that a particular action $a$ moves to the left of all actions $b$ that can immediately precede $a$ except for those $c$ such that $c\alipotsox {\mathbf{e}} a$, then $a$ will be shown to be a left-mover in that equivalence class.
%These arguments will be readily used to argue the mover types of statements of a program.  

\begin{lemma}
Let $\mathbf{e}$ be a TSO-execution and let $A_r$ denote the set of actions $\{\mathbf{e}[i] \mid \mathbf{e}[i]\in\tsoalph_{\remwrite,-,-}\}$ and $A_l$ denote the set of actions $\{\mathbf{e}[i] \mid \mathbf{e}[i]\in\tsoalph_{\locwrite,-,-}\}$.
If each action in $A_r$ is a left-mover or each action in $A_l$ is a right-mover in $[\mathbf{e}]_{\tsoequiv}$, then $\mathbf{e}$ is SC-like.
\end{lemma}

Thus, it is sufficient to show that all remote write actions are left-movers or all local write actions are right-movers to conclude that a TSO-execution is SC-like.
It is possible to weaken the condition further as follows.

\begin{corollary}

\end{corollary}


}



\COMMENT{
Consider the code given in Fig.~\ref{fig:store-buffer}, which is derived from a classic example illustrating the main behavioral difference between TSO and SC architectures.
\begin{figure}[h]
\begin{tabular}{p{.14\textwidth}p{.14\textwidth}p{.14\textwidth}}
\begin{alltt}m1()
\{
 \(A\):=1;
 \(\alireg\)[1]:=\(B\);
\}\end{alltt} 
&
\begin{alltt}m2()
\{
 \(B\):=2;
\}\end{alltt}
&
\begin{alltt}m3()
\{
 \(\alireg\)[1]:=\(B\);
 \(\alireg\)[2]:=\(A\);
\}\end{alltt}\\
\multicolumn{3}{l}%
{$\locwritex t A 1 \cdot \alireadx t B 0 \cdot \locwritex u B 2 \cdot \remwritex u B 2$}\\
\multicolumn{3}{c}%
{$ \cdot\ \alireadx v B 2 \cdot \alireadx v A 0 \cdot \remwritex t A 1$}
\end{tabular}
\caption{Store buffering with non SC-like memory trace.}
\label{fig:store-buffer}
\end{figure}
The memory trace given in Fig.~\ref{fig:buffer-store} is TSO compliant.
It belongs to a run in which thread $t$ runs {\tt m1}, $u$ runs {\tt m2} and $v$ runs {\tt m3} exactly once.




In the remainder of this section, we work through an example program which contains a triangular race and hence is not initially amenable to an SC analysis.
We show how abstracting the program leads to a program which is SC-like.
}


\COMMENT{
Consider a sequence of memory accesses done by two threads, $t$ and $u$, given as:
\[
{\aliwritex t y 1}\ \cdot\ {\alireadx t x 0}\ \cdot\ {\aliwritex u x 2}
\]
%Informally, this represents a sequence with an arbitrary sequence of memory operations followed by a write to $y$ by thread $t$ followed by a (possibly empty) sequence of reads done by $t$ followed by a read to $x$ followed by a write $x$ by a different thread $u$.
This represents a sequence of a write to $y$ of value 1 by thread $t$, followed by a read of $x$ returning 0 again by $t$, followed by a write to $x$ of 2 by thread $u$.

To see why this is a {\em bad} (non-SC) sequence for TSO, we will refer to an alternative characterization of memory accesses in TSO.
We represent read accesses as usual, but change the way write accesses are represented.
We split each write access $W$ into two distinct accesses, {\locwrite} and {\remwrite}.
Each sequence over simple memory accesses as given above will be identified with a set of sequences over the new alphabet such that i) each {\aliwrite} is replaced with {\locwrite}, ii) each {\locwrite} has a matching {\remwrite} such that the thread local order among {\locwrite}'s is preserved by their associated {\remwrite}'s.
For instance, the following sequence is one possible TSO-expansion of the SC-sequence above.
\[
{\locwritex t y 1}\ \cdot\ {\alireadx t x 0}\ \cdot\ {\locwritex u x 2}\ \cdot\ {\remwritex u x 2}\ \cdot\ {\remwritex t y 1}
\]
Observe that the ordering between the remote updates to $y$ and $x$ is the opposite of the ordering between their corresponding local updates. 
Intuitively this means that the order of updates to $x$ and $y$ as seen by $t$ ($y$ precedes $x$) and the order seen by a different thread $v$ ($x$ precedes $y$) will be different, which is impossible under SC.

We show that a similar characterization can be obtained using reduction arguments.
An execution can be generated under SC if it is possible to {\em move} all local and their matching remote updates next to each other without changing the end state of the execution.
For the example above, the question is whether it is possible to move either $\remwritex t y \dontcare$ to the left of every possible concurrent action or $\locwritex t y \dontcare$ to the right of every possible concurrent action.
The trace above itself is not problematic because both remote writes are left-movers. 
In other words, the above trace is equivalent to the following:
\[
{\locwritex t y 1}\,\cdot\,{\remwritex t y 1}\,\cdot\,{\alireadx t x 0}\,\cdot\,{\locwritex u x 2}\,\cdot\,{\remwritex u x 2}
\]
However, in the presence of other writes such a transformation can become impossible.
Consider for instance
\begin{eqnarray*}
&{\locwritex t y 1}\,\cdot\,{\alireadx t y 1}\,\cdot\,{\alireadx t x 0}\,\cdot\,{\locwritex u x 2}\,\cdot\,{\alireadx u x 2}\,\cdot\,{\alireadx u y 0}\\
&\cdot\ {\remwritex u x 2}\,\cdot\,{\remwritex t y 1}
\end{eqnarray*}
It is possible to move the local and remote writes to $x$ done by thread $u$ without changing the values read.
However, it is impossible to make the local and remote writes to $y$ done by thread $t$ without changing one of the reads: if $\remwritex t y 1$ is to the left of $\locwritex u x 2$, the read of $y$ by thread $u$ will return 1 instead of 0; if the $\locwritex t y 1$ is to the right of $\locwritex u x 2$, the read of $x$ by thread $t$ will return 2 instead of 0.

\begin{figure}[t]
\begin{alltt}
Recv()          Send(d)
 while (R);      while (!R);
 d := Data;      Data := d;
 R := \(\alitrue\);       R := \(\alifalse\);
 return d;\end{alltt}
\caption{A binary synchronous rendez-vous with two threads.}
\label{fig:rendez-vous}
\end{figure}
Consider a simple template for a binary synchronous rendez-vous message passing as given in Fig.~\ref{fig:rendez-vous}.
There are two methods, {\tt Recv} and {\tt Send}.
The receiver method {\tt Recv} spins for the flag {\tt R} until it is set to false.
Once that happens, it reads the value stored in {\tt Data}, resets {\tt R} to true and returns the value it has read.
The sender method {\tt Send} operates dually: it spins on {\tt R} until it is set to true, updates the value in {\tt Data} and resets {\tt R} to false.

Recall that a write can be treated as atomic if for any execution in which both its local and remote updates occur, one can find an equivalent execution in which they occur consecutively. 
For instance, if we want to show that the write to {\tt R} done by {\tt Recv} is atomic, we have to show that if the following is an execution
\[
A\,\cdot\,{\locwritex t R {\alitrue}}\,\cdot\,B\,\cdot\,{\remwritex t R {\alitrue}}
\]
there is a partitioning of $B$ into $B_1$ and $B_2$ such that
\[
A\,\cdot\,B_1\,\cdot\,{\locwritex t R {\alitrue}}\,\cdot\,{\remwritex t R {\alitrue}}\,\cdot\,B_2
\]
is also an execution.
One can construct such an execution because i) one can always move a local update to the right of other thread's actions, ii) there cannot be any remote updates to $R$ in the sequence of accesses represented by $B$, iii) thus it is safe to move any reads of $R$ done by a subsequent call to {\tt Send} all of which will return {\alitrue}.
Thus setting $B_1=B$ and $B_2=\varepsilon$ gives the desired execution.
Similar arguments are used to prove that all the write accesses in Fig.~\ref{fig:rendez-vous} are atomic.

\paragraph{Abstracting memory accesses.}

\paragraph{Transforming TSO into SC.}

\paragraph{Local analysis for reduction.}
Once the program is transformed, one can appeal to the reduction theorems for TSO.
}
