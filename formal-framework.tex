% !TEX root = main-tsoreduction.tex
\section{Formal Framework}
\label{sec:formal-framework}
In this section, we describe the notation used in the rest of the paper, formalize TSO and SC semantics, and present a simple programming language.

\begin{figure}
\begin{tabular}{lcll}
action ($a$,$b$) & ::= & \aliwritex t x v & (direct write: shared memory\\
& & & update of location $x$\\ 
& & & with value $v$ by thread $t$)\\
& $|$ & \alireadx t x v & (a read of value $v$ from $x$ by $t$)\\
& $|$ & \locwritex t x v & (local write: insertion of the\\
& & & write of $v$ into $x$ by $t$)\\
& $|$ & \remwritex t x v & (remote write: shared memory\\
& & & update of $x$ with $v$ by $t$)\\
& $|$ & \alibarrierx t & (memory barrier by $t$)
%& $|$ & \alilockx t & (the beginning of a \\
%& $|$ & \aliunlockx t
\end{tabular}
\caption{The alphabet $\tsoalph$ of memory actions in TSO.}
\label{fig:grammar-tso}
\end{figure}
\paragraph{Notation.} 
Let $\mathbf{a}$ be a sequence over $A$.
We will use indexed notation to refer to the elements in $\mathbf{a}$: $\mathbf{a}[i]$ is the $i^{th}$ action in $\mathbf{a}$.
Similarly, we let $\mathbf{a}\langle i,j\rangle$ denote the segment of $\mathbf{a}$ from $\mathbf{a}[i]$ to $\mathbf{a}[j]$ with both ends inclusive.
The length of $\mathbf{a}$ is written as $\alilenx {\mathbf{a}}$ and gives the number of symbols in $\mathbf{a}$.
Let $\pi$ be a permutation over $[1,\alilenx {\mathbf{a}}]$.
We use $\pi(\mathbf{a})$ to denote the sequence $\mathbf{a}[\pi(1)]\ldots \mathbf{a}[\pi(\alilenx {\mathbf{a}})]$.
Two sequences $\mathbf{a}$ and $\mathbf{b}$ are {\em permutationally equivalent}, written $\mathbf{a} \aliperm \mathbf{b}$, if there is a permutation $\pi$ such that $\mathbf{a}=\pi(\mathbf{b})$.

As far as the memory accesses are concerned, the symbols we are going to use and their informal meanings are given in Fig.~\ref{fig:grammar-tso}.
For simplicity, we omit locked operations which can be modeled by the given actions.
We will use the notation $\tsoalph_{op,thr,loc}$ to denote the subset of $\tsoalph$ which contains all actions of type $O\subseteq\{\aliwrite, \aliread, \locwrite, \remwrite, \alibarrier\}$, by thread $thr$, and into location $loc$.
When $O$ is a singleton, we will ignore the braces.
Omitting parameters denotes existential quantification; e.g. $\tsoalph_{-,t,-}$ is the set of all actions done by thread $t$.
The projection of a sequence $\mathbf{e}$ over $\tsoalph$ into $\tsoalph_{O,t,l}$, written as $\mathbf{e}\downarrow_{O,t,l}$, is the subsequence obtained by keeping only those actions in $\tsoalph_{O,t,l}$.
For instance, $\aliprojx {\mathbf{e}} {\aliread,t,-}$ is the subsequence of $\mathbf{e}$ consisting of all the read actions done by $t$.

\subsection{TSO and SC Executions}
\label{subsec:tso-sc-semantics}
In this section, we define correct TSO and SC executions, and define what we mean by an equivalent SC-execution to a given TSO-execution whenever the former exists.

\paragraph{TSO-executions.}
{\em TSO actions}, subset of $\tsoalph$, exclude only the first kind, $\aliwritex t x v$.
A remote write $\remwritex t x v$ {\em matches} the local write $\locwritex u y w$ iff $t=u$, $x=y$ and $v=w$.
Let $\mathbf{e}$ be a sequence of TSO actions.
It is {\em matched} if for all $t$, $\aliprojx {\mathbf{e}} {\remwrite,t,-}[i]$ matches $\aliprojx {\mathbf{e}} {\locwrite,t,-}[i]$.
A local write $\mathbf{e}[i]$ is {\em buffered at $j$} if it is matched by a remote write at position $l>j$.
We will call a matched $\mathbf{e}$ {\em complete} if there are no buffered writes at $|\mathbf{e}|$.
Unless stated otherwise, sequences of TSO actions are assumed to be complete.

A TSO-execution is a sequence $\mathbf{e}$ of TSO actions subject to well-formedness constraints:
\begin{itemize}
\item Local write actions occur before their matching remote write actions.
Formally, if $\mathbf{e}[i]=\locwritex t x v$ and $\mathbf{e}[j]=\remwritex t x v$ are respectively the $k^{th}$ local and remote write actions by thread $t$, then $i<j$.
\item The read values are consistent.
Formally, if $\mathbf{e}[j]=\alireadx t x v$, then either i) $\locwritex t x v$ is the most recent buffered write at $j$ by $t$ to $x$, or ii) no buffered write to $x$ by $t$ at $j$ exists and $\remwritex u x v$ is the most recent remote write, or iii) neither condition applies and $v$ is the initial value.
\item The barrier operations can only happen when the buffer is empty.
Formally, if $\mathbf{e}[j]=\alibarrierx t$ and if $\locwritex u x v$ is buffered at $j$, then $t\neq u$.
\end{itemize}

\begin{definition}[Synchronous, Buffer-free]
A local write $\mathbf{e}[i]$ is {\em synchronous} if it is not buffered at $i+1$.
A complete TSO-execution $\mathbf{e}$ is called {\em buffer-free} if all of its local writes are synchronous.
\end{definition}
Synchronous local writes form the crux of the reduction we present in this paper.
The following observations state that a thread $t$ can perform a synchronous local write only when its buffer is empty.
\begin{lemma}\label{lem:synchronous-write}
Let the local write $\mathbf{e}[i]=\locwritex t x v$ of a TSO-execution $\mathbf{e}$ be synchronous.
Then $\mathbf{e}\langle 1,i+1\rangle\cdot\ \alibarrierx t\cdot\ \mathbf{e}\langle i+2,\alilenx{\mathbf{e}}\rangle$ is also a (well-formed) TSO-execution.
\end{lemma} 
\begin{proof}
This follows from the definition of well-formedness which requires local and remote writes of the same thread to be in the same order because that implies that no local write can be buffered by $t$ at $i$.
\end{proof}

Each TSO-execution $\mathbf{e}$ induces a partial order $\alipotsox {\mathbf{e}}$ over $\{\mathbf{e}[i] \mid i\in[1,|\mathbf{e}|]\}$ such that $a\alipotsox {\mathbf{e}} b$ if $a$ occurs before $b$ in $\mathbf{e}$, $a,b\in\tsoalph_{-,t,-}$ and one of the following holds:
\begin{itemize}
\item both $a$ and $b$ are remote write actions.
\item neither $a$ nor $b$ is a remote write action. 
\item $a$ is the $i^{th}$ local write by $t$ and $b$ is the $i^{th}$ remote write by $t$.
\end{itemize}

\paragraph{SC-executions.}
{\em SC actions}, subset of $\tsoalph$, include only the first two kinds of actions; i.e., $\{\aliwritex t x v\}\cup\{\alireadx t x v\}$.
%The actions of the form $\aliwritex t x v$ are (direct) {\em write} actions.
SC-executions have to satisfy: if $\mathbf{s}[i]=\alireadx t x v$, then either i) there is $j<i$ with $\mathbf{s}[j]=\aliwritex u x v$ and there are no write actions to $x$ in $\mathbf{s}\langle j+1,i\rangle$, or ii) there is no write action in $\mathbf{s}\langle 1,i\rangle$ and $v$ is $\bot$.


%Similar to TSO-executions, each SC-execution $\mathbf{s}$ induces a partial order $\aliposcx {\mathbf{s}}$ such that $a\aliposcx {\mathbf{s}} b$ if $a$ occurs before $b$ in $\mathbf{s}$ and both belong to the same thread.

\newcommand{\SCofTSO}{\ensuremath{\mathsf{SC}}}
\newcommand{\SCofTSOx}[1]{\ensuremath{\SCofTSO(#1)}}
\newcommand{\SCofTSOstrict}{\ensuremath{\SCofTSO_S}}
\newcommand{\SCofTSOstrictx}[1]{\ensuremath{\SCofTSOstrict(#1)}}
\newcommand{\aliconvert}{\ensuremath{\ulcorner \urcorner}}
\newcommand{\aliconvertx}[1]{\ensuremath{\ulcorner #1 \urcorner}}
\newcommand{\tsoequiv}{\ensuremath{\approx}}
\newcommand{\tsoequivstrict}{\ensuremath{\tsoequiv_S}}
\newcommand{\equivclassx}[2]{\ensuremath{[#1]_{#2}}}
\newcommand{\tighter}{\ensuremath{\sqsubseteq}}
\newcommand{\tighterstrict}{\ensuremath{\tighter_S}}
\newcommand{\tightclosure}{\ensuremath{\mathsf{T}}}
\newcommand{\tightclosurex}[1]{\ensuremath{\tightclosure(#1)}}
\newcommand{\tightclosurestrict}{\ensuremath{\tightclosure_S}}
\newcommand{\tightclosurestrictx}[1]{\ensuremath{\tightclosurestrict(#1)}}

\paragraph{Correspondence between TSO and SC.}
We define a mapping $\aliconvert$ from the actions of TSO to those of SC as follows.
\[
\aliconvertx a \stackrel{def}{=} 
 \begin{cases}
  \aliwritex t x v & \text{, if } a=\locwritex t x v\\
  \alireadx t x v & \text{, if } a=\alireadx t x v\\
  \varepsilon & \text{, if } a=\remwritex t x v\\
  \varepsilon & \text{, if } a=\alibarrierx t
 \end{cases}
\]
For a TSO-execution $\mathbf{e}$, let $\aliconvertx {\mathbf{e}}$ denote the sequence obtained by the point-wise extension of $\aliconvert$.

Let $\mathbf{e}$ be a complete TSO-execution.
We define $\SCofTSOx {\mathbf{e}}$ to be the set of all SC-executions $\mathbf{s}$ such that $\aliconvertx {\aliprojx {\mathbf{e}} {-,t,-}}=\aliprojx {\mathbf{s}} {-,t,-}$ and $\aliprojx {\mathbf{e}} {\remwrite,-,-}[i] = \remwritex t x v$ iff $\aliprojx {\mathbf{s}} {\aliwrite,-,-}[i] = \aliwritex t x v$.
Informally, $\mathbf{s}$ belongs to $\SCofTSOx {\mathbf{e}}$ if it is an SC-execution, respects the thread local ordering of read and local write actions, and preserves the order among the remote write actions in $\mathbf{e}$.
We have the following result following immediately from definitions.

%We define $\SCofTSOx {\mathbf{e}}$ to be the set of all SC-executions $\mathbf{s}$ such that $\aliconvertx {\aliprojx {\mathbf{e}} {-,t,-}}=\aliprojx {\mathbf{s}} {-,t,-}$.
%Informally, $\mathbf{s}$ belongs to $\SCofTSOx {\mathbf{e}}$ if it is an SC-execution and respects the thread local ordering of read and local write actions.

%A particular subset of $\SCofTSOx {\mathbf{e}}$ is of interest to us.
%Let $\SCofTSOstrictx {\mathbf{e}}\subseteq \SCofTSOx {\mathbf{e}}$ be the set of all $\mathbf{s}$ such that $\aliprojx {\mathbf{e}} {\remwrite,-,-} [i] = \remwritex t x v$ iff $\aliprojx {\mathbf{s}} {\aliwrite,-,-}[i] = \aliwritex t x v$.
%Informally, $\mathbf{s}$ will be in $\SCofTSOstrictx {\mathbf{e}}$ if additionally the order among the write actions in $\mathbf{s}$ preserves the order among the remote write actions in $\mathbf{e}$.

\begin{proposition}\label{prop:complete-tso}
For any buffer-free TSO-execution $\mathbf{e}$, $\aliconvertx {\mathbf{e}} \in \SCofTSOx {\mathbf{e}}$.
\end{proposition}
The proof follows from the fact that in a buffer-free TSO-execution, all reads are from the memory (only condition (ii) of TSO well-formedness for consistent read values applies).
The desired SC-execution then can be obtained by simply replacing each (adjacent) pair of local and remote write actions by their image under $\aliconvert$.


\paragraph{Equivalence ($\tsoequiv$) and partial order ($\tighter$) over TSO-executions.}
Two complete TSO-executions $\mathbf{e}$ and $\mathbf{f}$ are {\em equivalent}, written $\mathbf{e}\tsoequiv\mathbf{f}$, if they are permutationally equivalent and they both induce the same partial order. 
Formally, $\mathbf{e}\tsoequiv\mathbf{f}$ if $\mathbf{e}\aliperm\mathbf{f}$ and $\alipotsox {\mathbf{e}}=\alipotsox {\mathbf{f}}$.
Whenever no confusion is likely to arise, we will use $\pi$ to denote one of the permutations between two equivalent TSO-executions establishing their equivalence.

Synchronous writes induce a partial order $\tighter$ on equivalent TSO-execution.
A TSO-execution $\mathbf{f}$ is {\em tighter} than $\mathbf{e}$, written $\mathbf{f}\tighter\mathbf{e}$ if $\mathbf{f}\tsoequiv\mathbf{e}$ and whenever $\mathbf{e}[i]$ is a synchronous local write, then so is $\mathbf{f}[\pi(i)]$.
In other words, $\mathbf{f}$ is tighter than an equivalent $\mathbf{e}$ if all synchronous writes of the latter are also synchronous in the former. 
%The execution $\mathbf{e}$ is strictly tighter than $\mathbf{f}$, written $\mathbf{e}\tighterstrict\mathbf{f}$, if $\mathbf{e}$ is tighter than $\mathbf{f}$ and they are strictly equivalent.
Each TSO-execution thus induces a set of tighter executions.
Let $\tightclosurex{\mathbf{e}}=\{\mathbf{f} \mid \mathbf{f}\tighter\mathbf{e}\}$; that is, the set of all TSO-executions tighter than $\mathbf{e}$.
%Let $\tightclosurestrictx {\mathbf{e}}$ denote the subset of $\tightclosurex{\mathbf{e}}$ consisting of only strictly equivalent executions.

\begin{definition}[SC-like]
Let $\mathbf{e}$ be a TSO-execution.
%It is called {\em observationally SC-like} if $\tightclosurex {\mathbf{e}}$ contains an atomic execution.
It is called {\em SC-like} if $\tightclosurex {\mathbf{e}}$ contains a buffer-free execution.
Otherwise, it is called {\em TSO-specific}.
\end{definition}
%Since for any TSO-execution $\mathbf{e}$ we have $\tightclosurex {\mathbf{e}}\subseteq \tightclosurex {\mathbf{e}}$, SC-like is a stronger property than observationally SC-like.
The terms are not arbitrarily named as the following proposition shows.

\begin{proposition}
A TSO-execution $\mathbf{e}$ is SC-like iff $\tightclosurex {\mathbf{e}} \cap \SCofTSOx {\mathbf{e}} \neq \emptyset$.
%It is SC-like iff $\tightclosurestrictx {\mathbf{e}} \cap \SCofTSOstrictx {\mathbf{e}} \neq \emptyset$.
\end{proposition}
Intuitively, permutationally equivalent TSO-executions have identical observations about the state of the memory; i.e. the values returned by the reads are identical.
A tighter execution is one which has at most as many buffered local writes as those of the less tight one. 
At the limit, all local writes are synchronous (no buffered writes), which makes the execution buffer-free.
Thus, if an execution is equivalent to a buffer-free execution, then its behavior is identical to that of an SC-execution which is obtained by replacing all synchronous write actions with direct write actions.

\newcommand{\independent}{\ensuremath{\nleftarrow}}
\newcommand{\independentstrict}{\ensuremath{\independent_S}}
\newcommand{\rulelocal}{(\ensuremath{\mathbf{Loc}})}
%\newcommand{\ruleread}{(\ensuremath{\mathbf{Read}})}
\newcommand{\rulereml}{(\ensuremath{\mathbf{RemL}})}
\newcommand{\ruleremr}{(\ensuremath{\mathbf{RemR}})}

\newcommand{\rulethr}{(\ensuremath{\mathbf{Thr}})}
\newcommand{\rulemat}{(\ensuremath{\mathbf{Mat}})}
\newcommand{\ruleloc}{(\ensuremath{\mathbf{Loc}})}
\newcommand{\rulerem}{(\ensuremath{\mathbf{Rem}})}
\newcommand{\ruleremstrict}{(\ensuremath{\mathbf{RemS}})}
\newcommand{\aliconflict}{\ensuremath{\mathsf{Conf}}}
\newcommand{\aliconflictstrict}{\ensuremath{\aliconflict_S}}

\newcommand{\shuffleset}{\ensuremath{\mathsf{Shuff}}}
\newcommand{\shufflesetx}[1]{\ensuremath{\shuffleset(#1)}}



%%%%%%%%%%%%%%%%%%%%%%%%%%%%%%%%%%%%%%%%%%%%%%%%%%%%%%%%%
%%%%%%%%%%%%%%%%%%%%%%%%%%%%%%%%%%%%%%%%%%%%%%%%%%%%%%%%%
%%%%%%%%%%%%%%%%%%%%%%%%%%%%%%%%%%%%%%%%%%%%%%%%%%%%%%%%%
%%%% The following will probably used in the related %%%%
%%%% work, when comparing ours with Bouajjani's. %%%%%%%%
%%%%%%%%%%%%%%%%%%%%%%%%%%%%%%%%%%%%%%%%%%%%%%%%%%%%%%%%%
%%%%%%%%%%%%%%%%%%%%%%%%%%%%%%%%%%%%%%%%%%%%%%%%%%%%%%%%%
%%%%%%%%%%%%%%%%%%%%%%%%%%%%%%%%%%%%%%%%%%%%%%%%%%%%%%%%%
\COMMENT{

%\paragraph{Conflict Relations.}
%Two TSO actions $a$ and $b$ {\em conflict}, written $\aliconflict(a,b)$, if one of the following holds:
%\begin{description}
%\item[\rulerem] $a,b\in\tsoalph_{\remwrite,t,-}$.
%\item[\rulethr] $a,b\in\tsoalph_{\{\locwrite,\aliread,\alibarrier\},t,-}$.
%\item[\ruleloc] $a,b\in\tsoalph_{-,-,x}$ and $\{a,b\}\nsubseteq\tsoalph_{\aliread,-,-}$.
%\end{description}
%They {\em strictly} conflict, $\aliconflictstrict(a,b)$, if {\rulerem} is replaced with
%\begin{description}
%\item[\ruleremstrict] $a,b\in\tsoalph_{\remwrite,-,-}$.
%\end{description}
%
%Let $\mathbf{e}$ be a TSO-execution of length $k$.
%A sequence $\mathbf{f}$ is a {\em shuffling} of $\mathbf{e}$ if $\mathbf{e}\aliperm\mathbf{f}$ and there is some $i$ such that for all $j\notin\{i,i+1\}$, $\mathbf{e}[j]=\mathbf{f}[j]$ and $\mathbf{e}[i]=\mathbf{f}[i+1]$.
%The position $i$ is called the {\em pivot} (of shuffling).
%A shuffling of $\mathbf{e}$ with pivot $i$ is {\em valid} if $\pi(\mathbf{e})$ is a TSO-execution and $\mathbf{e}[i]$ does not conflict with $\mathbf{e}[i+1]$.
%Let $\shufflesetx {\mathbf{e}}$ denote the closure of the valid shufflings on $\mathbf{e}$.
%Formally, $\shufflesetx {\mathbf{e}}$ is defined as the set satisfying
%\begin{itemize}
%\item $\mathbf{e}\in\shufflesetx {\mathbf{e}}$,
%\item $\mathbf{f}\in\shufflesetx {\mathbf{e}}$ implies there is $\mathbf{g}\in\shufflesetx {\mathbf{e}}$ and $\mathbf{f}$ is a valid shuffling of $\mathbf{g}$.
%\end{itemize}
%
%Observe that there are two further constraints implicitly enforced by requiring a valid shuffling with pivot $i$. 
%First, if $\mathbf{e}[i+1]$ is a barrier action by thread $t$ and $\mathbf{e}[i]$ is a remote write action by the same thread, then they cannot be reordered as barrier cannot occur when there is a buffered write. 
%Second, if $\mathbf{e}[i]$ is a local write action by $t$ and $\mathbf{e}[i+1]$ is the matching remote write action, reordering them is not allowed since the resulting sequence will not be a TSO-execution.
%
%The shuffling closure of $\mathbf{e}$ gives an operational under-approximation of TSO-execution equivalence $\tsoequiv$ as the following result indicates.
%
%\begin{proposition}
%For complete TSO-execution $\mathbf{e}$, $\shufflesetx {\mathbf{e}} \subseteq \equivclassx {\mathbf{e}} {\tsoequiv}$.
%\end{proposition}

}






\newcommand{\aliprog}{\ensuremath{\mathtt{prog}}}
\newcommand{\alimem}{\ensuremath{\mathtt{mem}}}
\newcommand{\alimemx}[1]{\ensuremath{\alimem[#1]}}
\newcommand{\alifence}{\ensuremath{\mathtt{fence}}}
\newcommand{\aliassume}{\ensuremath{\mathtt{assume}}\xspace}
\newcommand{\aliassumex}[1]{\ensuremath{\mathtt{assume}\ #1}}
\newcommand{\aliassert}{\ensuremath{\mathtt{assert}}\xspace}
\newcommand{\aliassertx}[1]{\ensuremath{\mathtt{assert}\ #1}}
\newcommand{\alireg}{\ensuremath{\mathtt{r}}}
\newcommand{\aliregx}[1]{\ensuremath{\alireg[#1]}}
\newcommand{\aliif}{\ensuremath{\mathtt{if}}}
\newcommand{\alithen}{\ensuremath{\mathtt{then}}}
\newcommand{\alielse}{\ensuremath{\mathtt{else}}}
\newcommand{\aliifelsex}[3]{\ensuremath{\aliif\ #1\ \alithen\ \{#2\}\ \alielse\ \{#3\}}}
\newcommand{\aliwhile}{\ensuremath{\mathtt{while}}}
\newcommand{\aliwhilex}[2]{\ensuremath{\aliwhile(#1)\ \{#2\}}}
\newcommand{\aliatomic}{\ensuremath{\mathtt{atomic}}}
\newcommand{\aliatomicx}[1]{\ensuremath{\aliatomic\{#1\}}}
\newcommand{\aliskip}{\ensuremath{\mathtt{skip}}}
\newcommand{\alitid}{\ensuremath{\mathtt{tid}}}




\begin{figure}
\begin{tabular}{lcl}
%$P$ & ::= & $M,P \mid \varepsilon$\\
$M$ ($m$,$m',\ldots$) & ::= & $N()\ \{ C \}$\\
$C$ ($c$,$c',\ldots$) & ::= & $S$ {\tt ;} $C \mid \varepsilon$\\
$S$ ($s$,$s',\ldots$) & ::= & $R := \alimemx {E} \mid \alimemx {E} := R \mid$\\
& &  $R := E \mid \alifence \mid$\\
& & $\aliifelsex {E} {C} {C}\mid$\\ 
& & $\aliwhilex E C\mid \aliatomicx {C}$\\
& & $\aliassume\ E \mid \aliassert\ E \mid \aliskip$\\
$E$ ($e$,$e',\ldots$) & ::= & $R \mid I \mid N() \mid \alitid \mid$\\
& & $E+E \mid E-E \mid \ldots$\\
& & $E\wedge E \mid E\vee E \mid \neg E \mid \ldots $\\
$R$ ($r$,$r',\ldots$) & ::= & $\aliregx I$\\
$N$ & ::= & $\langle String\rangle$\\
$I$ & ::= & $\langle Integer\rangle$
%$T$ & ::= & $\langle Bool\rangle \mid \langle Int\rangle$\\
%$L$ & ::= & $\langle Label\rangle$\\
\end{tabular}
\label{fig:program-grammar}
\caption{A simple programming language.}
\end{figure}

\newcommand{\aliprogstmt}{\ensuremath{\mathsf{Stmt}}}
\newcommand{\aliprogstmtx}[1]{\ensuremath{\aliprogstmt(#1)}}
\newcommand{\alilabel}{\ensuremath{\mathsf{Lab}}}
\newcommand{\alilabelx}[1]{\ensuremath{\alilabel(#1)}}

\subsection{Programming Language}
\label{subsec:programming-language}
In this subsection, we formalize the programming language we are going to use in the rest of the paper.

\paragraph{Syntax.}
We use a simple programming language, whose grammar for method definitions is given in Fig.~\ref{fig:program-grammar}.
A method $M$ consists of a {\em name} ($N$) and a {\em code} ($C$).
A name is simply a string.
A code is a, possibly empty, sequence of statements ($S$) sequentially composed ({\tt ;}).
Statements can be used to read from memory ($r := \alimemx e$), update the contents of a memory location ($\alimemx e := e'$), empty the store buffer (\alifence), assign a value to a register ($r := e$).
The model control flow, we have the usual branching ($\aliifelsex e {e_1} {e_2}$) and looping ($\aliwhilex e$) statements.
Additionally, we will also use an explicit blocking statement ($\aliassume\ e$), which blocks as long as $e$ evaluates to $\alifalse$.
Finally, the statement ($\aliassert\ e$) is used to claim that $e$ should hold whenever this statement can execute.
The nonterminal $R$ refers to {\em registers} and are used to model thread local address space.
The nonterminal $E$ refers to well-formed {\em expressions} that can be written using registers, mathematical and logical operators, and the return values of procedures.
The shared data space is mapped by $\alimemx e$, where $e$ is an expression which evaluates to $\mathbb{N}$.
As a syntactic sugar, we will use words in italic font with first letters capitalized to denote a location in shared memory, e.g. $Obj$ will stand for some $\alimemx i$.

A {\em program} is identified with a set of methods.
We assume that each statement of a program is uniquely identified, captured by a labelling function $\alilabel: \aliprogstmtx P \rightarrow L$, where $\aliprogstmtx P$ denotes the set of statements used in program $P$ and $L$ is the (universal) {\em label} set.

\paragraph{Operational Semantics.}
% !TEX root = main-tsoreduction.tex

\begin{figure*}[th]
\begin{mathpar}
\fbox{$(\alicontrol,\alivaluation,\alibuffer,\aliexecmode)\xrightarrow{Rl,t:s}(\alicontrol',\alivaluation',\alibuffer',\aliexecmode')$}\\

\inferrule*
{Rl=\textnormal{\sc\small Init} \\ M_i\in P \\ M_i= n()\ \{c\} \\\\ 
\alicontrolx t =\varepsilon \\ s=a \\ \alienabledx t}
{\alivaluation'=\alivaluation[(t,\alireg_{in})\mapsto a] \\ \alicontrol'=\alicontrol[t\mapsto \alifirstx C]}

\inferrule*
{Rl=\textnormal{\sc\small Rd} \\ \alicontrolx t = \alisequencex l \cdot l' \\ \alilabelx s = l' \\\\
s=r:=\alimemx e \\ \alienabledx t}
{\alivaluation'=\alivaluation[(t,r)\mapsto \alivaluationx {\alivaluation{\alievalx e}t}] \\ \alicontrol'=\alicontrol[t\mapsto \alisequencex l \cdot \alisuccx{l'}]}

\inferrule*
{Rl=\textnormal{\sc\small RdM} \\ \alicontrolx t = \alisequencex l \cdot l' \\ \alilabelx s = l' \\\\ 
\alivaluation{\alievalx e}t \notin \alibufferx t \\ s=r := \alimemx e \\ \alienabledx t}
{\alivaluation'=\alivaluation[(t,r)\mapsto \alivaluationx {\alivaluation{\alievalx e}t}] \\ \alicontrol'=\alicontrol[t\mapsto \alisequencex l \cdot \alisuccx{l'}]}

\inferrule*
{Rl=\textnormal{\sc\small RdB} \\ \alicontrolx t = \alisequencex l \cdot l' \\ \alilabelx s = l' \\\\ 
s=r := \alimemx e \\ \alivaluation{\alievalx e}t \in \alibufferx t \\ \alienabledx t}
{\alivaluation'=\alivaluation[(t,r)\mapsto \alireadbufferx t {\alivaluation{\alievalx e}t}] \\ \alicontrol'=\alicontrol[t\mapsto \alisequencex l \cdot \alisuccx{l'}]}


\inferrule*
{Rl=\textnormal{\sc\small Wr} \\ \alicontrolx t = \alisequencex l \cdot l' \\ \alilabelx s = l' \\\\
s=\alimemx e := r \\ \alienabledx t}
{\alivaluation'=\alivaluation[(t,r)\mapsto {\alivaluation \alievalx e}t] \\ \alicontrol'=\alicontrol[t\mapsto \alisequencex l \cdot \alisuccx{l'}]}

\inferrule*
{Rl=\textnormal{\sc\small WrB} \\ \alicontrolx t = \alisequencex l \cdot l' \\ \alilabelx s = l' \\\\ 
\alibufferx t = \alisequencex {b} \\ s=\alimemx e := r \\ \alienabledx t}
{\alibuffer'=\alibuffer[t\mapsto \alisequencex b \cdot(\alivaluation{\alievalx e}t,\alivaluationx {t,r})] \\ \alicontrol'=\alicontrol[t\mapsto \alisequencex l \cdot \alisuccx{l'}]}

\inferrule*
{Rl=\textnormal{\sc\small WrM} \\ \alibufferx t = (a,d)\cdot\alisequencex {b} \\\\
s=\varepsilon \\ \alienabledx t}
{\alibuffer'=\alibuffer[t\mapsto \alisequencex b] \\ \alivaluation'=\alivaluation[a\mapsto d]}

\inferrule*
{Rl=\textnormal{\sc\small WrR} \\ \alicontrolx t = \alisequencex l \cdot l' \\ \alilabelx s = l' \\\\ 
s= r := e \\ \alienabledx t}
{\alivaluation'=\alivaluation[(t,r)\mapsto \alivaluation{\alievalx e}t] \\ \alicontrol'=\alicontrol[t\mapsto \alisequencex l \cdot \alisuccx{l'}]}

\inferrule*
{Rl=\textnormal{\sc\small Fnc} \\ \alicontrolx t = \alisequencex l \cdot l' \\ \alilabelx s = l' \\\\
\alibufferx t = \varepsilon \\ s=\alifence \\ \alienabledx t}
{\alicontrol'=\alicontrol[t\mapsto \alisequencex l \cdot \alisuccx{l'}]}

\inferrule*
{Rl=\textnormal{\sc\small IfT} \\ \alicontrolx t = \alisequencex l \cdot l' \\ \alilabelx s = l' \\\\
\alivaluation{\alievalx e}t=\alitrue \\ s=\aliifelsex e {c_1} {c_2} \\ \alienabledx t}
{\alicontrol'=\alicontrol[t\mapsto \alisequencex l \cdot \alisuccx{l'} \cdot \alifirstx {c_1}]}  

\inferrule*
{Rl=\textnormal{\sc\small IfE} \\ \alicontrolx t = \alisequencex l \cdot l' \\ \alilabelx s = l' \\\\
\alivaluation{\alievalx e}t=\alifalse \\ s=\aliifelsex e {c_1} {c_2} \\ \alienabledx t}
{\alicontrol'=\alicontrol[t\mapsto \alisequencex l \cdot \alisuccx{l'} \cdot \alifirstx {c_2}]}  

\inferrule*
{Rl=\textnormal{\sc\small WhI} \\ \alicontrolx t = \alisequencex l \cdot l' \\ \alilabelx s = l' \\\\
\alivaluation{\alievalx e}t=\alitrue \\ s=\aliwhilex e {c} \\ \alienabledx t}
{\alicontrol'=\alicontrol[t\mapsto \alisequencex l \cdot l' \cdot {\alifirstx {c}}]}

\inferrule*
{Rl=\textnormal{\sc\small WhE} \\ \alicontrolx t = \alisequencex l \cdot l' \\ \alilabelx s = l' \\\\
\alivaluation{\alievalx e}t=\alifalse \\ s=\aliwhilex e {c} \\ \alienabledx t}
{\alicontrol'=\alicontrol[t\mapsto \alisequencex l \cdot \alisuccx {l'}]}

\inferrule*
{Rl=\textnormal{\sc\small Skp} \\ \alicontrolx t = \alisequencex l \cdot l' \\ \alilabelx s = l' \\\\ 
s=\aliskip \\ \alienabledx t}
{\alicontrol'=\alicontrol[t\mapsto \alisequencex l \cdot \alisuccx {l'}]}

\inferrule*
{Rl=\textnormal{\sc\small AtB} \\ \alicontrolx t = \alisequencex l \cdot l' \\ \alilabelx s = l' \\\\
s=\aliatomicx c \\ \aliexecmode = -1}
{\alicontrol'=\alicontrol[t\mapsto \alisequencex l \cdot \alisuccx {l'} \cdot \aliatomicexitlabel \cdot \alifirstx {c}] \\ \aliexecmode' = t}

\inferrule*
{Rl=\textnormal{\sc\small AtE} \\ \alicontrolx t = \alisequencex l \cdot \aliatomicexitlabel \\\\
s=\varepsilon \\ \aliexecmode=t}
{\alicontrol'=\alicontrol[t\mapsto \alisequencex l] \\ \aliexecmode' = -1}

\inferrule*
{Rl=\textnormal{\sc\small AsT} \\ \alicontrolx t = \alisequencex l \cdot l' \\ \alilabelx s = l' \\\\
\alivaluation{\alievalx e}t=\alitrue \\ s=\aliassertx{e} \\ \alienabledx t}
{\alicontrol'=\alicontrol[t\mapsto \alisequencex l \cdot \alisuccx {l'}]}

\inferrule*
{Rl=\textnormal{\sc\small AsF} \\ \alicontrolx t = \alisequencex l \cdot l' \\ \alilabelx s = l' \\\\
\alivaluation{\alievalx e}t=\alifalse \\ s=\aliassertx{e} \\ \alienabledx t}
{\aliexecmode' = -2}

\inferrule*
{Rl=\textnormal{\sc\small Asm} \\ \alicontrolx t = \alisequencex l \cdot l' \\ \alilabelx s = l' \\\\
\alivaluation{\alievalx e}t=\alitrue \\ s=\aliassumex{e} \\ \alienabledx t}
{\alicontrol'=\alicontrol[t\mapsto \alisequencex l \cdot \alisuccx {l'}]}
\end{mathpar}
\caption{Operational semantics for TSO and SC.}
\label{fig:operational-semantics}
\end{figure*}


%A {\em labelled program} is a pair $(P,\alilabel)$ consisting of a program $P$ and a {\em labelling function} $\alilabel:\aliprogstmtx P \mapsto L$ mapping each statement of $P$ to a unique {\em label} in the set of labels, $L$.
%When no confusion is likely to arise, we use $P$ to refer also to a labelled program in which case $\alilabel$ is implicit.
Two statements $s$, $s'$ in $P$ are {\em consecutive} if $s\mathtt{;}s'$ appears in some method of $P$.
In such a case, $s'$ is the {\em sequential successor} of $s$ and $s$ the {\em sequential predecessor} of $s'$, denoted as $s'=\alisuccx s$ and $s=\aliprecx {s'}$.
If $s$ has no sequential successor, we let $\alisuccx s=\varepsilon$.
A {\em code block} is any code ($C$) delimited by two matching braces.
The unique first statement in a non-empty code block is referred to as $\alifirstx C$.
Observe that if $s$ is the last statement in a code block, then $\alisuccx s=\varepsilon$.

For a given program $P$, we use a labelled transition system (LTS), denoted as $\aliltsx P=(Q,q_{init},Tr,\rightarrow)$, to define the semantics of a program under either TSO or SC.
A {\em program state} $q\in Q$ of $P$ is a tuple $(\alicontrol,\alivaluation,\alibuffer,\aliexecmode)$.
The {\em control flow} $\alicontrol:T \mapsto L^*$ maps each thread in the set of thread identifiers $T$ to some sequence over labels.
Intuitively, $\alicontrol$ encodes the execution context for each thread in the program.
If $\alicontrolx t=\alisequencex {l}$, then $\alisequencex {l} \langle 1,\alilenx{\alisequencex {l}}-1\rangle$ keeps the nesting history and $\alisequencex {l}[\alilenx{\alisequencex {l}}]$ is the label of the next statement to be executed by thread $t$.
The {\em valuation} $\alivaluation:\mathbb{N}\cup (T\times R) \rightarrow \mathbb{N}$ maps each memory location and register of each thread to its content, assumed to be a non-negative integer.
Intuitively, $\alivaluation$ holds the storage state: the contents of each register of each thread and the contents of the (single) global memory.
The contents of a memory location $i$, $\alimemx i$, is accessed by $\alivaluationx {i}$, and the content of some register $r$ used by a thread $t$ is accessed by $\alivaluationx {t,r}$.
The {\em buffer view} $\alibuffer:T \rightarrow (\mathbb{N}\times\mathbb{N})^*$ maps each thread to a sequence over pairs of natural numbers.
Intuitively, $\alibufferx t=(a_1,d_1)\ldots(a_k,d_k)$ means that the store buffer of thread $t$ contains $k$ buffered writes, the oldest being to location $a_1$ of value $d_1$ and the newest being to location $a_k$ of value $d_k$.
We use $a\in\alibufferx t$ to denote the fact that there is $j\in[1,k]$ with $a_j=a$.
In such a case, let $\alireadbufferx t a$ denote $d_j$ such that $a_j=a$ and for all $i>j$, $a_i\neq a$.
Finally, the {\em execution mode} $\aliexecmode$ is a variable ranging over $\{-1,-2\}\cup T$.
Intuitively, $\aliexecmode$ is $-1$ if no thread is currently executing an atomic block, is $t$ if thread $t$ is executing an atomic block.
When $\aliexecmode$ is set to $-2$, it denotes an assertion violation on occurrence of which all threads are blocked.

We use dot notation to access individual components of a state $q$; e.g. $q.\alicontrol$, $q.\alivaluation$.
We drop $q$ when the state is understood from the context.
The distinguished state $q_{init}$ is the {\em initial state}, such that $q_{init}.\alicontrol=\varepsilon$, $q_{init}.\alivaluation$ maps everything to 0, $q_{init}.\alibuffer=\varepsilon$, and $q_{init}.\aliexecmode=-1$.

The transition relation $\rightarrow\subseteq Q\times Tr\times Q$ of $\aliltsx P$ is given in Fig.~\ref{fig:operational-semantics}.
The label of a transition is written as $\tau=R,t:s$; it is called an $R$ transition, {\em executed by} thread $t$, and the {\em execution of} statement $s$ when $s\neq \varepsilon$.
The auxiliary predicate $\alienabled$ used in the premise of a rule determines whether a particular thread can execute.
Formally, $\alienabledx t$ is true iff $\aliexecmode\in\{-1,t\}$. 
The notation $\alivaluation \alievalx e t$ denotes the value of expression $e$ when each free variable $x$ is replaced with $\alivaluationx x$ and {\tt tid} is replaced with $t$.


\paragraph{Program runs.}
A {\em run} of program $P$ is an alternating sequence of states and transition labels conforming to the operational semantics of Fig.~\ref{fig:operational-semantics}.
Formally, a run $\alisequencex r$ is a sequence $q_0\tau_1q_1\ldots \tau_kq_k$, where $q_i$'s are states of $\aliltsx P$, $q_0=q_{init}$, and for each $i\in[1,k]$, $q_{i-1}\xrightarrow{\tau_i}q_i$ is a transition of $\aliltsx P$.
The sequence of transitions $\tau_1\ldots \tau_k$ is called the {\em trace} of the run, denoted by $\alitracex {\alisequencex r}$.
The run is {\em failed} if $q_k.\aliexecmode=-2$.
It is called {\em terminated} if $q_k.\alicontrolx t=\varepsilon$ for all $t$.
A run that is either failed or terminated is {\em complete}.

Observe that we have not constrained the set of thread identifiers in the operational semantics.
The {\em restriction} of $P$ to $T$, written $P[T]$, gives the set of all runs of $P$ in which each transition is executed by some thread $t\in T$. 

A program run is {\em TSO-compliant} if it does not contain the {\sc \small Rd, Wr} transitions.
Similarly, a program run is {\em SC-compliant} if it does not contain the {\sc \small WrB, WrM, RdB} and {\sc \small RdM} transitions.
%For notational convenience we leave {\sc \small Fnc} rules in SC compliant runs.
It is easy to show that in SC compliant runs {\alifence} and {\aliskip} are interchangeable. 
Let $\alirunsx {tso} P$ denote the set of all TSO-compliant runs of $P$.
Similarly, let $\alirunsx {sc} P$ denote the set of all SC-compliant runs of $P$.


\begin{figure*}[th]
\[
\alimemtracex {q,(Rl,t:s)} \stackrel{def}{=} 
 \begin{cases}
  \alireadx t {q.\alivaluation{\alievalx e}t} {q.\alivaluationx {q.\alivaluation{\alievalx e}t}} & ,\ Rl\in\{\textnormal{\sc\small Rd,RdM}\},\ s=r:=\alimemx e\\
%  \alireadx t {q.\alivaluation{\alievalx e}t} {q.\alivaluationx {q.\alivaluation{\alievalx e}t}} &,\ Rl=\textnormal{\sc\small RdM},\ s=r:=\alimemx e\\
  \alireadx t {q.\alivaluation{\alievalx e}t} {q.\alireadbufferx t {q.\alivaluation{\alievalx e}t}} &,\ Rl=\textnormal{\sc\small RdB},\ s=r:=\alimemx e\\
  \aliwritex t {q.\alivaluation{\alievalx e}t} {q.\alivaluationx {t,r}} &,\ Rl=\textnormal{\sc\small Wr},\ s=\alimemx e:= r\\
  \locwritex t {q.\alivaluation{\alievalx e}t} {q.\alivaluationx {t,r}} &,\ Rl=\textnormal{\sc\small WrB}\\
  \remwritex t a d &,\ Rl=\textnormal{\sc\small WrM},\ q.\alibufferx t=(a,d)\cdot \alisequencex b\\
  \alibarrierx t &,\ Rl=\textnormal{\sc \small Fnc}\\
  \varepsilon &,\ \textnormal{otherwise}
 \end{cases}
\]
\caption{Mapping transitions to memory operations.}
\label{fig:trans-to-memory}
\end{figure*}


\paragraph{Memory traces.}
In order to relate program runs to executions of the previous section, we define a mapping $\alimemtrace:\aliltsstatesx P\times\aliltstransitionsx P \rightarrow \tsoalph$ in Fig.~\ref{fig:trans-to-memory}.

Let $\alisequencex r=q_0t_1q_1\ldots t_kq_k$ be a run.
The memory trace of $\alisequencex r$ is the sequence $\alimemtracex {q_0,t_1}\cdot \alimemtracex {q_1,t_2} \cdot \ldots \cdot \alimemtracex {q_{k-1},t_k}$.
With an abuse in notation, we let $\alimemtracex {\alisequencex r}$ denote the memory trace of $\alisequencex r$.
The following result establishes the link between runs and executions.

\begin{proposition}
Let $\alisequencex r$ be a TSO (resp. SC) compliant run.
Then $\alimemtracex {\alisequencex r}$ is a TSO-execution (resp. SC-execution).
\end{proposition}
