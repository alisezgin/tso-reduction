\section{Formal Framework}
\label{sec:formal-framework}

\begin{figure}
\begin{tabular}{lcll}
action ($a$,$b$) & ::= & \locwritex t x v & (local write: insertion of the write\\
& & & of value $v$ into location $x$ by thread $t$)\\
& $|$ & \remwritex t x v & (remote write: shared memory update\\
& & & of $x$ with $v$ by $t$)\\
& $|$ & \alireadx t x v & (a read of value $v$ from $x$ by $t$)\\
& $|$ & \alibarrierx t & (memory barrier by $t$)
%& $|$ & \alilockx t & (the beginning of a \\
%& $|$ & \aliunlockx t
\end{tabular}
\caption{The alphabet $\tsoalph$ of memory actions in TSO.}
\label{fig:grammar-tso}
\end{figure}
\paragraph{Notation.} 
The symbols we are going to use and their meanings are given in Fig.~\ref{fig:grammar-tso}.
For simplicity, we omit locked operations which can be modeled by the given actions.
A local write $\locwritex t x v$ {\em matches} the remote write $\remwritex u y w$ iff $t=u$, $x=y$ and $v=w$.
We will use the notation $\tsoalph_{op,thr,loc}$ to denote the subset of $\tsoalph$ which contains all actions of type $op\in\{\locwrite, \remwrite, \aliread, \alibarrier\}$, by thread $thr$, and into location $loc$.
Omitting parameters denotes existential quantification; e.g. $\tsoalph_{-,t,-}$ is the set of all actions done by thread $t$.


Let $\mathbf{e}$ be a sequence over $\tsoalph$.
We will use indexed notation to refer to the elements in $\mathbf{e}$: $\mathbf{e}[i]$ is the $i^{th}$ action in $\mathbf{e}$.
Similarly, we let $\mathbf{e}\langle i,j\rangle$ denote the segment of $\mathbf{e}$ from $\mathbf{e}[i]$ to $\mathbf{e}[j]$ with both ends inclusive.
The length of $\mathbf{e}$ is written as $|\mathbf{e}|$ and gives the number of actions in $\mathbf{e}$.
Let $\pi$ be a permutation over $[1,k]$ and $\mathbf{e}$ be of length $k$.
We use $\pi(\mathbf{e})$ to denote the sequence $\mathbf{e}[\pi(1)]\ldots \mathbf{e}[\pi(k)]$.
Two executions $\mathbf{e}$ and $\mathbf{f}$ are {\em permutationally equivalent}, written $\mathbf{e} \aliperm \mathbf{f}$, if there is a permutation $\pi$ such that $\mathbf{e}=\pi(\mathbf{f})$.

The projection of $\mathbf{e}$ into $\tsoalph_{o,t,l}$, written as $\mathbf{e}\downarrow_{o,t,l}$, is the subsequence obtained by keeping only those actions in $\tsoalph_{o,t,l}$.
For instance, $\aliprojx {\mathbf{e}} {\locwrite,t,-}$ is the subsequence of $\mathbf{e}$ consisting of all the local writes done by $t$.


\paragraph{TSO-executions.}
Let $\mathbf{e}$ be a sequence over $\tsoalph$.
It is {\em matched} if for any $t$ there is an injection $\mu$ from $\aliprojx {\mathbf{e}} {\remwrite,t,-}$ to $\aliprojx {\mathbf{e}} {\locwrite,t,-}$ such that $\mu(a)=b$ implies that $b$ matches $a$.
A local write $\mathbf{e}[i]$ is {\em buffered at $j$} if it is matched by a remote write at position $l>j$.
We will call a matched $\mathbf{e}$ {\em complete} if there are no buffered writes at $|\mathbf{e}|$.

A TSO-execution is a matched sequence $\mathbf{e}$ over $\tsoalph$ subject to well-formedness constraints:
\begin{itemize}
\item The order of remote writes per thread respects the order of their matching local writes.
Formally, for $i<j$ such that $\mathbf{e}[i]=\locwritex t x v$ and $\mathbf{e}[j]=\locwritex t y w$ and there is $k$ with $\mu(\mathbf{e}[k])=\mathbf{e}[j]$, then there is $l$ such that $j<l<k$ and $\mu(\mathbf{e}[l])=\mathbf{e}[i]$.
\item The read values are consistent.
Formally, if $\mathbf{e}[j]=\alireadx t x v$, then either i) $\locwritex t x v$ is the most recent buffered write at $j$ by $t$ to $x$, or ii) no buffered write to $x$ by $t$ at $j$ exists and $\remwritex u x v$ is the most recent remote write, or iii) neither condition applies and $v$ is the initial value.
\item The barrier operations can only happen when the buffer is empty.
Formally, if $\mathbf{e}[j]=\alibarrierx t$ and if $\locwritex u x v$ is buffered at $j$, then $t\neq u$.
\end{itemize}



\begin{definition}[Atomic]
A local write $\mathbf{e}[i]$ is {\em atomic} if it is not buffered at $i+1$.
A complete TSO-execution $\mathbf{e}$ is called {\em atomic} if all of its local writes are atomic.
\end{definition}

Each TSO-execution $\mathbf{e}$ induces a partial order $\alipotsox {\mathbf{e}}$ over $\{\mathbf{e}[i] \mid i\in[1,|\mathbf{e}|]\}$ such that $a\alipotsox {\mathbf{e}} b$ if $a$ occurs before $b$ in $\mathbf{e}$, $a,b\in\tsoalph_{-,t,-}$ and one of the following holds:
\begin{itemize}
\item $a\in\tsoalph_{\remwrite,-,-}$ iff $b\in\tsoalph_{\remwrite,-,-}$.
\item $b=\alimatchx a$.
\end{itemize}

\paragraph{SC-executions.}
An SC-execution $\mathbf{s}$ is a sequence over the alphabet $\{\aliwritex t x v\}\cup\{\alireadx t x v\}$.
The actions of the form $\aliwritex t x v$, denoting the write of value $v$ into location $x$ by thread $t$, are called {\em write} actions.
Each SC-execution has to satisfy: if $\mathbf{s}[i]=\alireadx t x v$, then either i) there is $j<i$ with $\mathbf{s}[j]=\aliwritex u x v$ and there are no write actions to $x$ in $\mathbf{s}\langle j+1,i\rangle$, or ii) there is no write action in $\mathbf{s}\langle 1,i\rangle$ and $v$ is $\bot$.

Similar to TSO-executions, each SC-execution $\mathbf{s}$ induces a partial order $\aliposcx {\mathbf{s}}$ such that $a\aliposcx {\mathbf{s}} b$ if $a$ occurs before $b$ in $\mathbf{s}$ and both belong to the same thread.

\newcommand{\SCofTSO}{\ensuremath{\mathsf{SC}}}
\newcommand{\SCofTSOx}[1]{\ensuremath{\SCofTSO(#1)}}
\newcommand{\SCofTSOstrict}{\ensuremath{\SCofTSO_S}}
\newcommand{\SCofTSOstrictx}[1]{\ensuremath{\SCofTSOstrict(#1)}}
\newcommand{\aliconvert}{\ensuremath{\ulcorner \urcorner}}
\newcommand{\aliconvertx}[1]{\ensuremath{\ulcorner #1 \urcorner}}
\newcommand{\tsoequiv}{\ensuremath{\approx}}
\newcommand{\tsoequivstrict}{\ensuremath{\tsoequiv_S}}
\newcommand{\equivclassx}[2]{\ensuremath{[#1]_{#2}}}
\newcommand{\tighter}{\ensuremath{\sqsubseteq}}
\newcommand{\tighterstrict}{\ensuremath{\tighter_S}}
\newcommand{\tightclosure}{\ensuremath{\mathsf{T}}}
\newcommand{\tightclosurex}[1]{\ensuremath{\tightclosure(#1)}}
\newcommand{\tightclosurestrict}{\ensuremath{\tightclosure_S}}
\newcommand{\tightclosurestrictx}[1]{\ensuremath{\tightclosurestrict(#1)}}

\paragraph{Correspondence between TSO and SC.}
We define a mapping $\aliconvert$ from the actions of TSO to those of SC as follows.
\[
\aliconvertx a \stackrel{def}{=} 
 \begin{cases}
  \aliwritex t x v & \text{, if } a=\locwritex t x v\\
  \alireadx t x v & \text{, if } a=\alireadx t x v\\
  \varepsilon & \text{, if } a=\remwritex t x v\\
  \varepsilon & \text{, if } a=\alibarrierx t
 \end{cases}
\]
For a TSO-execution $\mathbf{e}$, let $\aliconvertx {\mathbf{e}}$ denote the sequence obtained by the point-wise extension of $\aliconvert$.

Let $\mathbf{e}$ be a complete TSO-execution.
We define $\SCofTSOx {\mathbf{e}}$ to be the set of all SC-executions $\mathbf{s}$ such that $\aliconvertx {\aliprojx {\mathbf{e}} {-,t,-}}=\aliprojx {\mathbf{s}} {-,t,-}$.
Informally, $\mathbf{s}$ belongs to $\SCofTSOx {\mathbf{e}}$ if it is an SC-execution and respects the thread local ordering of read and local write actions.

A particular subset of $\SCofTSOx {\mathbf{e}}$ is of interest to us.
Let $\SCofTSOstrictx {\mathbf{e}}\subseteq \SCofTSOx {\mathbf{e}}$ be the set of all $\mathbf{s}$ such that $\aliprojx {\mathbf{e}} {\remwrite,-,-} [i] = \remwritex t x v$ iff $\aliprojx {\mathbf{s}} {\aliwrite,-,-}[i] = \aliwritex t x v$.
Informally, $\mathbf{s}$ will be in $\SCofTSOstrictx {\mathbf{e}}$ if additionally the order among the write actions in $\mathbf{s}$ preserves the order among the remote write actions in $\mathbf{e}$.
We have the following result following immediately from definitions.

\begin{proposition}\label{prop:complete-tso}
For any atomic TSO-execution $\mathbf{e}$, $\aliconvertx {\mathbf{e}} \in \SCofTSOstrictx {\mathbf{e}}$.
\end{proposition}
Proposition~\ref{prop:complete-tso} implies that an atomic TSO-execution can always be transormed into an SC-execution by simply replacing each (adjacent) pair of local and remote write actions by their image under $\aliconvert$.

\paragraph{Equivalence ($\tsoequiv$) and partial order ($\tighter$) over TSO-executions.}
Two complete TSO-executions $\mathbf{e}$ and $\mathbf{f}$ are {\em equivalent}, written $\mathbf{e}\tsoequiv\mathbf{f}$, if one can be obtained from the other by moving the actions such that the per thread order is preserved.
Formally, $\mathbf{e}\tsoequiv\mathbf{f}$ if $\mathbf{e}\aliperm\mathbf{f}$ and $\aliprojx {\mathbf{e}} {-,t,-}=\aliprojx {\mathbf{f}} {-,t,-}$.
They are {\em strictly} equivalent, written $\mathbf{e}\tsoequivstrict\mathbf{f}$, if the order among the remote writes is also preserved.
Formally, $\mathbf{e}\tsoequivstrict\mathbf{f}$ if $\mathbf{e}\tsoequiv\mathbf{f}$ and $\aliprojx {\mathbf{e}} {\remwrite,-,-}=\aliprojx {\mathbf{f}} {\remwrite,-,-}$.
Whenever no confusion is likely to arise, we will use $\pi$ to denote one of the permutations between two (strictly) equivalent TSO-executions establishing their (strict) equivalence.

Atomicity induces a partial order $\tighter$ on TSO-equivalent traces.
A TSO-execution $\mathbf{e}$ is {\em tighter} than $\mathbf{f}$, written $\mathbf{e}\tighter\mathbf{f}$ if $\mathbf{e}\tsoequiv\mathbf{f}$ and whenever $\mathbf{f}[i]$ is an atomic local write, then so is $\mathbf{e}[\pi^{-1}(i)]$.
In other words, $\mathbf{e}$ is tighter than an equivalent $\mathbf{f}$ if all the atomic writes of the latter are also atomic in the former. 
The execution $\mathbf{e}$ is strictly tighter than $\mathbf{f}$, written $\mathbf{e}\tighterstrict\mathbf{f}$, if $\mathbf{e}$ is tighter than $\mathbf{f}$ and they are strictly equivalent.

Each TSO-execution thus induces a set of (strictly) tighter executions.
Let $\tightclosurex{\mathbf{e}}=\{\mathbf{f} \mid \mathbf{f}\tighter\mathbf{e}\}$; that is, the set of all TSO-executions tighter than $\mathbf{e}$.
Let $\tightclosurestrictx {\mathbf{e}}$ denote the subset of $\tightclosurex{\mathbf{e}}$ consisting of only strictly equivalent executions.

\begin{definition}[SC-like]
Let $\mathbf{e}$ be a TSO-execution.
It is called {\em observationally SC-like} if $\tightclosurex {\mathbf{e}}$ contains an atomic execution.
It is called {\em SC-like} if $\tightclosurestrictx {\mathbf{e}}$ contains an atomic execution.
\end{definition}
Since for any TSO-execution $\mathbf{e}$ we have $\tightclosurestrictx {\mathbf{e}}\subseteq \tightclosurex {\mathbf{e}}$, SC-like is a stronger property than observationally SC-like.
The terms are not arbitrarily named as the following proposition shows.

\begin{proposition}
Let $\mathbf{e}$ be a complete TSO-execution.
It is observationally SC-like iff $\tightclosurex {\mathbf{e}} \cap \SCofTSOx {\mathbf{e}} \neq \emptyset$.
It is SC-like iff $\tightclosurestrictx {\mathbf{e}} \cap \SCofTSOstrictx {\mathbf{e}} \neq \emptyset$.
\end{proposition}


\newcommand{\independent}{\ensuremath{\nleftarrow}}
\newcommand{\independentstrict}{\ensuremath{\independent_S}}
\newcommand{\rulelocal}{(\ensuremath{\mathbf{Loc}})}
%\newcommand{\ruleread}{(\ensuremath{\mathbf{Read}})}
\newcommand{\rulereml}{(\ensuremath{\mathbf{RemL}})}
\newcommand{\ruleremr}{(\ensuremath{\mathbf{RemR}})}

\newcommand{\rulethr}{(\ensuremath{\mathbf{Thr}})}
\newcommand{\rulemat}{(\ensuremath{\mathbf{Mat}})}
\newcommand{\ruleloc}{(\ensuremath{\mathbf{Loc}})}
\newcommand{\rulerem}{(\ensuremath{\mathbf{Rem}})}
\newcommand{\ruleremstrict}{(\ensuremath{\mathbf{RemS}})}
\newcommand{\aliconflict}{\ensuremath{\mathsf{Conf}}}
\newcommand{\aliconflictstrict}{\ensuremath{\aliconflict_S}}

\newcommand{\shuffleset}{\ensuremath{\mathsf{Shuff}}}
\newcommand{\shufflesetx}[1]{\ensuremath{\shuffleset(#1)}}

\paragraph{Conflict Relations.}
Two TSO actions $a$ and $b$ {\em conflict}, written $\aliconflict(a,b)$, if one of the following holds:
\begin{description}
\item[\rulerem] $a,b\in\tsoalph_{\remwrite,t,-}$.
\item[\rulethr] $a,b\in\tsoalph_{\{\locwrite,\aliread,\alibarrier\},t,-}$.
\item[\ruleloc] $a,b\in\tsoalph_{-,-,x}$ and $\{a,b\}\nsubseteq\tsoalph_{\aliread,-,-}$.
\end{description}
They {\em strictly} conflict, $\aliconflictstrict(a,b)$, if {\rulerem} is replaced with
\begin{description}
\item[\ruleremstrict] $a,b\in\tsoalph_{\remwrite,-,-}$.
\end{description}

Let $\mathbf{e}$ be a TSO-execution of length $k$.
A sequence $\mathbf{f}$ is a {\em shuffling} of $\mathbf{e}$ if $\mathbf{e}\aliperm\mathbf{f}$ and there is some $i$ such that for all $j\notin\{i,i+1\}$, $\mathbf{e}[j]=\mathbf{f}[j]$ and $\mathbf{e}[i]=\mathbf{f}[i+1]$.
The position $i$ is called the {\em pivot} (of shuffling).
A shuffling of $\mathbf{e}$ with pivot $i$ is {\em valid} if $\pi(\mathbf{e})$ is a TSO-execution and $\mathbf{e}[i]$ does not conflict with $\mathbf{e}[i+1]$.
Let $\shufflesetx {\mathbf{e}}$ denote the closure of the valid shufflings on $\mathbf{e}$.
Formally, $\shufflesetx {\mathbf{e}}$ is defined as the set satisfying
\begin{itemize}
\item $\mathbf{e}\in\shufflesetx {\mathbf{e}}$,
\item $\mathbf{f}\in\shufflesetx {\mathbf{e}}$ implies there is $\mathbf{g}\in\shufflesetx {\mathbf{e}}$ and $\mathbf{f}$ is a valid shuffling of $\mathbf{g}$.
\end{itemize}

Observe that there are two further constraints implicitly enforced by requiring a valid shuffling with pivot $i$. 
First, if $\mathbf{e}[i+1]$ is a barrier action by thread $t$ and $\mathbf{e}[i]$ is a remote write action by the same thread, then they cannot be reordered as barrier cannot occur when there is a buffered write. 
Second, if $\mathbf{e}[i]$ is a local write action by $t$ and $\mathbf{e}[i+1]$ is the matching remote write action, reordering them is not allowed since the resulting sequence will not be a TSO-execution.

The shuffling closure of $\mathbf{e}$ gives an operational under-approximation of TSO-execution equivalence $\tsoequiv$ as the following result indicates.

\begin{proposition}
For complete TSO-execution $\mathbf{e}$, $\shufflesetx {\mathbf{e}} \subseteq \equivclassx {\mathbf{e}} {\tsoequiv}$.
\end{proposition}


\newcommand{\aliprog}{\ensuremath{\mathtt{prog}}}
\newcommand{\alimem}{\ensuremath{\mathtt{mem}}}
\newcommand{\alimemx}[1]{\ensuremath{\alimem[#1]}}
\newcommand{\alifence}{\ensuremath{\mathtt{fence}}}
\newcommand{\aliassume}{\ensuremath{\mathtt{assume}}\xspace}
\newcommand{\aliassumex}[1]{\ensuremath{\mathtt{assume}\ #1}}
\newcommand{\aliassert}{\ensuremath{\mathtt{assert}}\xspace}
\newcommand{\aliassertx}[1]{\ensuremath{\mathtt{assert}\ #1}}
\newcommand{\alireg}{\ensuremath{\mathtt{r}}}
\newcommand{\aliregx}[1]{\ensuremath{\alireg[#1]}}
\newcommand{\aliif}{\ensuremath{\mathtt{if}}}
\newcommand{\alithen}{\ensuremath{\mathtt{then}}}
\newcommand{\alielse}{\ensuremath{\mathtt{else}}}
\newcommand{\aliifelsex}[3]{\ensuremath{\aliif\ #1\ \alithen\ \{#2\}\ \alielse\ \{#3\}}}
\newcommand{\aliwhile}{\ensuremath{\mathtt{while}}}
\newcommand{\aliwhilex}[2]{\ensuremath{\aliwhile(#1)\ \{#2\}}}
\newcommand{\aliatomic}{\ensuremath{\mathtt{atomic}}}
\newcommand{\aliatomicx}[1]{\ensuremath{\aliatomic\{#1\}}}
\newcommand{\aliskip}{\ensuremath{\mathtt{skip}}}
\newcommand{\alitid}{\ensuremath{\mathtt{tid}}}




\begin{figure}
\begin{tabular}{lcl}
$P$ & ::= & $M,P \mid \varepsilon$\\
$M$ ($m$,$m',\ldots$) & ::= & $T\ N(T)\ \{ C \}$\\
$C$ ($c$,$c',\ldots$) & ::= & $S$ ; $C \mid \varepsilon$\\
$S$ ($s$,$s',\ldots$) & ::= & $R := \alimemx {E} \mid \alimemx {E} := R \mid$\\
& &  $R := E \mid \alifence \mid$\\
& & $\aliifelsex {E} {C} {C}\mid$\\ 
& & $\aliwhilex E C\mid \aliatomicx {C}$\\
& & $\aliassume\ E \mid \aliassert\ E \mid \aliskip$\\
$T$ & ::= & $\langle Bool\rangle \mid \langle Int\rangle$\\
$N$ & ::= & $\langle Name\rangle$\\
%$L$ & ::= & $\langle Label\rangle$\\
$R$ ($r$,$r',\ldots$) & ::= & $\aliregx i$ ($i\in\mathbb{N}$)\\
$E$ ($e$,$e',\ldots$) & ::= & $R \mid i \mid N(T) \mid \alitid \mid E+E \mid E-E \mid \ldots$\\
& & $E\wedge E \mid E\vee E \mid \neg E \mid \ldots $
\end{tabular}
\label{fig:program-grammar}
\caption{A simple programming language.}
\end{figure}

\newcommand{\aliprogstmt}{\ensuremath{\mathsf{Stmt}}}
\newcommand{\aliprogstmtx}[1]{\ensuremath{\aliprogstmt (#1)}}

\subsection{Programming Language}
\label{subsec:programming-language}
In this subsection, we formalize the programming language we are going to use in the rest of the paper.

\paragraph{Syntax.}
We use a simple programming language, whose grammar is given in Fig.~\ref{fig:program-grammar}.
The nonterminal $R$, ranged over by $r$, refers to {\em registers} and are used to model thread local address space.
The nonterminal $E$, ranged over by $e$, refers to well-formed {\em expressions} that can be written using registers, mathematical and logical operators, and the return values of procedures.
The shared data space is mapped by $\alimemx e$, where $e$ is an expression which evaluates to $\mathbb{N}$.
A 
Instructions can be used to read from memory ($r := \alimemx e$), update the contents of a memory location ($\alimemx e := e'$), empty the store buffer (\alifence), assign a value to a register ($r := e$).
The model control flow, we have the usual branching ($\aliifelsex e {e_1} {e_2}$) and looping ($\aliwhilex e$) instructions.
Additionally, we will also use an explicit blocking instruction ($\aliassume\ e$), which blocks as long as $e$ evaluates to $\alifalse$.
Finally, the instruction ($\aliassert\ e$) is used to claim that $e$ should hold whenever this instruction can execute.
Let $\aliprogstmtx P$ denote the set of statements used in program $P$.

\newcommand{\alilabel}{\ensuremath{\mathsf{Lab}}}
\newcommand{\alilabelx}[1]{\ensuremath{\alilabel(#1)}}
\newcommand{\alicontrol}{\ensuremath{\mathsf{Ctrl}}}
\newcommand{\alicontrolx}[1]{\ensuremath{\alicontrol(#1)}}
\newcommand{\alivaluation}{\ensuremath{\mathsf{Val}}}
\newcommand{\alivaluationx}[1]{\ensuremath{\alivaluation(#1)}}
\newcommand{\alisucc}{\ensuremath{\mathsf{Succ}}}
\newcommand{\alisuccx}[1]{\ensuremath{\alisucc(#1)}}
\newcommand{\aliprec}{\ensuremath{\mathsf{Prec}}}
\newcommand{\aliprecx}[1]{\ensuremath{\aliprec(#1)}}
\newcommand{\alifirst}{\ensuremath{\mathsf{Fst}}}
\newcommand{\alifirstx}[1]{\ensuremath{\alifirst(#1)}}
\newcommand{\alisequencex}[1]{\ensuremath{\vec{#1}}}
\newcommand{\alieval}{\ensuremath{\llbracket \rrbracket}}
\newcommand{\alievalx}[1]{\ensuremath{\llbracket #1 \rrbracket}}
\newcommand{\alibuffer}{\ensuremath{\mathsf{Buf}}}
\newcommand{\alibufferx}[1]{\ensuremath{\alibuffer(#1)}}
\newcommand{\alireadbuffer}{\ensuremath{\mathsf{RdBuf}}}
\newcommand{\alireadbufferx}[2]{\ensuremath{\alireadbuffer(#1,#2)}}
\newcommand{\aliatomiclock}{\ensuremath{\mathsf{lck}}}
\newcommand{\aliatomicexitlabel}{\ensuremath{l_{x}}}
\newcommand{\alienabled}{\ensuremath{\mathsf{Enb}}}
\newcommand{\alienabledx}[1]{\ensuremath{\alienabled(#1)}}
\newcommand{\alilts}{\ensuremath{\mathnormal{LTS}}}
\newcommand{\aliltsx}[1]{\ensuremath{\alilts(#1)}}
\newcommand{\alitrace}{\ensuremath{\mathsf{Tr}}}
\newcommand{\alitracex}[1]{\ensuremath{\alitrace(#1)}}
\newcommand{\alimemtrace}{\ensuremath{\mathsf{Mem}}}
\newcommand{\alimemtracex}[1]{\ensuremath{\alimemtrace(#1)}}
\newcommand{\aliosrules}{\ensuremath{Rules}}

\begin{figure*}[th]
\begin{mathpar}
\fbox{$(\alicontrol,\alivaluation,\alibuffer,\aliatomiclock)\xrightarrow{Rl,t:s}(\alicontrol',\alivaluation',\alibuffer',\aliatomiclock')$}\\

\inferrule*
{Rl=\textnormal{\sc\small Init} \\ M_i\in P \\ M_i= n(a)\ \{C\} \\ \alicontrolx t =\varepsilon}
{\alivaluation[(t,\alireg_{in})\mapsto a] \\ \alicontrol[t\mapsto \alifirstx C]}

\inferrule*
{Rl=\textnormal{\sc\small Rd} \\ \alicontrolx t = \alisequencex l \cdot l' \\ \alilabelx s = l' \\\\
s=r:=\alimemx e \\ \alienabledx t}
{\alivaluation[(t,r)\mapsto \alivaluationx {\alivaluation{\alievalx e}t}] \\ \alicontrol[t\mapsto \alisequencex l \cdot \alisuccx{l'}]}

\inferrule*
{Rl=\textnormal{\sc\small RdM} \\ \alicontrolx t = \alisequencex l \cdot l' \\ \alilabelx s = l' \\\\ 
s=r := \alimemx e \\ \alivaluation{\alievalx e}t \notin \alibufferx t \\ \alienabledx t}
{\alivaluation[(t,r)\mapsto \alivaluationx {\alivaluation{\alievalx e}t}] \\ \alicontrol[t\mapsto \alisequencex l \cdot \alisuccx{l'}]}

\inferrule*
{Rl=\textnormal{\sc\small RdB} \\ \alicontrolx t = \alisequencex l \cdot l' \\ \alilabelx s = l' \\\\ 
s=r := \alimemx e \\ \alivaluation{\alievalx e}t \in \alibufferx t \\ \alienabledx t}
{\alivaluation[(t,r)\mapsto \alireadbufferx t {\alivaluation{\alievalx e}t}] \\ \alicontrol[t\mapsto \alisequencex l \cdot \alisuccx{l'}]}


\inferrule*
{Rl=\textnormal{\sc\small Wr} \\ \alicontrolx t = \alisequencex l \cdot l' \\ \alilabelx s = l' \\\\
s=\alimemx e := r \\ \alienabledx t}
{\alivaluation[(t,r)\mapsto {\alivaluation \alievalx e}t] \\ \alicontrol[t\mapsto \alisequencex l \cdot \alisuccx{l'}]}

\inferrule*
{Rl=\textnormal{\sc\small WrB} \\ \alicontrolx t = \alisequencex l \cdot l' \\ \alilabelx s = l' \\\\ 
s=\alimemx e := r \\ \alibufferx t = \alisequencex {b} \\ \alienabledx t}
{\alibuffer[t\mapsto \alisequencex b \cdot(\alivaluation{\alievalx e}t,\alivaluationx {t,r})] \\ \alicontrol[t\mapsto \alisequencex l \cdot \alisuccx{l'}]}

\inferrule*
{Rl=\textnormal{\sc\small WrM} \\ \alibufferx t = (a,d)\cdot\alisequencex {b} \\ \alienabledx t}
{\alibuffer[t\mapsto \alisequencex b] \\ \alivaluation[a\mapsto d]}

\inferrule*
{Rl=\textnormal{\sc\small WrR} \\ \alicontrolx t = \alisequencex l \cdot l' \\ \alilabelx s = l' \\\\ 
s= r := e \\ \alienabledx t}
{\alivaluation[(t,r)\mapsto \alivaluation{\alievalx e}t] \\ \alicontrol[t\mapsto \alisequencex l \cdot \alisuccx{l'}]}

\inferrule*
{Rl=\textnormal{\sc\small Fnc} \\ \alicontrolx t = \alisequencex l \cdot l' \\ \alilabelx s = l' \\\\
s=\alifence \\ \alibufferx t = \varepsilon \\ \alienabledx t}
{\alicontrol[t\mapsto \alisequencex l \cdot \alisuccx{l'}]}

\inferrule*
{Rl=\textnormal{\sc\small IfT} \\ \alicontrolx t = \alisequencex l \cdot l' \\ \alilabelx s = l' \\\\
s=\aliifelsex e {c_1} {c_2} \\ \alivaluation{\alievalx e}t=\alitrue \\ \alienabledx t}
{\alicontrol[t\mapsto \alisequencex l \cdot \alisuccx{l'} \cdot \alifirstx {c_1}]}  

\inferrule*
{Rl=\textnormal{\sc\small IfE} \\ \alicontrolx t = \alisequencex l \cdot l' \\ \alilabelx s = l' \\\\
s=\aliifelsex e {c_1} {c_2} \\ \alivaluation{\alievalx e}t=\alifalse \\ \alienabledx t}
{\alicontrol[t\mapsto \alisequencex l \cdot \alisuccx{l'} \cdot \alifirstx {c_2}]}  

\inferrule*
{Rl=\textnormal{\sc\small WhI} \\ \alicontrolx t = \alisequencex l \cdot l' \\ \alilabelx s = l' \\\\
s=\aliwhilex e {c} \\ \alivaluation{\alievalx e}t=\alitrue \\ \alienabledx t}
{\alicontrol[t\mapsto \alisequencex l \cdot l' \cdot {\alifirstx {c}}]}

\inferrule*
{Rl=\textnormal{\sc\small WhE} \\ \alicontrolx t = \alisequencex l \cdot l' \\ \alilabelx s = l' \\\\
s=\aliwhilex e {c} \\ \alivaluation{\alievalx e}t=\alifalse \\ \alienabledx t}
{\alicontrol[t\mapsto \alisequencex l \cdot \alisuccx {l'}]}

\inferrule*
{Rl=\textnormal{\sc\small Skp} \\ \alicontrolx t = \alisequencex l \cdot l' \\ \alilabelx s = l' \\ s=\aliskip}
{\alicontrol[t\mapsto \alisequencex l \cdot \alisuccx {l'}]}

\inferrule*
{Rl=\textnormal{\sc\small AtB} \\ \alicontrolx t = \alisequencex l \cdot l' \\ \alilabelx s = l' \\\\
s=\aliatomicx c \\ \aliatomiclock = -1}
{\alicontrol[t\mapsto \alisequencex l \cdot \alisuccx {l'} \cdot \aliatomicexitlabel \cdot \alifirstx {c}] \\ \aliatomiclock = t}

\inferrule*
{Rl=\textnormal{\sc\small AtE} \\ \alicontrolx t = \alisequencex l \cdot \aliatomicexitlabel \\ \aliatomiclock=t}
{\alicontrol[t\mapsto \alisequencex l] \\ \aliatomiclock = -1}

\inferrule*
{Rl=\textnormal{\sc\small AsT} \\ \alicontrolx t = \alisequencex l \cdot l' \\ \alilabelx s = l' \\\\
s=\aliassertx{e} \\ \alivaluation{\alievalx e}t=\alitrue \\ \alienabledx t}
{\alicontrol[t\mapsto \alisequencex l \cdot \alisuccx {l'}]}

\inferrule*
{Rl=\textnormal{\sc\small AsF} \\ \alicontrolx t = \alisequencex l \cdot l' \\ \alilabelx s = l' \\\\
s=\aliassertx{e} \\ \alivaluation{\alievalx e}t=\alifalse \\ \alienabledx t}
{\aliatomiclock = -2}

\inferrule*
{Rl=\textnormal{\sc\small Asm} \\ \alicontrolx t = \alisequencex l \cdot l' \\ \alilabelx s = l' \\\\
s=\aliassumex{e} \\ \alivaluation{\alievalx e}t=\alitrue \\ \alienabledx t}
{\alicontrol[t\mapsto \alisequencex l \cdot \alisuccx {l'}]}
\end{mathpar}
\caption{Operational semantics for TSO and SC.}
\label{fig:operational-semantics}
\end{figure*}


\newcommand{\aliltsstates}{\ensuremath{Q}}
\newcommand{\aliltsstatesx}[1]{\ensuremath{\aliltsstates^{#1}}}
\newcommand{\aliltstransitions}{\ensuremath{Tr}}
\newcommand{\aliltstransitionsx}[1]{\ensuremath{\aliltstransitions^{#1}}}


\paragraph{Operational Semantics.}
A {\em labelled program} is a pair $(P,\alilabel)$ consisting of a program $P$ and a {\em labelling function} $\alilabel:\aliprogstmtx P \mapsto L$ mapping each statement of $P$ to a unique {\em label} in the set of labels, $L$.
When no confusion is likely to arise, we use $P$ to refer also to a labelled program in which case $\alilabel$ is implicit.
We use a labelled transition system (LTS) to define the semantics of a labelled program under either TSO or SC.

Two statements $s$, $s'$ in $P$ are {\em consecutive} if $s;s'$ appears in $P$.
In such a case, $s'$ is the {\em sequential successor} of $s$ and $s$ the {\em sequential predecessor} of $s'$, denoted as $s'=\alisuccx s$ and $s=\aliprecx {s'}$.
If $s$ has no sequential successor, we let $\alisuccx s=\varepsilon$.
A {\em code block} is any code ($C$) delimited by two matching braces.
The unique first statement in a non-empty code block is referred to as $\alifirstx C$.
Observe that if $s$ is the last statement in a code block, then $\alisuccx s=\varepsilon$.

A {\em program state} of $P$ is a tuple $(\alicontrol,\alivaluation,\alibuffer,\aliatomiclock)$.
The {\em control flow} $\alicontrol:T \mapsto L^*$ maps each thread in the set of thread identifiers $T$ to some sequence over labels.
Intuitively, $\alicontrol$ encodes the execution context for each thread in the program.
If $\alicontrolx t=\alisequencex {l}$, then $\alisequencex {l} \langle 1,|\alisequencex {l}|-1\rangle$ keeps the nesting history and $\alisequencex {l}[|\alisequencex {l}|]$ is the label of the next statement to be executed by thread $t$.
The {\em valuation} $\alivaluation:(\mathbb{N}\cup T\times R)\mapsto \mathbb{N}$ maps each memory location and register of each thread to its content, assumed to be a non-negative integer.
Intuitively, $\alivaluation$ holds the storage state: the contents of each register of each thread and the contents of the (single) global memory.
The contents of a memory location $i$, $\alimemx i$, is accessed by $\alivaluationx {i}$, and the content of some register $r$ used by a thread $t$ is accessed by $\alivaluationx {t,r}$.
The {\em buffer view} $\alibuffer:T \mapsto (\mathbb{N}\times\mathbb{N})^*$ maps each thread to a sequence over pairs of natural numbers.
Intuitively, $\alibufferx t=(a_1,d_1)\ldots(a_k,d_k)$ means that the store buffer of thread $t$ contains $k$ buffered writes, the oldest being to location $a_1$ of value $d_1$ and the newest being to location $a_k$ of value $d_k$.
We use $a\in\alibufferx t$ to denote the fact that there is $j\in[1,k]$ with $a_j=a$.
In such a case, let $\alireadbufferx t a$ denote $d_j$ such that $a_j=a$ and for all $i>j$, $a_i\neq a$.
Finally, the {\em lock state} $\aliatomiclock$ is a variable ranging over $\{-1,-2\}\cup T$.
Intuitively, $\aliatomiclock$ is $-1$ if no thread is currently executing an atomic block, is $t$ if thread $t$ is executing an atomic block.
When $\aliatomiclock$ is set to $-2$, it denotes an assertion violation on occurrence of which all threads are blocked.
The transition relation of program $P$ is given in Fig.~\ref{fig:operational-semantics}.

\newcommand{\aliruns}{\ensuremath{\mathbf{R}}}
\newcommand{\alirunsx}[2]{\ensuremath{\aliruns_{#1}(#2)}}


\paragraph{Program runs.}
A {\em run} of program $P$ is an alternating sequence of states and transition labels conforming to the operational semantics of Fig.~\ref{fig:operational-semantics}.
Formally, a run $\alisequencex r$ is a sequence $q_0t_1q_1\ldots t_kq_k$, where $q_i$'s are states of $\aliltsx P$ and for each $i\in[1,k]$, $q_{i-1}\xrightarrow{t_i}q_i$ is a transition of $\aliltsx P$.
The sequence of transitions $t_1\ldots t_k$ is called the {\em trace} of the run, denoted by $\alitracex {\alisequencex r}$.

A program run is {\em TSO compliant} if it does not contain the {\sc \small Wr} transition.
Similarly, a program run is {\em SC compliant} if it does not the {\sc \small WrB, WrM} transitions.
For notational convenience we leave {\sc \small Fnc} rules in SC compliant runs.
It is easy to show that in SC compliant runs {\alifence} and {\aliskip} are interchangeable. 
Let $\alirunsx {tso} P$ denote the set of all TSO compliant runs of $P$.
Similarly, let $\alirunsx {sc} P$ denote the set of all SC compliant runs of $P$.

\begin{figure*}[th]
\[
\alimemtracex {q,(Rl,t:s)} \stackrel{def}{=} 
 \begin{cases}
  \alireadx t {q.\alivaluation{\alievalx e}t} {q.\alivaluationx {q.\alivaluation{\alievalx e}t}} & ,\ Rl\in\{\textnormal{\sc\small Rd,RdM}\},\ s=r:=\alimemx e\\
%  \alireadx t {q.\alivaluation{\alievalx e}t} {q.\alivaluationx {q.\alivaluation{\alievalx e}t}} &,\ Rl=\textnormal{\sc\small RdM},\ s=r:=\alimemx e\\
  \alireadx t {q.\alivaluation{\alievalx e}t} {q.\alireadbufferx t {q.\alivaluation{\alievalx e}t}} &,\ Rl=\textnormal{\sc\small RdB},\ s=r:=\alimemx e\\
  \aliwritex t {q.\alivaluation{\alievalx e}t} {q.\alivaluationx {t,r}} &,\ Rl=\textnormal{\sc\small Wr},\ s=\alimemx e:= r\\
  \locwritex t {q.\alivaluation{\alievalx e}t} {q.\alivaluationx {t,r}} &,\ Rl=\textnormal{\sc\small WrB}\\
  \remwritex t a d &,\ Rl=\textnormal{\sc\small WrM},\ q.\alibufferx t=(a,d)\cdot \alisequencex b\\
  \alibarrierx t &,\ Rl=\textnormal{\sc \small Fnc}\\
  \varepsilon &,\ \textnormal{otherwise}
 \end{cases}
\]
\caption{Mapping transitions to memory operations.}
\label{fig:trans-to-memory}
\end{figure*}


\paragraph{Memory traces.}
In order to relate program runs to executions of the previous section, we define a mapping $\alimemtrace:\aliltsstatesx P\times\aliltstransitionsx P \rightarrow \tsoalph$ in Fig.~\ref{fig:trans-to-memory}.

Let $\alisequencex r=q_0t_1q_1\ldots t_kq_k$ be a run.
The memory trace of $\alisequencex r$ is the sequence $\alimemtracex {q_0,t_1}\cdot \alimemtracex {q_1,t_2} \cdot \ldots \cdot \alimemtracex {q_{k-1},t_k}$.
With an abuse in notation, we let $\alimemtracex {\alisequencex r}$ denote the memory trace of $\alisequencex r$.
The following result establishes the link between runs and executions.

\begin{proposition}
Let $\alisequencex r$ be a TSO (resp. SC) compliant run.
Then $\alimemtracex {\alisequencex r}$ is a TSO-execution (resp. SC-execution).
\end{proposition}
